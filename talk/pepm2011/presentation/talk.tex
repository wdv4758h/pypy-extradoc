\documentclass[utf8x]{beamer}

% This file is a solution template for:

% - Talk at a conference/colloquium.
% - Talk length is about 20min.
% - Style is ornate.

\mode<presentation>
{
  \usetheme{Warsaw}
  % or ...

  %\setbeamercovered{transparent}
  % or whatever (possibly just delete it)
}


\usepackage[english]{babel}
\usepackage{listings}

\usepackage[utf8x]{inputenc}
% or whatever

% Or whatever. Note that the encoding and the font should match. If T1
% does not look nice, try deleting the line with the fontenc.


\title{Allocation Removal by Partial Evaluation in a Tracing JIT}

\author[Bolz et. al.]{\emph{Carl Friedrich Bolz}\inst{1} \and Antonio Cuni\inst{1} \and Maciej Fijałkowski\inst{2} \and Michael Leuschel\inst{1} \and \\
            Samuele Pedroni\inst{3} \and Armin Rigo\inst{1}}
\author{\emph{Carl Friedrich Bolz} \and Antonio Cuni \and Maciej Fijałkowski \and Michael Leuschel \and Samuele Pedroni \and Armin Rigo}
% - Give the names in the same order as the appear in the paper.
% - Use the \inst{?} command only if the authors have different
%   affiliation.

\institute[Heinrich-Heine-Universität Düsseldorf]
{Heinrich-Heine-Universität Düsseldorf, STUPS Group, Germany \and

 merlinux GmbH, Hildesheim, Germany \and

 Open End, Göteborg, Sweden \and
}

\date{2011 Workshop on Partial Evaluation and Program Manipulation, January 24, 2011}
% - Either use conference name or its abbreviation.
% - Not really informative to the audience, more for people (including
%   yourself) who are reading the slides online


% If you have a file called "university-logo-filename.xxx", where xxx
% is a graphic format that can be processed by latex or pdflatex,
% resp., then you can add a logo as follows:




% Delete this, if you do not want the table of contents to pop up at
% the beginning of each subsection:
%\AtBeginSubsection[]
%{
%  \begin{frame}<beamer>
%    \frametitle{Outline}
%    \tableofcontents[currentsection,currentsubsection]
%  \end{frame}
%}


% If you wish to uncover everything in a step-wise fashion, uncomment
% the following command: 

%\beamerdefaultoverlayspecification{<+->}


\begin{document}

\begin{frame}
  \titlepage
\end{frame}

%\begin{frame}
%  \frametitle{Outline}
%  \tableofcontents
  % You might wish to add the option [pausesections]
%\end{frame}


% Structuring a talk is a difficult task and the following structure
% may not be suitable. Here are some rules that apply for this
% solution: 

% - Exactly two or three sections (other than the summary).
% - At *most* three subsections per section.
% - Talk about 30s to 2min per frame. So there should be between about
%   15 and 30 frames, all told.

% - A conference audience is likely to know very little of what you
%   are going to talk about. So *simplify*!
% - In a 20min talk, getting the main ideas across is hard
%   enough. Leave out details, even if it means being less precise than
%   you think necessary.
% - If you omit details that are vital to the proof/implementation,
%   just say so once. Everybody will be happy with that.

\begin{frame}
  \frametitle{Dynamic Languages are Slow}
  \begin{itemize}
      \item Interpretation overhead
      \item Constant type dispatching
      \item Boxing of primitive types
      \pause
      \begin{itemize}
          \item A lot of allocation of short-lived objects
      \end{itemize}
  \end{itemize}
\end{frame}

\begin{frame}
  \frametitle{Dynamic Languages are Slow: Example}
  Evaluate \texttt{x = a + b; y = x + c} in an interpreter:
  \pause
  \begin{enumerate}
      \item What's the type of a? \texttt{Integer}
      \item What's the type of b? \texttt{Integer}
  \pause
      \item unbox a
      \item unbox b
      \item compute the sum
      \item box the result
      \item store into x
  \pause
      \item What's the type of x? \texttt{Integer}
      \item What's the type of c? \texttt{Integer}
  \pause
      \item unbox x
      \item unbox c
      \item compute the sum
      \item box the result
      \item store into y
  \end{enumerate}
\end{frame}

\begin{frame}
  \frametitle{What to do?}
  \begin{itemize}
      \item Hard to improve in an interpreter
      \item Use a JIT compiler
      \item \textbf{Add a optimization that can deal with heap operations}
  \end{itemize}
\end{frame}

\begin{frame}
  \frametitle{Context: The PyPy Project}
  A general environment for implementing dynamic languages
  \pause
  \begin{block}{Approach}
      \begin{itemize}
          \item write an interpreter for the language in RPython
          \item compilable to an efficient C-based VM
          \pause
          \item (RPython is a restricted subset of Python)
      \end{itemize}
  \end{block}
\end{frame}

\begin{frame}
  \frametitle{PyPy's Tracing JIT}
  the feature that makes PyPy interesting:
  \begin{itemize}
      \item a meta-JIT, applicable to many languages
      \item needs a few source-code hints (or user annotations) in the interpreter
      \item JIT is a tracing JIT compiler
  \end{itemize}
\end{frame}

\begin{frame}
  \frametitle{Tracing JITs}
  \begin{itemize}
      \item VM contains both an interpreter and the tracing JIT compiler
      \item JIT works by observing and logging what the interpreter does
      \item for interesting, commonly executed code paths
      \item produces a linear list of operations (trace)
      \item trace is optimized turned into machine code
  \end{itemize}
\end{frame}

\begin{frame}[containsverbatim]
  \frametitle{Example Trace}
  Trace of \texttt{x = a + b; y = x + c}:
\begin{verbatim}
guard_class(a, Integer)
guard_class(b, Integer)
i1 = get(a, intval)
i2 = get(b, intval)
i3 = int_add(i1, i2)
x = new(Integer)
set(x, intval, i3)
\end{verbatim}
\pause
\begin{verbatim}
guard_class(x, Integer)
guard_class(c, Integer)
i4 = get(x, intval)
i5 = get(c, intval)
i6 = int_add(i4, i5)
y = new(Integer)
set(y, intval, i6)
\end{verbatim}
\end{frame}

\begin{frame}
  \frametitle{Tracing JIT: Advantages}
  \begin{itemize}
      \item Traces are interesting linear pieces of code
      \item most of the time correspond to loops
      \item everything called in the trace is inlined
      \item can perform good optimizations on the loop
  \end{itemize}
\end{frame}

\begin{frame}
  \frametitle{Optimizing the Heap Operations in a Trace}
  \begin{itemize}
      \item Contribution of our paper
      \item A simple, efficient and effective optimization of heap operations in a trace
      \item using online partial evaluation
  \end{itemize}
\end{frame}


\begin{frame}
  \frametitle{Heap Model}
  \includegraphics[scale=0.9]{figures/heap01}
\end{frame}

\begin{frame}
  \frametitle{Heap Model}
  \includegraphics[scale=0.9]{figures/heap02}
\end{frame}

\begin{frame}
  \frametitle{Heap Model}
  \includegraphics[scale=0.9]{figures/heap03}
\end{frame}

\begin{frame}[plain]
  \frametitle{Operations: New}
  \includegraphics[scale=0.8]{figures/new01}
\end{frame}

\begin{frame}[plain]
  \frametitle{Operations: New}
  \includegraphics[scale=0.8]{figures/new02}
\end{frame}

\begin{frame}[plain]
  \frametitle{Operations: Get}
  \includegraphics[scale=0.8]{figures/get01}
\end{frame}

\begin{frame}[plain]
  \frametitle{Operations: Get}
  \includegraphics[scale=0.8]{figures/get02}
\end{frame}

\begin{frame}[plain]
  \frametitle{Operations: Set}
  \includegraphics[scale=0.8]{figures/set01}
\end{frame}

\begin{frame}[plain]
  \frametitle{Operations: Set}
  \includegraphics[scale=0.8]{figures/set02}
\end{frame}

\begin{frame}[plain]
  \frametitle{Operations: Guard}
  \includegraphics[scale=0.8]{figures/guard01}
\end{frame}

\begin{frame}[plain]
  \frametitle{Operations: Guard}
  \includegraphics[scale=0.8]{figures/guard02}
\end{frame}

\begin{frame}[plain]
  \frametitle{Operations: Guard}
  \includegraphics[scale=0.8]{figures/guard03}
\end{frame}

\begin{frame}[plain]
  \frametitle{Operations: Guard}
  \includegraphics[scale=0.8]{figures/guard04}
\end{frame}

\begin{frame}
  \frametitle{Conclusion}
  \begin{itemize}
      \item straightforward interpreter can be efficient, given enough technology
      \item successful application of partial evaluation
      \item Prolog can benefit from dynamic compilation
  \end{itemize}
  \pause
  \begin{block}{Future}
      \begin{itemize}
          \item Scale up to larger programs
          \item need some optimization on the interpreter level – indexing
          \item investigate memory usage
      \end{itemize}
  \end{block}
\end{frame}


\end{document}


