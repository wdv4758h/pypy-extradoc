%% LyX 2.1.0 created this file.  For more info, see http://www.lyx.org/.
%% Do not edit unless you really know what you are doing.
\documentclass{article}
\usepackage[T1]{fontenc}
\usepackage[utf8]{inputenc}
\synctex=-1
\usepackage{color}
\usepackage{amsmath}
\usepackage{amssymb}
\usepackage{fixltx2e}
\usepackage{graphicx}
\usepackage[unicode=true,pdfusetitle,
 bookmarks=true,bookmarksnumbered=false,bookmarksopen=false,
 breaklinks=false,pdfborder={0 0 1},backref=false,colorlinks=false]
 {hyperref}

\makeatletter
%%%%%%%%%%%%%%%%%%%%%%%%%%%%%% Textclass specific LaTeX commands.
\usepackage{enumitem}		% customizable list environments
\newlength{\lyxlabelwidth}      % auxiliary length

%%%%%%%%%%%%%%%%%%%%%%%%%%%%%% User specified LaTeX commands.
% IEEE standard conference template; to be used with:
%   spconf.sty  - LaTeX style file, and
%   IEEEbib.bst - IEEE bibliography style file.
% --------------------------------------------------------------------------

\usepackage{spconf}
\usepackage{multicol}

% bold paragraph titles
\newcommand{\mypar}[1]{{\bf #1.}}

\newcommand{\mynote}[2]{%
  \textcolor{red}{%
    \fbox{\bfseries\sffamily\scriptsize#1}%
    {\small$\blacktriangleright$\textsf{\emph{#2}}$\blacktriangleleft$}%
  }%
}
\newcommand\remi[1]{\mynote{Remi}{#1}}
\newcommand\cfbolz[1]{\mynote{cfbolz}{#1}}

% Title.
% ------
\title{Virtual Memory Assisted Transactional Memory for Dynamic Languages}
%
% Single address.
% ---------------
%\name{Markus P\"uschel\thanks{The author thanks Jelena Kovacevic. This paper
%is a modified version of the template she used in her class.}}
%\address{Department of Computer Science\\ ETH Z\"urich\\Z\"urich, Switzerland}

% For example:
% ------------
%\address{School\\
%		 Department\\
%		 Address}
%
% Two addresses (uncomment and modify for two-address case).
% ----------------------------------------------------------
\twoauthors
    {Remigius Meier}
    {Department of Computer Science\\
      ETH Zürich, Switzerland\\
      \nolinkurl{remi.meier@inf.ethz.ch}}
    {Armin Rigo}
    {www.pypy.org\\
      \nolinkurl{arigo@tunes.org}}

% nice listings
\usepackage{xcolor}
\usepackage{newverbs}

\usepackage{color}
\definecolor{verylightgray}{rgb}{0.93,0.93,0.93}
\definecolor{darkblue}{rgb}{0.2,0.2,0.6}
\definecolor{commentgreen}{rgb}{0.25,0.5,0.37}
\usepackage{letltxmacro}

\usepackage{listings}
\makeatletter
\LetLtxMacro{\oldlstinline}{\lstinline}

\renewcommand\lstinline[1][]{%
  \Collectverb{\@@myverb}%
}

\def\@@myverb#1{%
    \begingroup
    \fboxsep=0.2em
    \colorbox{verylightgray}{\oldlstinline|#1|}%
    \endgroup
}
\makeatother



\usepackage{listings}

\lstset{backgroundcolor={\color{verylightgray}},
  basicstyle={\scriptsize\ttfamily},
  commentstyle={\ttfamily\color{commentgreen}},
  keywordstyle={\bfseries\color{darkblue}},
  morecomment={[l]{//}},
  morekeywords={foreach,in,def,type,dynamic,Int,
    Boolean,infer,void,super,if,boolean,int,else,
    while,do,extends,class,assert,for,switch,case,
    private,protected,public,const,final,static,
    interface,new,true,false,null,return}}
\renewcommand{\lstlistingname}{Listing}

\begin{document}
\ninept
\maketitle
\begin{abstract}
  ...
\end{abstract}

\section{Introduction}


Dynamic languages like Python, PHP, Ruby, and JavaScript are usually
regarded as very expressive but also very slow. In recent years, the
introduction of just-in-time compilers (JIT) for these languages (e.g.
PyPy, V8, Tracemonkey) started to change this perception by delivering
good performance that enables new applications. However, a parallel
programming model was not part of the design of those languages. Thus,
the reference implementations of e.g. Python and Ruby use a single,
global interpreter lock (GIL) to serialize the execution of code in
threads.

While this GIL prevents any parallelism from occurring, it also
provides some useful guarantees. Since this lock is always acquired
while executing bytecode instructions and it may only be released
in-between such instructions, it provides perfect isolation and
atomicity between multiple threads for a series of
instructions. Another technology that can provide the same guarantees
is transactional memory (TM).

There have been several attempts at replacing the GIL with TM. Using
transactions to enclose multiple bytecode instructions, we can get the
very same semantics as the GIL while possibly executing several
transactions in parallel. Furthermore, by exposing these
interpreter-level transactions to the application in the form of
\emph{atomic blocks}, we give dynamic languages a new synchronization
mechanism that avoids several of the problems of locks as they are
used now.

\cfbolz{the above is good, here is something missing: problems with current STM approaches, outlining the intuition behind the new one}

Our contributions include:
\begin{itemize}[noitemsep]
\item We introduce a new software transactional memory (STM) system
  that performs well even on low numbers of CPUs. It uses a novel
  combination of hardware features and garbage collector (GC)
  integration in order to keep the overhead of STM very low.
\item This new STM system is used to replace the GIL in Python and is
  then evaluated extensively.
\item We introduce atomic blocks to the Python language to provide a
  backwards compatible, composable synchronization mechanism for
  threads.
\end{itemize}



\section{Background}


\subsection{Transactional Memory}

Transactional memory (TM) is a concurrency control mechanism that
comes from database systems. Using transactions, we can group a series
of instructions performing operations on memory and make them happen
atomically and in complete isolations from other
transactions. \emph{Atomicity} means that all these instructions in
the transaction and their effects seem to happen at one, undividable
point in time. Other transactions never see inconsistent state of a
partially executed transaction which is called \emph{isolation}.

If we start multiple such transactions in multiple threads, the TM
system guarantees that the outcome of running the transactions is
\emph{serializable}. Meaning, the outcome is equal to some sequential
execution of these transactions. This means that the approach provides the same
semantics as using the GIL
while still allowing the TM system to
run transactions in parallel as an optimization.


\subsection{Python}

\cfbolz{a pypy introduction needs to go somewhere, a paragraph or so. maybe in the evaluation section}

We implement and evaluate our system for the Python language. For the
actual implementation, we chose the PyPy interpreter because replacing
the GIL there with a TM system is just a matter of adding a new
transformation to the translation process of the interpreter.

Over the years, Python added multiple ways to provide concurrency and
parallelism to its applications. We want to highlight two of them,
namely \emph{threading} and \emph{multiprocessing}.

\emph{Threading} employs operating system (OS) threads to provide
concurrency. It is, however, limited by the GIL and thus does not
provide parallelism. At this point we should mention that it is indeed
possible to run external functions written in C instead of Python in
parallel. Our work focuses on Python itself and ignores this aspect as
it requires writing in a different language.

The second approach, \emph{multiprocessing}, uses multiple instances
of the interpreter itself and runs them in separate OS processes.
Here we actually get parallelism because we have one GIL per
interpreter, but of course we have the overhead of multiple processes
/ interpreters and also need to exchange data between them explicitly
and expensively.

We focus on the \emph{threading} approach. This requires us to remove
the GIL from our interpreter in order to run code in parallel on
multiple threads. One approach to this is fine-grained locking instead
of a single global lock. Jython and IronPython are implementations of
this. It requires great care in order to avoid deadlocks, which is why
we follow the TM approach that provides a \emph{direct} replacement
for the GIL. It does not require careful placing of locks in the right
spots. We will compare our work with Jython for evaluation.


\subsection{Synchronization}

\cfbolz{citation again needed for the whole subsection}

It is well known that using locks to synchronize multiple threads is
hard. They are non-composable, have overhead, may deadlock, limit
scalability, and overall add a lot of complexity. For a better
parallel programming model for dynamic languages, we want to implement
another, well-known synchronization mechanism: \emph{atomic blocks}.

Atomic blocks are composable, deadlock-free, higher-level and expose
useful atomicity and isolation guarantees to the application for a
series of instructions. An implementation using a GIL would simply
guarantee that the GIL is not released during the execution of the
atomic block. Using TM, we have the same effect by guaranteeing that
all instructions in an atomic block are executed inside a single
transaction.


\remi{STM, how atomicity \& isolation; reasons for overhead}


\section{Method}

\subsection{Transactional Memory Model}

In this section, we describe the general model of our TM system. This
should clarify the general semantics in commonly used terms from
the literature.

\cfbolz{there is an overview paragraph of the idea missing, maybe in the introduction}

\cfbolz{this all feels very much dumping details, needs more overview. why is this info important? the subsubsections don't have any connections}

\subsubsection{Conflict Handling}


Our conflict detection works with \emph{object
  granularity}. Conceptually, it is based on \emph{read} and
\emph{write sets} of transactions.  Two transactions conflict if they
have accessed a common object that is now in the write set of at least
one of them.

The \emph{concurrency control} works partly \emph{optimistically} for
reading of objects, where conflicts caused by just reading an object
in transactions are detected only when the transaction that writes the
object actually commits. For write-write conflicts we are currently
\emph{pessimistic}: Only one transaction may have a certain object in
its write set at any point in time, others trying to write to it will
have to wait or abort.

We use \emph{lazy version management} to ensure that modifications by
a transaction are not visible to another transaction before the former
commits.




\subsubsection{Semantics}

As required for TM systems, we guarantee complete \emph{isolation}
and \emph{atomicity} for transactions at all times. Furthermore,
the isolation provides full \emph{opacity} to always guarantee a consistent
read set.

To support irreversible operations that cannot be undone when we abort
a transaction (e.g. I/O, syscalls, and non-transactional code in
general), we employ \emph{irrevocable} or \emph{inevitable
transactions}. These transactions are always guaranteed to
commit. There is always at most one such transaction running in the
system, thus their execution is serialised. With this guarantee,
providing \emph{strong isolation} and \emph{serializability} between
non-transactional code is possible by making the current transaction
inevitable right before running irreversible operations.


\subsubsection{Contention Management}

When a conflict is detected, we perform some simple contention
management.  First, inevitable transactions always win. Second, the
older transaction wins. Different schemes are possible.


\subsubsection{Software Transactional Memory}

Generally speaking, the system is fully implemented in
software. However, we exploit some more advanced features of current
CPUs, especially \emph{memory segmentation, virtual memory,} and the
64-bit address space.


\subsection{Implementation}

In this section, we will present the general idea of how the TM model
is implemented. Especially the aspects of providing isolation and
atomicity, as well as conflict detection are explained. We try to do
this without going into too much detail about the implementation.  The
later section \ref{sub:Low-level-Implementation} will discuss it in
more depth.


\subsubsection{Memory Segmentation}

A naive approach to providing complete isolation between threads is to
partition the virtual memory of a process into $N$ segments, one per
thread. Each segment then holds a copy of all the memory available to
the program. Thus, each thread automatically has a private copy of
every object that it can modify in complete isolation from other
threads.

In order to reference these objects, we need references that are valid
in all threads and automatically point to the private copies. Since
an object's offset inside a segment is the same in all segments, we
can use this offset to reference objects. Because all segments are
copies of each other, this \emph{Segment Offset (SO)} points to the
private version of an object in all threads\,/\,segments. To then
translate this SO to a real virtual memory address when used inside a
thread, we need to add the thread's segment start address to the
SO. The result of this operation is called a \emph{Linear Address
  (LA)}. This is illustrated in Figure \ref{fig:Segment-Addressing}.

x86-CPUs provide a feature called \emph{memory segmentation}. It
performs this translation from a SO to a LA directly in hardware.  We
can use the segment register $\%gs$, which is mostly unused in current
applications. When this register points to a thread's segment start
address, we can instruct the CPU to perform the above translation from
a reference of the form $\%gs{::}SO$ to the right LA on its own. This
process is efficient enough that we can do it on every access to an
object.

In summary, we can use a single SO to reference the same object in all
threads, and it will be translated by the CPU to a LA that always
points to the thread's private version of this object. Thereby,
threads are fully isolated from each other. However, $N$ segments
require $N$-times the memory and modifications on an object need to be
propagated to all segments.

\begin{figure*}[t]
  \begin{centering}
    \includegraphics[scale=0.8]{\string"segment addressing\string".pdf}
    \par\end{centering}

    \protect\caption{Segment Addressing\label{fig:Segment-Addressing}}
\end{figure*}



\subsubsection{Page Sharing}

In order to eliminate the prohibitive memory requirements of keeping
around $N$ segment copies, we share memory between them. The segments
are initially allocated in a single range of virtual memory by a call
to \lstinline!mmap()!.  As illustrated in Figure
\ref{fig:mmap()-Page-Mapping}, \lstinline!mmap()! creates a mapping
between a range of virtual memory pages and virtual file pages. The
virtual file pages are then mapped lazily by the kernel to real
physical memory pages. The mapping generated by \lstinline!mmap()! is
initially linear but can be changed arbitrarily. Especially, we can
remap so that multiple virtual memory pages map to a single virtual
file page. This is what we use to share memory between the segments
since then we also only require one page of physical memory.

\begin{figure}[h]
  \begin{centering}
    \includegraphics[scale=0.8]{\string"mmap pages\string".pdf}
    \par\end{centering}

    \protect\caption{\texttt{mmap()} Page Mapping\label{fig:mmap()-Page-Mapping}}
\end{figure}


As illustrated in Figure \ref{fig:Page-Remapping}, in our initial
configuration (I) all segments are backed by their own range of
virtual file pages. This is the share-nothing configuration.

We then designate segment 0 to be the \emph{Sharing-Segment}. No
thread gets this segment assigned to it, it simply holds the pages
shared between all threads. So in (II), we remap all virtual pages of
the segments $>0$ to the file pages of our sharing-segment. This is
the fully-shared configuration.

During runtime, we can then privatize single pages in segments $>0$
again by remapping single pages as seen in (III).

Looking back at address translation for object references, we see now
that this is actually a two-step process. First, $\%gs{::}SO$ gets
translated to different linear addresses in different threads by the
CPU. Then, depending on the current mapping of virtual pages to file
pages, these LAs can map to a single file page in the sharing-segment,
or to privatized file pages in the corresponding segments. This
mapping is also performed efficiently by the CPU and can easily be
done on every access to an object.

In summary, $\%gs{::}SO$ is translated efficiently by the CPU to
either a physical memory location which is shared between several
threads/segments, or to a location in memory private to the
segment/thread.  This makes the memory segmentation model for
isolation memory efficient again.

\begin{figure}[h]
  \begin{centering}
    \includegraphics[width=1\columnwidth]{\string"page remapping\string".pdf}
    \par\end{centering}

    \protect\caption{Page Remapping: (I) after \texttt{mmap()}. (II) remap all pages to
      segment 0, fully shared memory configuration. (III) privatize single
      pages.\label{fig:Page-Remapping}}
\end{figure}



\subsubsection{Isolation: Copy-On-Write}

We now use these mechanisms to provide isolation for transactions.
Using write barriers, we implement a \emph{Copy-On-Write (COW)} on the
level of pages. Starting from the initial fully-shared configuration
(Figure \ref{fig:Page-Remapping}, (II)), when we need to modify an
object without other threads seeing the changes immediately, we ensure
that all pages belonging to the object are private to our segment.

To detect when to privatize pages, we use write barriers before every
write. When the barrier detects that the object is not in a private
page (or any pages that belong to the object), we remap and copy the
pages to the thread's segment. From now on, the translation of
$\%gs{::}SO$ in this particular segment will resolve to the private
version of the object. Note, the SO used to reference the object does
not change during that process.



\subsubsection{Isolation: Barriers}

The job of barriers is to ensure complete isolation between transactions
and to register the objects in the read or write set. We insert read
and write barriers before reading or modifying an object except if
we statically know an object to be readable or writable already.
\begin{description}
\item [{Read~Barrier:}] Adds the object to the read set of the current
  transaction. Since our two-step address translation automatically
  resolves the reference to the private version of the object on every
  access anyway, the read barrier does not need to do address translation anymore.
\item [{Write~Barrier:}] Adds the object to the read and write set of
  the current transaction and checks if all pages of the object are
  private, doing COW otherwise.\\
  Furthermore, we currently allow only one transaction modifying an
  object at a time. To ensure this, we acquire a write lock on the object
  and also eagerly check for a write-write conflict at this point. If
  there is a conflict, we do some contention management to decide which
  transaction has to wait or abort. Eagerly detecting this kind of conflict
  is not inherent to our system, future experiments may show that we
  want to lift this restriction.
\end{description}



\subsubsection{Atomicity: Commit \& Abort}

To provide atomicity for a transaction, we want to make changes globally
visible on commit. We also need to be able to completely abort a
transaction without a trace, like it never happened.
\begin{description}
\item [{Commit:}] If a transaction commits, we synchronize all threads
  so that all of them are waiting in a safe point. In the committing
  transaction, we go through all objects in the write set and check if
  another transaction in a different segment read the same object.
  Conflicts are resolved again by either the committing or the other
  transaction waiting or aborting.\\
  We then push all changes of modified objects in private pages to all
  the pages in other segments, including the sharing-segment (segment
  0).
\item [{Abort:}] On abort the transaction will forget about all the
  changes it has done. All objects in the write set are reset by
  copying their previous version from the sharing-segment into the
  private pages of the aborting transaction.\\
  Re-sharing these pages instead of resetting the changes would not
  only be slower because it involves a system call, it is also very
  likely the page would get privatised again immediately after re-trying
  the transaction.  Since this again involves a full page copy,
  resetting should be faster than re-sharing.
\end{description}

\cfbolz{random question: did we investigate the extra memory requirements? we should characterize memory overhead somewhere, eg at least one byte per object for the read markers}

\subsubsection{Summary}

We provide isolation between transactions by privatizing the pages of
the segments belonging to the threads the transactions run in.  To
detect when and which pages need privatization, we use write barriers
that trigger a COW of one or several pages. Conflicts, however, are
detected on the level of objects; based on the concept of read and
write sets. Barriers before reading and writing add objects to the
corresponding set; particularly detecting write-write conflicts
eagerly.  On commit, we resolve read-write conflicts and push
modifications to other segments. Aborting transactions simply undo
their changes by copying from the sharing-segment.


\subsection{Low-level Implementation\label{sub:Low-level-Implementation}}

In this section, we will provide details about the actual
implementation of the system and discuss some of the issues that we
encountered.


\subsubsection{Architecture}

Our TM system is designed as a library that covers all aspects around
transactions and object management. The library consists of two parts:
(I) It provides a simple interface to starting and committing
transactions, as well as the required read and write barriers. (II) It
also includes a \emph{garbage collector (GC)} that is closely
integrated with the TM part (e.g. it shares the write barrier). The
close integration helps in order to know more about the lifetime of an
object, as will be explained in the following sections.


\subsubsection{Application Programming Interface\label{sub:Application-Programming-Interfac}}

\begin{lstlisting}[basicstyle={\footnotesize\ttfamily},tabsize=4]
  void stm_start_transaction(tl, jmpbuf)
  void stm_commit_transaction()
  void stm_read(object_t *obj)
  void stm_write(object_t *obj)
  object_t *stm_allocate(ssize_t size_rounded)
  STM_PUSH_ROOT(tl, obj)
  STM_POP_ROOT(tl, obj)
\end{lstlisting}


\lstinline!stm_start_transaction()!  starts a transaction. It requires
two arguments, the first being a thread-local data structure and the
second a buffer for use by \lstinline!setjmp()!.
\lstinline!stm_commit_transaction()!  tries to commit the current
transaction. \lstinline!stm_read()!, \lstinline!stm_write()!  perform
a read or a write barrier on an object and \lstinline!stm_allocate()!
allocates a new object with the specified size (must be a multiple of
16). \lstinline!STM_PUSH_ROOT()!  and \lstinline!STM_POP_ROOT()!  push
and pop objects on the shadow stack~\footnote{A stack for pointers to
  GC objects that allows for precise garbage collection. All objects
  on that stack are never seen as garbage and are thus always kept
  alive.}.  Objects have to be saved using this stack around calls
that may cause a GC cycle to happen, and also while there is no
transaction running. In this simplified API, only
\lstinline!stm_allocate()!  and \lstinline!stm_commit_transaction()!
require saving object references.

The type \lstinline!object_t!  is special as it causes the
compiler~\footnote{Clang 3.5 with some patches to this address-space
  256 feature} to make all accesses through it relative to the $\%gs$
register.  With exceptions, nearly all accesses to objects managed by
the TM system should use this type so that the CPU will translate the
reference to the right version of the object.


\subsubsection{Setup\label{sub:Setup}}

On startup, we reserve a big range of virtual memory with a call to
\lstinline!mmap()! and partition this space into $N+1$ segments. We
want to run $N$ threads in parallel while segment 0 is designated as
the \emph{sharing-segment} that is never assigned to a thread.

The next step involves using \lstinline!remap_file_pages()!, a Linux
system call, to establish the fully-shared configuration.  All pages
of segments $>0$ map to the pages of the sharing-segment.

However, the layout of a segment is not uniform and we actually
privatize a few areas again right away. These areas are illustrated in
Figure \ref{fig:Segment-Layout} and explained here:
\begin{description}[noitemsep]
\item [{NULL~page:}] This page is unmapped and will produce a
  segmentation violation when accessed. We use this to detect
  erroneous dereferencing of \lstinline!NULL! references.  All
  $\%gs{::}SO$ translated to linear addresses will point to NULL pages
  if SO is set to \lstinline!NULL!.
\item [{Segment-local~data:}] Some area private to the segment that
  contains segment-local information.
\item [{Read~markers:}] These are pages that store information about
  which objects were read in the current transaction running in this
  segment.
\item [{Nursery:}] This area contains all the freshly allocated
  objects (\emph{young objects}) of the current transaction. The GC
  uses pointer-bump allocation in this area to allocate objects in the
  first generation.
\item [{Old~object~space:}] These pages are the ones that are really
  shared between segments. They mostly contain old objects but also
  some young ones that were too big to be allocated in the nursery.
\end{description}


\begin{figure*}[t]
  \begin{centering}
    \includegraphics[scale=0.8]{\string"segment layout\string".pdf}
    \par\end{centering}

    \protect\caption{Segment Layout\label{fig:Segment-Layout}}
\end{figure*}



\subsubsection{Assigning Segments}

From the above setup it is clear that the number of segments is
statically set to some $N$. That means that at any point in time, a
maximum of $N$ threads and their transactions can be running in
parallel.  To support an unlimited number of threads in applications
that use this TM system, we assign segments dynamically to threads.

At the start of a transaction, the thread it is running in acquires a
segment. It may have to wait until another thread finishes its
transaction and releases a segment. Fairness is not guaranteed yet, as
we simply assume a fair scheduling policy in the operating system when
waiting on a condition variable.

Therefore, a thread may be assigned to different segments each time it
starts a transaction. Although, we try to assign it the same segment
again if possible.




\subsubsection{Garbage Collection}

Garbage collection plays a big role in our TM system. The GC is
generational and has two generations: the \emph{young} and the
\emph{old} generation.

The \textbf{young generation}, where objects are considered to be
\emph{young} and reside in the \emph{Nursery}, is collected by
\emph{minor collections}. These collections move the surviving objects
out of the nursery into the old object space, which can be done
without stopping other threads. This is done either if the nursery has
no space left anymore or if we are committing the current
transaction. Consequently, all objects are old and the nursery empty
after a transaction commits.  Furthermore, all objects in the nursery
were always created in the current transaction. This fact is useful
since we do not need to call any barrier on this kind of objects.

To improve this situation even more, we introduce the concept of
\emph{overflow objects}. If a minor collection needs to occur during a
transaction, we empty the nursery and mark each surviving object in
the old object space with an \lstinline!overflow_number!  globally
unique to the current transaction. That way we can still detect in a
medium-fast path inside barriers that the object still belongs to the
current transaction.

The \textbf{old generation}, where objects are considered to be
\emph{old} and never move again, is collected by \emph{major
  collections}.  These collections are implemented in a stop-the-world
kind of way and first force minor collections in all threads. The
major goal is to free objects in the old objects space. Furthermore,
we optimistically re-share pages that do not need to be private
anymore.

As seen in the API (section~\ref{sub:Application-Programming-Interfac}),
we use a \emph{shadow stack} in order to provide precise garbage
collection.  Any time we call a function that possibly triggers a
collection, we need to save the objects that we need afterwards on the
shadow stack using \lstinline!STM_PUSH_ROOT()!.  That way, they will
not be freed. And in case they were young, we get their new location
in the old object space when getting them back from the stack using
\lstinline!STM_POP_ROOT()!.




\subsubsection{Read Barrier}

The point of the read barrier is to add the object to the read set of
the transaction. This information is needed to detect conflicts
between transactions. In other STM systems, it also resolves an object reference to
a private copy, but since the CPU performs our address translation on
every object access efficiently, we do not need to do that in our
barrier.

To add the object to the read set, for us it is enough to mark it as
read. Since this information needs to be local to the segment, we need
to store it in private pages. The area is called \emph{read markers
}and already mentioned in section \ref{sub:Setup}. This area can be
seen as a continuous array of bytes that is indexed from the start of
the segment by an object's reference ($SO$) divided by 16 (this
requires objects of at least 16 bytes in size). Instead of just
setting the byte to \lstinline!true!  if the corresponding object was
read, we set it to a \lstinline!read_version!  belonging to the
transaction, which will be incremented on each commit.  Thereby, we
can avoid resetting the bytes to \lstinline!false!  on commit and only
need to do this every 255 transactions. The whole code for the barrier
is easily optimizable for compilers as well as perfectly predictable
for CPUs:

\begin{lstlisting}[basicstyle={\footnotesize\ttfamily},tabsize=4]
void stm_read(SO):
    *(SO >> 4) = read_version
\end{lstlisting}



\subsubsection{Write Barrier}

The job of the write barrier is twofold: first, it serves as a write
barrier for the garbage collector and second, it supports
copy-on-write and adds objects to the write set of the transaction.

The \textbf{fast path} of the write barrier is very simple. We only
need to check for the flag \lstinline!WRITE_BARRIER!  in the object's
header and call the slow path if it is set. This flag is set either if
the object is old and comes from an earlier transaction, or if there
was a minor collection which will add the flag again on all
objects. It is never set on freshly allocated objects that still
reside in the nursery.

\begin{lstlisting}[basicstyle={\footnotesize\ttfamily},tabsize=4]
void stm_write(SO):
	if SO->flags & WRITE_BARRIER:
		write_slowpath(SO)
\end{lstlisting}


The \textbf{slow path} is shown here:

\begin{lstlisting}[basicstyle={\footnotesize\ttfamily},tabsize=4]
void write_slowpath(SO):
	// GC part:
	list_append(to_trace, SO)
	if is_overflow_obj(SO):
		SO->flags &= ~WRITE_BARRIER
		return
	// STM part
	stm_read(SO)
	lock_idx = SO >> 4
  retry:
	if write_locks[lock_idx] == our_num:
		// we already own it
	else if write_locks[lock_idx] == 0:
		if cmp_and_swap(&write_locks[lock_idx],
					    0, our_num):
			list_append(modified_old_objects, SO)
			privatize_pages(SO)
		else:
			goto retry
	else:
		w_w_contention_management()
		goto retry
	SO->flags &= ~WRITE_BARRIER
\end{lstlisting}


First comes the \emph{GC part}: In any case, the object will be added
to the list of objects that need tracing in the next minor collection
(\lstinline!to_trace!).  This is necessary in case we write a
reference to it that points to a young object. We then need to trace
it during the next minor collection in order to mark the young object
alive and to update its reference to the new location it gets moved
to. The check for \lstinline!is_overflow_obj()!  tells us if the
object was actually created in this transaction. In that case, we do
not need to execute the following \emph{TM part}.  We especially do
not need to privatize the page since no other transaction knows about
these ``old'' objects.

For TM, we first perform a read barrier on the object. We then try to
acquire its write lock. \lstinline!write_locks!  again is a simple
global array of bytes that is indexed with the SO of the object
divided by 16. If we already own the lock, we are done.  If someone
else owns the lock, we will do a write-write contention management
that will abort either us or the current owner of the object.  If we
succeed in acquiring the lock using an atomic
\lstinline!cmp_and_swap!, we need to add the object to the write set
(a simple list called \lstinline!modified_old_objects!)  and privatize
all pages belonging to it (copy-on-write).

In all cases, we remove the \lstinline!WRITE_BARRIER!  flag from the
object before we return. Thus, we never trigger the slow path again
before we do the next minor collection (also part of a commit) or we
start the next transaction.




\subsubsection{Abort}

Aborting a transaction is rather easy. The first step is to reset the
nursery and all associated data structures. The second step is to go
over all objects in the write set (\lstinline!modified_old_objects!)
and reset any modifications in our private pages by copying from the
sharing-segment. What is left is to use \lstinline!longjmp()!  to jump
back to the location initialized by a \lstinline!setjmp()!  in
\lstinline!stm_start_transaction()!.  Increasing the
\lstinline!read_version!  is also done there.




\subsubsection{Commit}

Committing a transaction needs a bit more work. First, we synchronize
all threads so that the committing one is the only one running and all
the others are waiting in a safe point. We then go through the write
set (\lstinline!modified_old_objects!)  and check the corresponding
\lstinline!read_markers!  in other threads/segments. If we detect a
read-write conflict, we do contention management to either abort us or
the other transaction, or to simply wait a bit (see \ref{subsub:contentionmanagement}).

After verifying that there are no conflicts anymore, we copy all our
changes done to the objects in the write set to all other segments,
including the sharing-segment. This is safe since we synchronised all
threads. We also need to push overflow objects generated by minor
collections to other segments, since they may reside partially in
private pages. At that point we also get a new
\lstinline!overflow_number!  by increasing a global one, so that it
stays globally unique for each transaction. Increasing the
\lstinline!read_version!  is then done at the start of a new
transaction.



\subsubsection{Thread Synchronization}

A requirement for performing a commit is to synchronize all threads so
that we can safely update objects in other segments. To make this
synchronization fast and cheap, we do not want to insert an additional
check regularly in order to see if synchronization is requested. We
use a trick relying on the fact that dynamic languages are usually
very high-level and thus allocate a lot of objects very regularly.
This is done through the function \lstinline!stm_allocate!  shown
below:

\begin{lstlisting}[basicstyle={\footnotesize\ttfamily},tabsize=4]
object_t *stm_allocate(ssize_t size_rounded):
    result = nursery_current
	nursery_current += size_rounded
	if nursery_current > nursery_end:
		return allocate_slowpath(size_rounded)
	return result
\end{lstlisting}


This code does simple pointer-bump allocation in the nursery. If there
is still space left in the nursery, we return
\lstinline!nursery_current!  and bump it up by
\lstinline!size_rounded!.  The interesting part is the check
\lstinline!nursery_current > nursery_end!  which will trigger the slow
path of the function to possibly perform a minor collection in order
to free up space in the nursery.

If we want to synchronize all threads, we can rely on this check being
performed regularly. So what we do is to set the
\lstinline!nursery_end!  to $0$ in all segments that we want to
synchronize. The mentioned check will then fail in those segments and
call the slow path. In \lstinline!allocate_slowpath!  they can simply
check for this condition and enter a safe point.

For other synchronization requirements, for example:
\begin{itemize}[noitemsep]
\item waiting for a segment to be released,
\item waiting for a transaction to abort or commit,
\item waiting for all threads to reach their safe points,
\end{itemize}
we use a set of condition variables to wait or signal other threads.


\subsubsection{Contention Management\label{subsub:contentionmanagement}}

On encountering conflicts, we employ contention management to solve
the problem as well as we can. The general rules are:
\begin{itemize}[noitemsep]
\item prefer transactions that started earlier to younger transactions
\item to support \emph{inevitable} transactions, we always prefer them
  to others since they cannot abort
\end{itemize}
We can either simply abort a transaction to let the other one succeed,
or we can also wait until the other transaction committed. The latter
is an interesting option if we are trying to commit a write and
another transaction already read the object. We can then signal the
other transaction to commit as soon as possible and wait. After
waiting, there is now no conflict between our write and the already
committed read anymore.




\section{Experimental Results}

compare some programs between
\begin{itemize}[noitemsep]
\item pypy
\item pypy-jit
\item pypy-stm
\item pypy-stm-jit
\item cpython
\item jython
\end{itemize}



\section{Related Work}


\section{Conclusions}




\section{Further comments}

Some citation: \cite{Higham:98}

\bibliographystyle{IEEEbib}
\bibliography{bibl_conf}

\end{document}
