\documentclass[utf8x]{beamer}

\mode<presentation>
{
  \usetheme{Warsaw}

  %\setbeamercovered{transparent}
}


\usepackage[english]{babel}

\usepackage[utf8x]{inputenc}

\usepackage{times}
\usepackage[T1]{fontenc}

\title{Back to the Future in one Week --- Implementing a Smalltalk VM in PyPy}

\author{Carl Friedrich Bolz, Adrian Kuhn, Adrian Lienhard, Nicholas D. Matsakis, Oscar Nierstrasz, Lukas Renggli, Armin Rigo, Toon Verwaest}


\date{Workshop on Self-sustaining Systems, May 16 2008}


% Delete this, if you do not want the table of contents to pop up at
% the beginning of each subsection:
%\AtBeginSubsection[]
%{
%  \begin{frame}<beamer>
%    \frametitle{Outline}
%    \tableofcontents[currentsection,currentsubsection]
%  \end{frame}
%}


% If you wish to uncover everything in a step-wise fashion, uncomment
% the following command: 

%\beamerdefaultoverlayspecification{<+->}


\begin{document}

\begin{frame}
  \titlepage
\end{frame}

%\begin{frame}
%  \frametitle{Outline}
%  \tableofcontents
  % You might wish to add the option [pausesections]
%\end{frame}

\frame{
    \frametitle{Scope}
    This talk is about: XXX
    \itemize{
    \item writing a Squeak implementation
    \item with eight people
    \item in five days
    \item using PyPy
    }
}


\frame{
    \frametitle{What is PyPy}
    \itemize{
    \item started as a Python implementation in Python
    \item developed into a general environment for implementing dynamic languages
    \item supports the language developer with a lot of infrastructure
    \item Open Source project, MIT license
    \item most important goal: abstracting over low-level details
    \item don't fix decisions about low-level details 
    }
}

\frame{
    \frametitle{PyPy's Approach to VM Construction}
    \itemize{
    \item implement an interpreter for the dynamic language in RPython
    \item translate this interpreter to a low-level language
    \item translating inserts low-level details
    \item a variety of target environment: C, LLVM, JVM, .NET
    % XXX write about model-driven development?
    }
    \pause
    \begin{block} {What is RPython?}
        \itemize{
        \item a more static subset of Python
        \item static enough to enable type inference
        \item still rather expressive: exceptions, inheritance, dynamic dispatch
        \item analysis starts after importing of interpreter
        \item enables compile-time metaprogramming
        }
    \end{block}
}

\frame{
    \begin{block} {Translation Aspects}
        \itemize{
        \item many aspects of the final VM are orthogonal to language semantics
        \item examples: GC strategy, threading model, many object details
        \item non-trivial translation aspect: auto-generating a dynamic compiler
        \item those shouldn't manifest in the interpreter source
        \item they are inserted during translation
        }
    \end{block}
    \pause
    \begin{block} {Compile-Time Metaprogramming}
        \itemize{
        \item PyPy's translation toolchain starts analysis after importing
        \item arbitrary (non-RPython) code can be executed during import
        \item this allows metaprogramming of parts of the interpreter
        }
    \end{block}
}

\frame{
    \frametitle{The SPy VM}
    \itemize{
        \item foo
    }
}
\end{document}


