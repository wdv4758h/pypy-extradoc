\documentclass[utf8x]{beamer}

\mode<presentation>
{
  \usetheme{Warsaw}

  %\setbeamercovered{transparent}
}


\usepackage[english]{babel}

\usepackage[utf8x]{inputenc}

\usepackage{times}
\usepackage[T1]{fontenc}

\title{PyPy – Status and Next Goals}

\author{Carl Friedrich Bolz}

\institute[Heinrich-Heine-Universität Düsseldorf]
{
  Institut für Informatik\\
  Heinrich-Heine-Universität Düsseldorf
}

\date{Ruby \& Python Conference, April 12th 2008, Poznań}

%\pgfdeclareimage[height=0.5cm]{pypy-logo}{image/py-web.png}
%\logo{\pgfuseimage{pypy-logo}}



% Delete this, if you do not want the table of contents to pop up at
% the beginning of each subsection:
%\AtBeginSubsection[]
%{
%  \begin{frame}<beamer>
%    \frametitle{Outline}
%    \tableofcontents[currentsection,currentsubsection]
%  \end{frame}
%}


% If you wish to uncover everything in a step-wise fashion, uncomment
% the following command: 

%\beamerdefaultoverlayspecification{<+->}


\begin{document}

\begin{frame}
  \titlepage
\end{frame}

%\begin{frame}
%  \frametitle{Outline}
%  \tableofcontents
  % You might wish to add the option [pausesections]
%\end{frame}

\begin{frame}
  \frametitle{What is PyPy?}
  ... pretty pictures and demos
\end{frame}


\begin{frame}
  \frametitle{What is PyPy?}
  \begin{itemize}
  \item
    A flexible toolkit for writing interpreters
    \begin{itemize}
      \item
        geared at dynamic languages
      \item
        giving you lots of help
      \item
        translation to various of environments (C, .NET, Java)
      \item
        a mostly good GC
      \item
        libraries
      \item
        a JIT
    \end{itemize}
  \pause
  \item
    a Python interpreter
    \begin{itemize}
      \item
        using the toolkit
      \item
        implementing Python 2.4 semantics fully
      \item
        extra features
      \item
        stackless
      \item
        various introspection enhancements
      \item
        experimental things like lazy evaluation
    \end{itemize}
  \item most important goal: flexibility at all levels
  \item
    Open Source project (MIT licenced)
  \end{itemize}
\end{frame}




\begin{frame}
  \frametitle{Last year's developments}
  \begin{itemize}
  \item
    cleanups
  \item
    implementing CTypes
  \item
    .NET integration
  \item
    GC improvements
  \item
    JIT improvements
  \end{itemize}
\end{frame}

\begin{frame}
  \frametitle{Cleanups}
  \begin{itemize}
  \item PyPy got EU funding for 2.5 years
  \item in that time many features were added because the proposal required them
  \item not all of them were implemented well
  \item many of them were unmaintained
  \item some of them were preventing progress in other areas
  \item so we killed those
  \item improvements and cleanups everywhere, to get ready for the next round of features
  \item example: better implementation for unicode strings
  \end{itemize}
\end{frame}

\begin{frame}
  \frametitle{CTypes}
  \begin{itemize}
  \item CTypes is a libffi wrapper for CPython
  \item no need to write C code to interface with libraries
  \item can be used to wrap C libraries rather effortlessly
  
  \end{itemize}
  \pause
  \begin{block}{
    CTypes implementation for PyPy}
    \begin{itemize}
    \item two parts
    \item a very thin wrapper around libffi
    \item normal Python code for all the CTypes logic
    \item some things missing but already very useful
    \item ... demo
    \end{itemize}
  \end{block}
\end{frame}


\begin{frame}
  \frametitle{.NET Integration}
  \begin{itemize}
  \item giving PyPy.NET similar capabilities as CPython
  \item accessing .NET libraries seamlessly from PyPy
  \item various improvements
  \item ... demo
  \end{itemize}
\end{frame}


\begin{frame}
  \frametitle{Garbage Collection}
  \begin{itemize}
  \item RPython itself is garbage collected
%        XXX this is the first time you mention RPython, that's
%        XXX either too late, or not mention it at all
%        no, it is mentioned in the pictures part
  \item when translating to C, a GC is integrated into the program
  \item GCs are completely written in RPython
  \item two fast GCs: semispace, generational
  \item lots of care to get the semantics right:
    \begin{itemize}
    \item finalizers, weakrefs
    \item id function
    \item identity hashes
    \end{itemize}
  \item performance pretty good
  \item memory-bound benchmarks faster than CPython
  \item ... tiny demo
  \end{itemize}
\end{frame}


\begin{frame}
  \frametitle{Currently Working on: Just-in-Time Compiler}
  Problems:
  \begin{itemize}
  \item writing good JITs for dynamic languages is hard work
  \item good performance needs lots of care
  \item keeping up with new language features is hard
  \item compilers are a bad encoding for language semantics
  \pause
  \begin{block}{
    Existing hand-written JITs}
    \begin{itemize}
    \item Psyco (mostly unmaintained)
    \item LuaJit (slightly simpler language, genius at work)
    \item Tamarin (company support, who knows how good it is)
    \item Self
    \item Smalltalk VMs
    \end{itemize}
  \end{block}
  \end{itemize}
\end{frame}

\begin{frame}
  \frametitle{If you cannot write them, grow your own}
  \begin{itemize}
  \item If you cannot write a JIT, don't do that then
  \pause
  \item Idea: Automatically generate the JIT from the interpreter
  \pause
  \end{itemize}
  \begin{block}{
    Partial Evaluation}
    \begin{itemize}
    \item old idea from CS research
    \item transforms an interpreter into a static compiler
    \item never worked well for real-life examples
    \item key insight: generating a JIT is \emph{easier} than a static compiler
    \end{itemize}
  \end{block}
 
\end{frame}

\begin{frame}
  \frametitle{Advantages of JIT Generation}
  Writing a JIT generator is hard, but:
  \begin{itemize}
  \item You only need to do it once
  \item can keep up with language evolution
  \item the JIT is correct by construction
  \item retargetting the JIT is possible (e.g. to .NET, as Antonio Cuni is doing right now)
  \item make the JIT behave radically differently is much easier with a generator
  \item it's fun too
  \end{itemize}
\end{frame}

\begin{frame}
  \frametitle{JIT Generation: Current Status}
  \begin{itemize}
  \item we had a working prototype at the end of the EU project
  \item speeds up simple arithmetic functions \emph{a lot}
  \item had limitations that were hard to fix
  \pause
  \item refactoring happened in the last two months to improve the situation
  \item still in progress, but already allows experimentation
  \item we used the new flexibility to try new, advanced things
  \item goes into the direction of a tracing JIT
  \item right now we are starting to hit the hard bits
  \item we hope to get really good results for more general code in a few months
  \end{itemize}
\end{frame}

\begin{frame}
  \frametitle{Conclusion}
  \begin{itemize}
  \item PyPy approaching real world usefulness
  \item using CTypes to interface with the outside
  \item hopefully with a JIT soon
  \end{itemize}
\end{frame}

\begin{frame}
  \frametitle{Help Wanted!}
  we are looking for new contributors!
  \begin{itemize}
  \item try out your own apps that don't require extension modules
  \item use CTypes to implement missing modules
  \item experiment with extending Python (logic, distribution, what suits you)
  \item write a new JIT backend (x86-64 anyone?)
  \end{itemize}
\end{frame}

\begin{frame}
  \frametitle{Questions?}
  \begin{block}{
    PyPy}
    \bigskip
    \hskip 1cm \url{http://codespeak.net/pypy/}
    \bigskip
  \end{block}
\end{frame}


\end{document}


