\section{CLI backend for Rainbow interpreter}

%\subsection{Promotion on CLI}
%
%Implementing promotion on top of CLI is not straightforward, as it needs to
%patch and modify the already generated code, and this is not possible on .NET.
%
%To solve, we do xxx and yyy etc. etc.
%

\subsection{Flexswitches}

As already explained, \dacom{I guess flexswitches will be introduced
  in the previous sect.} \emph{flexswitch} is one of the key
concepts allowing the JIT compiler generator to produce code which can
be incrementally specialized and compiled at run time.

A flexswitch is a special kind of switch which can be dynamically
extended with new cases; intuitively, its behavior can be described
well in terms of flow graphs. Indeed, a flexswitch can be considered 
as a special flow graph block where links to newly created blocks are
dynamically added whenever new cases are needed. 

\begin{figure}[h]
\begin{center}
\includegraphics[height=5cm]{flexswitch1}
\includegraphics[height=5cm]{flexswitch2}
\caption{An example of a flexswitch evolution: in the picture on the
  right a new block has been dynamically added.}\label{flexswitch-fig}
\end{center}
\end{figure}

In the pictures of Figure~\ref{flexswitch-fig}, the cyan block
corresponds to a flexswitch; initially (picture on the left) 
only the block containing the code to restart the JIT compilation
is connected to the flexswitch; the picture on the right
shows the graph after the first case has been dynamically added to the flexswitch,
by linking the cyan block with a freshly created new block.


\subsection{Implementing flexswitches in CLI}

Implementing flexswitches for backends generating machine code is
not too complex: basically, a new jump has to be inserted in the
existing code to point to the newly generated code fragment.

Unfortunately, the CLI VM does not allow modification of code which
has been already loaded and linked, therefore the simplest approach
taken for low level architectures does not work for higher level 
virtual machines as those for .NET and Java.

Since in .NET methods are the basic units of compilation, a possible
solution consists in creating a new method 
any time a new case has to be added to a flexswitch.
In this way, whereas flow graphs without flexswitches are translated
to a single method, the translation of flow graphs which can dynamically grow because of
flexswitches will be scattered over several methods.
Summarizing, the backend behaves in the following way:
\begin{itemize}
\item Each flow graph is translated in a collection of methods which
  can grow dynamically. Each collection contains at least one
  method, called \emph{primary}, which is the first to be created.
  All other methods, called \emph{secondary}, are added dynamically 
  whenever a new case is added to a flexswitch.

\item Each either primary or secondary method implements a certain
  number of blocks, all belonging to the same flow graph. Among these blocks
  there always exists an initial block whose input arguments 
  might be passed as arguments of the method; however, for
  implementation  reasons (see the details below) the input variables
  of all blocks (including the initial one)
  are implemented as local variables of the method. 
\end{itemize} 

When  a new case is added to a flexswitch, new blocks are generated
and translated by the backend in a new single method pointed
by a delegate \footnote{\emph{Delegates} are the .NET equivalent of function pointers}
 of  which is stored in the code implementing the flexswitch,
so that the method can be invoked later.

\subsubsection{Internal and external links}

A link is called \emph{internal} if it connects two blocks implemented
by the same method,
 \emph{external} otherwise.

Following an internal link would  not be difficult in IL bytecode: a jump to
the corresponding code fragment in the same method can be emitted 
to execute the new block, whereas the appropriate local variables can be
used for passing arguments. 
Also following an external link whose target is an initial block could
be easily implemented, by just invoking the corresponding method.

What cannot be easily implemented in CLI is following an external link
whose target is not an initial block; consider, for instance, the
outgoing link of the block dynamically added in the right-hand side
picture of Figure~\ref{flexswitch-fig}. How it is possible to pass the
right arguments to the target block?

To solve this problem every method contains a special code, called
\emph{dispatcher}; whenever a method is invoked, its dispatcher is
executed first\footnote{The dispatcher should not be
confused with the initial block of a method.} to
determine which block has to be executed.
This is done by passing to the method a 32 bits number, called 
\emph{block id}, which uniquely identifies the next block of the graph to be executed.
The high 2 bytes \dacom{word was meant as a fixed-sized group of bits} of a block id constitute the id of the method to which the block
belongs, whereas the low 2 bytes constitute a progressive number univocally identifying
each block implemented by the method.

The picture in Figure~\ref{block-id-fig} shows a graph composed of three methods (for
simplicity, dispatchers are not shown); method ids are in red, whereas
block numbers are in black. 
The graph contains three external links; in particular, note the link
between blocks \texttt{0x00020001} and \texttt{0x00010001} which
connects two blocks implemented by different methods.
\begin{figure}[h]
\begin{center}
\includegraphics[height=6cm]{blockid}
\caption{Method and block ids.\dacom{in the picture ``Main method'' must be
    replaced with ``Primary method'' and in the primary method the block with number 0 should be added}}\label{block-id-fig}
\end{center}
\end{figure}

For instance, the code\footnote{For simplicity we write C\# code instead of
the actual IL bytecode.} generated for the dispatcher of method \texttt{0x0002}
is similar to the following fragment: 
\begin{small}
\begin{lstlisting}[language={[Sharp]C}]
// dispatch block
int methodid = (blockid && 0xFFFF0000) >> 16; 
int blocknum = blockid && 0x0000FFFF;         
if (methodid != MY_METHOD_ID) {
  // jump_to_ext 
  ...
}
switch(blocknum) {
  case 0: goto block0;
  case 1: goto block1;
  default: throw new Exception("Invalid block id");
}
\end{lstlisting}
\end{small}
If the next block to be executed is implemented in the same method
({\small\lstinline{methodid == MY_METHOD_ID}}), then the appropriate
jump to the corresponding code is executed.
Otherwise, the \lstinline{jump_to_ext}
part of the dispatcher has to be executed.
The code that actually jumps to an external block is contained in
the dispatcher of the primary method, whereas the
\lstinline{jump_to_ext} code of dispatchers of secondary methods
simply delegates the dispatcher of the primary method of the same
graph (see later).

The primary method is responsible for the bookkeeping of the secondary
methods which are added to the same graph dynamically. This can be 
simply implemented with an array mapping method id of secondary methods
to the corresponding delegate. Therefore, the primary methods contain
the following \lstinline{jump_to_ext} code (where
\lstinline{FlexSwitchCase} is the type of delegates for secondary methods):
\begin{small}
\begin{lstlisting}[language={[Sharp]C}] 
// jump_to_ext
FlexSwitchCase meth = method_map[methodid];
blockid = meth(blockid, ...); // execute the method
goto dispatch_block;
\end{lstlisting}
\end{small}
Each secondary method returns the block id of the next block to be
executed; therefore, after the secondary method has returned, the
dispatcher of the primary method will be executed again to jump
to the correct next block. 

To avoid mutual recursion and an undesired growth of the stack,
the \lstinline{jump_to_ext} code in dispatchers of secondary methods
just returns the block id of the next block; since the primary method
is always the first method of the graph which is called, the correct
jump will be eventually executed by the dispatcher of the primary method.

Clearly this complex translation is performed only for flow graphs
having at least one flexswitch; flow graphs without flexswitches
are implemented in a more efficient and direct way by a unique method
with no dispatcher.

\subsubsection{Passing arguments to external links}

The main drawback of our solution is that passing arguments across
external links cannot be done efficiently by using the parameters of
methods for the following reasons:
\begin{itemize}
\item In general, the number and type of arguments is different for every block in a graph;

\item The number of blocks of a graph can grow dynamically, therefore
  it is not possible to compute in advance the union of the arguments
  of all blocks in a graph; 

\item Since external jumps are implemented with a delegate, all the
  secondary methods of a graph must have the same signature.
\end{itemize}

Therefore, the only solution we came up with is defining a class
\lstinline{InputArgs} for passing sequences of arguments whose length
and type is variable.
\begin{small}
\begin{lstlisting}[language={[Sharp]C}] 
public class InputArgs {
  public int[] ints;
  public float[] floats;
  public object[] objs;
  ...
}
\end{lstlisting}
\end{small}
Unfortunately, with this solution passing arguments to external links
becomes quite slow:
\begin{itemize}
\item When writing arguments, array re-allocation may be needed in
  case the number of arguments exceeds the dimension of the
  array. Furthermore the VM will always perform bound-checks, even
  when the size is explicitly checked in advance;

\item When reading arguments, a bound-check is performed by the VM for
  accessing each argument; furthermore, an appropriate downcast must be
  inserted anytime an argument of type object is read.
\end{itemize}
Of course, we do not need to create a new object of class
\lstinline{InputArgs} any time we need to perform an external jump;
instead, a unique object is created at the beginning of the execution
of the primary method. 

\subsubsection{Implementation of flexswitches}
Finally, we can have a look at the implementation of flexswitches.
The following snippet shows the special case of integer flexswitches.
\begin{small}
\begin{lstlisting}[language={[Sharp]C}] 
public class IntLowLevelFlexSwitch:BaseLowLevelFlexSwitch {
  public uint default_blockid = 0xFFFFFFFF;
  public int numcases = 0;
  public int[] values = new int[4];
  public FlexSwitchCase[] cases = new FlexSwitchCase[4];

  public void add_case(int value, FlexSwitchCase c)
  {
    ...
  }

  public uint execute(int value, InputArgs args)
  {
    for(int i=0; i<numcases; i++)
    if (values[i] == value) {
      return cases[i](0, args);
    }
    return default_blockid;
  }
}
\end{lstlisting}
\end{small}
The mapping from integers values to delegates (pointing to secondary
methods) is just implemented by the two arrays \lstinline{values} and
\lstinline{cases}. Method \lstinline{add_case} extends the mapping
whenever a new case is added to the flexswitch.
  
The most interesting part is the body of method \lstinline{execute},
which takes a value and a set of input arguments to be passed across
the link and jumps to the right block by performing a linear search in
array \lstinline{values}.

Recall that the first argument of delegate \lstinline{FlexSwitchCase} is the
block id to jump to. By construction, the target block of a flexswitch is
always the first in a secondary method, and we use the special value
\lstinline{0} to signal this.

The value returned by method \lstinline{execute} is the next block id
to be executed; 
in case no association is found for \lstinline{value},
\lstinline{default_blockid} is returned. The value of
\lstinline{default_blockid} is initially set by the JIT compiler and
usually corresponds to a block containing code to restart the JIT
compiler for creating a new secondary method with the new code for the
missing case, and updating the flexswitch by calling method
\lstinline{add_case}.

\subsection{Alternative implementations}
\dacom{need to be discussed with Antonio}

\anto{I propose to wait until the rest of paper is more or less in a final
  state, and see how much space is left}

\commentout{
Before implementing the solution described here, we carefully studied a lot of possible alternatives, but all of them either didn't work because of a limitation of the virtual machine or they could work but with terrible performances.

In particular, in theory it is possible to implement external links using tail calls, by putting each block in its own method and doing a tail call instead of a jump; this would also solve the problem of how to pass arguments, as each method could have its own signature matching the input args of the block. I would like to explain this solution in a more detailed way as I think it's really elegant and nice.

In theory, if the .NET JIT were smart enough it could inline and optimize away the tail calls (or at least many of those) and give us very efficient code. However, one benchmark I wrote shows that tail calls are up to 10 times slower (!!!) than normal calls, thus making impractical to use them for our purposes.
}

% LocalWords:  flexswitches backend flexswitch methodid blockid xFFFF blocknum
% LocalWords:  FFFF goto FlexSwitchCase meth InputArgs ints objs VM uint args
% LocalWords:  IntLowLevelFlexSwitch BaseLowLevelFlexSwitch xFFFFFFFF numcases
% LocalWords:  JIT
