\documentclass{llncs}


 %%%% Compression Light+: LNCS margin reduced by +/-7mm along all edges (RG).
%\textwidth=130mm   % LNCS: 122mm
%\textheight=203mm  % LNCS: 193mm

\renewcommand{\baselinestretch}{0.97}

\usepackage{amssymb}
\usepackage{amsmath}
\usepackage[sans]{dsfont}
\usepackage{color}
\usepackage{ifthen}
\usepackage{xspace}
\usepackage{listings}
\usepackage{fancyvrb}
\usepackage{multirow}
\usepackage[pdftex]{graphicx}

%\input{macros}

\pagestyle{plain}

%\lstset{mathescape=true,language=Java,basicstyle=\tt,keywordstyle=\bf}
\lstset{language=Python,
        basicstyle=\scriptsize\ttfamily,
        keywordstyle=\color{blue}, % I couldn't find a way to make chars both bold and tt
        frame=none,
        stringstyle=\color{blue},
        fancyvrb=true,
        xleftmargin=20pt,xrightmargin=20pt,
        showstringspaces=false}

\setlength{\tabcolsep}{1ex}


%\renewcommand{\baselinestretch}{.98}
\newboolean{showcomments}
\setboolean{showcomments}{false}
\ifthenelse{\boolean{showcomments}}
  {\newcommand{\nb}[2]{
    \fbox{\bfseries\sffamily\scriptsize#1}
    {\sf\small$\blacktriangleright$\textit{#2}$\blacktriangleleft$}
   }
   \newcommand{\version}{\emph{\scriptsize$-$Id: main.tex 19055 2008-06-05 11:20:31Z cfbolz $-$}}
  }
  {\newcommand{\nb}[2]{}
   \newcommand{\version}{}
  }

\newcommand\dacom[1]{\nb{DA}{#1}}
\newcommand\cfbolz[1]{\nb{CFB}{#1}}
\newcommand\anto[1]{\nb{ANTO}{#1}}
\newcommand\arigo[1]{\nb{AR}{#1}}
\newcommand{\commentout}[1]{}

\let\oldcite=\cite

\renewcommand\cite[1]{\ifthenelse{\equal{#1}{XXX}}{[citation~needed]}{\oldcite{#1}}}


\begin{document}
\title{Automatic generation of JIT compilers for dynamic languages
  in .NET\thanks{This work has been partially
supported by MIUR EOS DUE - Extensible Object Systems for Dynamic and
Unpredictable Environments and by the EU-funded project: IST 004779 PyPy
(PyPy: Implementing Python in Python).}}


\author{Davide Ancona\inst{1} \and Carl Friedrich Bolz\inst{2} \and Antonio Cuni\inst{1} \and Armin Rigo\inst{2}}

\institute{DISI, University of Genova, Italy 
\and 
Softwaretechnik und Programmiersprachen
 Heinrich-Heine-Universit\"at D\"usseldorf}

\maketitle

\commentout{
\section{Tentative structure}
\begin{itemize}
  \item Introduction \& background; main contributions:
    \begin{itemize}
      \item Promotion
      \item (unboxing)
      \item JIT layering
    \end{itemize}
  \item How the generated JITs work
    \begin{itemize}
      \item Promotion: allows intermixing compile-time and runtime; how to
        compare with polymorphic inline caches and partial evaluation
      \item unboxing: how they compare with tracing JITs
      \item (don't talk about merging)
    \end{itemize}
  \item .NET backend
  \item How the JIT generator works
  \item Benchmarks (TLC)
  \item Future works
  \item Conclusions
\end{itemize}
}    

\begin{abstract}
Writing an optimizing static compiler for dynamic languages is not an
easy task, since quite complex static analysis is required.
On the other hand, recent developments show that JIT compilers 
can exploit runtime type information to generate quite efficient code.
Unfortunately, writing a JIT compiler is far from being simple.
 
In this paper we report our positive experience with automatic generation
of JIT compilers as supported by the PyPy infrastructure, by
focusing  on JIT compilation for .NET.
The paper presents two main and novel contributions: we show that
partial evaluation can be used in practice for generating a JIT compiler,
and we experiment with the potentiality offered by the ability to
add a further level of JIT compilation on top of .NET.

The practicality of the approach is demonstrated by showing some
promising experiments done with benchmarks written in a simple dynamic language.
\end{abstract}

% LocalWords:  JIT PyPy

\section{Introduction}
Implementing a dynamic language such as Python with a compiler rather than with an interpreter improves performances at the cost of
an increasing complexity. Furthermore, generating code for high level virtual machines like CLI or JVM enhances portability and inter-operability.

Writing a compiler that targets the CLI or JVM is easier than targeting a real CPU, but
it still requires a lot of work, as IronPython\footnote{\texttt{http://www.codeplex.com/IronPython}},
Jython\footnote{\texttt{http://www.jython.org/}} and JRuby\footnote{\texttt{http://jruby.codehaus.org/}} demonstrate.
Finally, if one really seeks for an efficent language implementation, \emph{Just In Time} (JIT) compilation needs
to be considered; only in this way  the compiler can exploit runtime type
information to generate quite efficient code. Note that JIT compilation
has not to be confused with lazy compilation of IronPython and Jython which is exploited to save memory, 
since in these cases no runtime type information is ever used to generate more efficient code.

Unfortunately, writing a JIT compiler is a very complex task.  
To make this task simpler, the solution proposed by PyPy \cite{RiBo07_223}  
is automatic generation of JIT compilers with the help of partial evaluation techniques: 
the user has only to provide an interpreter manually annotated with \emph{hints}
specifying how interpretation and JIT compilation has to be interleaved \cite{PyPyJIT09}.

More precisely, in this paper we focus on the ability of generating a JIT compiler able to emit code
for the .NET virtual machine. To our knowledge, this is the first experiment with an interpreter with
two \emph{layers} of JIT compilation, since, before being executed, the
emitted code is eventually compiled again by .NET's own JIT compiler.

The main contribution of this paper is to demonstrate that the idea of
\emph{JIT layering} can give good results, as dynamic languages can be even
faster than their static counterparts.

\subsection{Overview of PyPy}

The \emph{PyPy} project\footnote{\texttt{http://codespeak.net/pypy/}}
\cite{RigoPedroni06} was initially conceived to develop an implementation of Python which
could be easily portable and extensible without renouncing efficiency.
To achieve these aims, the PyPy implementation is based on a highly
modular design which allows high-level aspects
to be separated from lower-level implementation details.
The abstract semantics of Python is defined by an interpreter written
in a high-level language, called RPython \cite{AACM-DLS07}, which is in fact a subset of
Python where some dynamic features have been sacrificed to allow an
efficient translation of the interpreter to low-level code\footnote{But note that it's a full Python interpreter; RPython is only the
language in which this interpreter is written.}.

Compilation of the interpreter is implemented as a stepwise
refinement by means of a translation toolchain which performs type
analysis, code optimizations and several transformations aiming at 
incrementally providing implementation details such as memory management or the threading model.
The different kinds of intermediate codes  which are refined 
during the translation process are all represented by a collection of control flow graphs,
at several levels of abstractions.

Finally, the low-level control flow graphs produced by the toolchain
can be translated to executable code for a specific platform by a
corresponding backend.
Currently, three fully developed backends are available to produce
executable C/POSIX code, Java and CLI/.NET bytecode. 

Despite having been specifically developed for Python, the PyPy infrastructure
can in fact be used for implementing other languages. Indeed, there were
successful experiments of using PyPy to implement several other languages such
as Smalltalk \cite{BolzEtAl08}, JavaScript, Scheme and Prolog.

\commentout{
As suggested by Figure~\ref{pypy-fig}, a portable interpreter for a
generic language $L$  can be
easily developed once an interpreter for $L$ has been implemented in
RPython.
}

\subsection{PyPy and JIT-Generation}
\label{sec:jitgen}

One of the most important aspects that PyPy's translation toolchain can weave
in is the \emph{automatic generation of a JIT compiler}.  This section will
give a high-level overview of how the JIT-generation process works. More
details can be found in \cite{PyPyJIT} and \cite{PyPyJIT09}.

The main difference between the JIT compilers generated by PyPy and the
ones found in other projects like IronPython is that the latter compile
code at the method granularity: they can do little to optimize most of
the operations inside, because few assumptions can be made about the
types of the arguments and the
global state of the program.  The PyPy JITs, on the other hand, work at
a sub-method granularity, as described next.

When using PyPy, the first step is to write an interpreter for the chosen language.  Since it
must be fed to the translation toolchain, the interpreter has to be written in
RPython.  Then, to guide the process, we need to add few manual
annotations (called hints) to the interpreter, in order to teach the JIT generator which
information is important to know at compile-time.  
From these annotations, PyPy will statically generate an interpreter and a JIT
compiler in a single executable (here a .NET executable).

The interesting property of the generated JIT compiler is to delay the
compilation until it knows all the information needed to generate
efficient code.  In other words, at runtime, when the interpreter notice
that it is useful to compile a given piece of code, it sends it to the
JIT compiler; however, if at some point the JIT compiler does not know
about something it needs, it generates a \emph{callback} into itself and stops
execution.

Later, when the generated code is executed, the callback might be hit and the JIT
compiler is restarted again.  At this point, the JIT knows exactly the state
of the program and can exploit all this extra knowledge to generate highly
efficient code.  Finally, the old code is patched and linked to the newly
generated code, so that the next time the JIT compiler will not be invoked
again.  As a result, \textbf{runtime and compile-time are continuously
interleaved}. 

Potentially, the JIT compiler generates new code for each different run-time
value seen in variables it is interested in.
This implies that the generated code needs to contain some sort of updatable
switch, called \emph{flexswitch}, which can pick the right code path based on the
run-time value.  Typically, the value we switch on is the runtime dynamic type
of a value, so that the JIT compiler has all information needed to produce
very good code for that specific case.

Modifying the old code to link to the newly generated one is very challenging on
.NET, as the framework does not offer any primitive to do this.  Section
\ref{sec:clibackend} explains how it is possible to obtain this behaviour.

\section{The TLC language}

In this section, we will briefly describe \emph{TLC}, a simple dynamic
language that we developed to exercise our JIT compiler generator.  As most of
dynamic languages around, \emph{TLC} is implemented through a virtual machine
that interprets a custom bytecode. Since our main interest is in the runtime
performance of the VM, we did not implement the parser nor the bytecode
compiler, but only the VM itself.

TLC provides four different types:
\begin{enumerate}
\item Integers
\item \lstinline{nil}, whose only value is the null value
\item Objects
\item Lisp-like lists
\end{enumerate}

Objects represent a collection of named attributes (much like JavaScript or
Self) and named methods.  At creation time, it is necessary to specify the set
of attributes of the object, as well as its methods.  Once the object has been
created, it is not possible to add/remove attributes and methods.

The virtual machine is stack-based, and provides several operations:

\begin{itemize}
\item \textbf{Stack manipulation}: standard operations to manipulate the
  stack, such as \lstinline{PUSH}, \lstinline{POP}, \lstinline{SWAP}, etc.
\item \textbf{Flow control} to do conditional and unconditional jumps.
\item \textbf{Arithmetic}: numerical operations on integers, like
  \lstinline{ADD}, \lstinline{SUB}, etc.
\item \textbf{Comparisons} like \lstinline{EQ}, \lstinline{LT},
  \lstinline{GT}, etc.
\item \textbf{Object-oriented}: operations on objects: \lstinline{NEW},
  \lstinline{GETATTR}, \lstinline{SETATTR}, \lstinline{SEND}.
\item \textbf{List operations}: \lstinline{CONS}, \lstinline{CAR},
  \lstinline{CDR}.
\end{itemize}

Obviously, not all the operations are applicable to all objects. For example,
it is not possible to \lstinline{ADD} an integer and an object, or reading an
attribute from an object which does not provide it.  Being a dynamic language,
the VM needs to do all these checks at runtime; in case one of the check
fails, the execution is simply aborted.

\anto{should we try to invent a syntax for TLC and provide some examples?}
\cfbolz{we should provide an example with the assembler syntax}


\section{Automatic generation of JIT compilers}

Traditional JIT compilers are hard to write, time consuming, hard to evolve,
etc. etc.

\begin{figure}[h]
\begin{center}
\includegraphics[width=.6\textwidth]{diagram1}
\caption{PyPy infrastructure for generating an interpreter of a
  language $L$ with JIT compilation for the .NET platform}
\end{center}
\end{figure}

The JIT generation framework uses partial evaluation techniques to generate a
dynamic compiler from an interpreter; the idea is inspired by Psyco, which
uses the same techniques but it's manually written instead of being
automatically generated.

The original idea is by Futamura \cite{Futamura99}. He proposed to generate compilers
from interpreters with automatic specialization, but his work has had
relatively little practical impact so far.

\subsection{Partial evaluation}

Assume the Python bytecode to be constant, and constant-propagate it into the
Python interpreter.
\cfbolz{note to self: steal bits from the master thesis?}

\cfbolz{I would propose to use either TLC as an example here, or something that
looks at least like an interpreter loop}

Example::
\begin{lstlisting}[language=Python]
  def f(x, y):    
    x2 = x * x    
    y2 = y * y    
    return x2 + y2

**case x=3** ::

  def f_3(y):    
    y2 = y * y   
    return 9 + y2

**case x=10** ::

  def f_10(y):    
    y2 = y * y   
    return 100 + y2
\end{lstlisting}

A shortcoming of PE is that in many cases not much can be really assumed
constant at compile-time, and this leads to poor results.  Effective dynamic
compilation requires feedback of runtime information into compile-time; for a
dynamic language, types are a primary example.

Partial evaluation (PE) comes in two flavors:

\begin{itemize}
\item \emph{On-line} partial evaluation: a compiler-like algorithm takes the
source code of the function \texttt{f(x, y)} (or its intermediate representation,
i.e. its control flow graph), and some partial
information, e.g. \texttt{x=5}.  From this, it produces the residual function
\texttt{g(y)} directly, by following in which operations the knowledge \texttt{x=5} can
be used, which loops can be unrolled, etc.

\item \emph{Off-line} partial evalution: in many cases, the goal of partial
evaluation is to improve performance in a specific application.  Assume that we
have a single known function \texttt{f(x, y)} in which we think that the value
of \texttt{x} will change slowly during the execution of our program – much
more slowly than the value of \texttt{y}.  An obvious example is a loop that
calls \texttt{f(x, y)} many times with always the same value \texttt{x}.  We
could then use an on-line partial evaluator to produce a \texttt{g(y)} for each
new value of \texttt{x}.  In practice, the overhead of the partial evaluator
might be too large for it to be executed at run-time.  However, if we know the
function \texttt{f} in advance, and if we know \emph{which} arguments are the
ones that we will want to partially evaluate \texttt{f} with, then we do not
need a full compiler-like analysis of \texttt{f} every time the value of
\texttt{x} changes.  We can precompute once and for all a specialized function
\texttt{f1(x)}, which when called produces the residual function \texttt{g(y)}
corresponding to \texttt{x}.  This is \emph{off-line partial evaluation}; the
specialized function \texttt{f1(x)} is called a \emph{generating extension}.
\end{itemize}

Off-line partial evaluation is usually based on \emph{binding-time analysis}, which
is the process of determining among the variables used in a function (or
a set of functions) which ones are going to be known in advance and
which ones are not.  In the example of \texttt{f(x, y)}, such an analysis
would be able to infer that the constantness of the argument \texttt{x}
implies the constantness of many intermediate values used in the
function.  The \emph{binding time} of a variable determines how early the
value of the variable will be known.

\cfbolz{XXX: unclear how the next paragraph will fit into the text in the end.
it's certainly wrong as it is}
Once binding times have been determined, one possible approach to
producing the generating extension itself is by self-applying on-line
partial evaluators.  This is known as the second Futamura projection
\cite{Futamura99}.  So far it is unclear if this approach can lead to optimal
results, or even if it scales well.  In PyPy we selected a more direct
approach: the generating extension is produced by transformation of the
control flow graphs of the interpreter, guided by the binding times.  We
call this process \emph{timeshifting}.


\subsection{Execution steps}


PyPy contains a framework for generating just-in-time compilers using
off-line partial evaluation.  As such, there are three distinct phases:

\begin{itemize}
\item \emph{Translation time:} during the normal translation of an RPython
program, say the TLC interpreter, we perform binding-time analysis on the
interpreter.  This produces a generating extension, which is linked with the
rest of the program. \cfbolz{XXX not quite right}

\item \emph{Compile time:} during the execution of the program, when a new
bytecode is about to be interpreted, the generating extension is invoked
instead.  As the generating extension is a compiler, all the computations it
performs are called compile-time computations.  Its sole effect is to produce
residual code.

\item \emph{Run time:} the normal execution of the program (which includes the
time spent running the residual code created by the generating extension).
\end{itemize}

Translation time is a purely off-line phase; compile time and run time are
actually highly interleaved during the execution of the program.

\subsection{Binding Time Analysis}

At translation time, PyPy performs binding-time analysis of the source
RPython program after it has been turned to low-level graphs, i.e. at
the level at which operations manipulate pointer-and-structure-like
objects.

The binding-time terminology that we are using in PyPy is based on the
colors that we use when displaying the control flow graphs:

\begin{itemize}
\item \emph{Green} variables contain values that are known at compile-time;
\item \emph{Red} variables contain values that are not known until run-time.
\end{itemize}

The binding-time analyzer of our translation tool-chain is based on the
same type inference engine that is used on the source RPython program,
the annotator.  In this mode, it is called the \emph{hint-annotator}; it
operates over input graphs that are already low-level instead of
RPython-level, and propagates annotations that do not track types but
value dependencies and manually-provided binding time hints.

The normal process of the hint-annotator is to propagate the binding
time (i.e. color) of the variables using the following kind of rules:

\begin{itemize}
\item For a foldable operation (i.e. one without side effect and which depends
only on its argument values), if all arguments are green, then the result can
be green too.

\item Non-foldable operations always produce a red result.

\item At join points, where multiple possible values (depending on control
flow) are meeting into a fresh variable, if any incoming value comes from a red
variable, the result is red.  Otherwise, the color of the result might be
green.  We do not make it eagerly green, because of the control flow
dependency: the residual function is basically a constant-folded copy of the
source function, so it might retain some of the same control flow.  The value
that needs to be stored in the fresh join variable thus depends on which
branches are taken in the residual graph.
\end{itemize}

\subsubsection{Hints}

Our goal in designing our approach to binding-time analysis was to
minimize the number of explicit hints that the user must provide in
the source of the RPython program.  This minimalism was not pushed to
extremes, though, to keep the hint-annotator reasonably simple.  

The driving idea was that hints should be need-oriented.  Indeed, in a
program like an interpreter, there are a small number of places where it
would be clearly beneficial for a given value to be known at
compile-time, i.e. green: this is where we require the hints to be
added.

The hint-annotator assumes that all variables are red by default, and
then propagates annotations that record dependency information.
When encountering the user-provided hints, the dependency information
is used to make some variables green.  All
hints are in the form of an operation \texttt{hint(v1, someflag=True)}
which semantically just returns its first argument unmodified.

The crucial need-oriented hint is \texttt{v2 = hint(v1, concrete=True)}
which should be used in places where the programmer considers the
knowledge of the value to be essential.  This hint is interpreted by
the hint-annotator as a request for both \texttt{v1} and \texttt{v2} to be green.  It
has a \emph{global} effect on the binding times: it means that not only
\texttt{v1} but all the values that \texttt{v1} depends on – recursively –
are forced to be green.  The hint-annotator complains if the
dependencies of \texttt{v1} include a value that cannot be green, like
a value read out of a field of a non-immutable structure.

Such a need-oriented backward propagation has advantages over the
commonly used forward propagation, in which a variable is compile-time
if and only if all the variables it depends on are also compile-time.  A
known issue with forward propagation is that it may mark as compile-time
either more variables than expected (which leads to over-specialization
of the residual code), or less variables than expected (preventing
specialization to occur where it would be the most useful).  Our
need-oriented approach reduces the problem of over-specialization, and
it prevents under-specialization: an unsatisfiable \texttt{hint(v1,
concrete=True)} is reported as an error.

In our context, though, such an error can be corrected.  This is done by
promoting a well-chosen variable among the ones that \texttt{v1} depends on.

Promotion is invoked with the use of a hint as well:
\texttt{v2 = hint(v1, promote=True)}.
This hint is a \emph{local} request for \texttt{v2} to be green, without
requiring \texttt{v1} to be green.  Note that this amounts to copying
a red value into a green one, which is not possible in classical
approaches to partial evaluation.  See the Promotion section XXX ref for a
complete discussion of promotion.



\section{Automatic Unboxing of Intermediate Results}
\label{sec:virtuals}

Interpreters for dynamic languages typically continuously allocate a lot of small
objects, for example due to boxing. This makes arithmetic operations extremely
inefficient. For this reason, we
implemented a way for the compiler to try to avoid memory allocations in the
residual code as long as possible. The idea is to try to keep new
run-time instances \emph{exploded}: instead of a single run-time object allocated on
the heap, the object is \emph{virtualized} as a set
of fresh local variables, one per field. Only when the object can be accessed by from
somewhere else is it actually allocated on the heap. The effect of this is similar to that of
escape analysis \cite{Blanchet99escapeanalysis}, \cite{Choi99escapeanalysis},
which also prevents allocations of objects that can be proven to not escape a
method or set of methods (the algorithms however are a lot more advanced than
our very simple analysis).

It is not always possible to keep instances virtual.  The main
situation in which it needs to be \emph{forced} (i.e. actually allocated at
run-time) is when the pointer escapes to some non-virtual location like
a field of a real heap structure.  Virtual instances still avoid the run-time
 allocation of most short-lived objects, even in non-trivial situations.  

In addition to virtual instances, the compiler can also handle virtual
containers, namely lists and dictionaries\footnote{(R)Python's dictionaries
  are equivalent to .NET \lstinline{Hashtable}s}.  If the indexing operations
can be evaluated at compile-time (i.e., if the variables holding the indexes
are green), the compiler internally keeps track of the state of the container
and store the items as local variables.

Look again at figure \ref{fig:tlc-folded}: the list in the \lstinline{stack}
variable never escapes from the function.  Moreover, all the indexing
operations (either done explicitly or implicitly by \lstinline{append} and
\lstinline{pop}) are evaluable at compile-time.  Thus, the list is kept
\emph{virtual} and its elements are stored in variables $v_n$, where $n$
represents the index in the list.  Figure \ref{fig:tlc-folded-virtualized}
show how the resulting code looks like; to ease the reading, the state of the
\lstinline{stack} as kept by the compiler is shown in the comments.

\begin{figure}[h]
\begin{center}
\input{tlc-folded-virtualized.py}
\caption{The result of virtualizing the \lstinline{stack} list}
\label{fig:tlc-folded-virtualized}
\end{center}
\end{figure}

Even if not shown in the example, \lstinline{stack} is not the only
virtualized object.  In particular the two objects created by
\lstinline{IntObj(0)} are also virtualized, and their fields are stored as
local variables as well.  Virtualizion of instances is important not only
because it avoids the allocation of unneeded temporary objects, but also
because it makes possible to optimize method calls on them, as the JIT
compiler knows their exact type in advance.


\section{Promotion}
\label{sec:promotion}

In the sequel, we describe in more details one of the main new
techniques introduced in our approach, which we call \emph{promotion}.  In
short, it allows an arbitrary run-time (i.e. red) value to be turned into a
compile-time (i.e. green) value at any point in time.  Promotion is thus the central way by
which we make use of the fact that the JIT is running interleaved with actual
program execution. Each promotion point is explicitly defined with a hint that
must be put in the source code of the interpreter.

From a partial evaluation point of view, promotion is the converse of
the operation generally known as \emph{lift}.  Lifting a value means
copying a variable whose binding time is compile-time into a variable
whose binding time is run-time – it corresponds to the compiler
``forgetting'' a particular value that it knew about.  By contrast,
promotion is a way for the compiler to gain \emph{more} information about
the run-time execution of a program. Clearly, this requires
fine-grained feedback from run-time to compile-time, thus a
dynamic setting.

Promotion requires interleaving compile-time and run-time phases,
otherwise the compiler can only use information that is known ahead of
time. It is impossible in the ``classical'' approaches to partial
evaluation, in which the compiler always runs fully ahead of execution.
This is a problem in many realistic use cases.  For example, in an
interpreter for a dynamic language, there is mostly no information
that can be clearly and statically used by the compiler before any
code has run.

A very different point of view on promotion is as a generalization of
techniques that already exist in dynamic compilers as found in modern virtual
machines for object-oriented language, like \emph{Polymorphic Inline Cache}
(PIC, \cite{hoelzle_optimizing_1991}) and its variations, whose main goal is
to optimize and reduce the overhead of dynamic dispatching and indirect
invocation: the dynamic lookups are cached and the corresponding generated
machine code contains chains of compare-and-jump instructions which are
modified at run-time.  These techniques also allow the gathering of
information to direct inlining for even better optimization results. Compared
to PICs, promotion is more general because it can be applied not only to
indirect calls but to any kind of value, including instances of user-defined
classes or integer numbers.

In the presence of promotion, dispatch optimization can usually be
reframed as a partial evaluation task.  Indeed, if the type of the
object being dispatched to is known at compile-time, the lookup can be
folded, and only a (possibly even inlined) direct call remains in the
generated code.  In the case where the type of the object is not known
at compile-time, it can first be read at run-time out of the object and
promoted to compile-time.  As we will see in the sequel, this produces
machine code very similar to that of polymorphic inline
caches.

The essential advantage of promotion is that it is no longer tied to the details of
the dispatch semantics of the language being interpreted, but applies in
more general situations.  Promotion is thus the central enabling
primitive to make partial evaluation a practical approach to language
independent dynamic compiler generation.

Promotion is invoked with the use of a hint as well:
\lstinline{v2 = hint(v1, promote=True)}.
This hint is a \emph{local} request for \texttt{v2} to be green, without
requiring \texttt{v1} to be green.  Note that this amounts to copying
a red value into a green one, which is not possible in classical
approaches to partial evaluation. A slightly different hint can be used to
promote the \emph{class} of an instance. This is done with
\lstinline{hint(v1, promote_class=True)}. It does not have an effect on the
bindings of any variable.


\subsection{Implementing Promotion}
\label{sec:implementing-promotion}

The implementation of promotion requires a tight coupling between
compile-time and run-time: a \emph{callback}, put in the generated code,
which can invoke the compiler again.  When the callback is actually
reached at run-time, and only then, the compiler resumes and uses the
knowledge of the actual run-time value to generate more code.

The new generated code is potentially different for each run-time value
seen.  This implies that the generated code needs to contain some sort
of updatable switch, or \emph{flexswitch}, which can pick the right code path based on the
run-time value.

Let us look again at the TLC example.  To ease the reading, figure
\ref{fig:tlc-main} showed a simplified version of TLC's main loop, which did
not include the hints.  The implementation of the \lstinline{LT} opcode with
hints added is shown in figure \ref{fig:tlc-main-hints}.

\begin{figure}[h]
\begin{center}
\begin{tabular}{l|l}
\begin{lstlisting}
def interp_eval(code, pc, args, pool):
  code_len = len(code)
  stack = []
  while pc < code_len:
      opcode = ord(code[pc])
      opcode = hint(opcode, concrete=True)
      pc += 1

      if opcode == PUSH:
          ...
      elif opcode == LT:
        a, b = stack.pop(), stack.pop()
        hint(a, promote_class=True)
        hint(b, promote_class=True)
        stack.append(IntObj(b.lt(a)))
\end{lstlisting}
&
\hspace{2pt}
\begin{lstlisting}
class IntObj(Obj):

  def lt(self, other): 
    return (self.value < 
            other.int_o())

  def sub(self, other):
    return IntObj(self.value -
                  other.int_o())

  def int_o(self):
    return self.value

  ...
\end{lstlisting}
\end{tabular}
\end{center}
\caption{Usage of hints in TLC's main loop and excerpt of the \lstinline{IntObj} class}
\label{fig:tlc-main-hints}
\end{figure}

By promoting the class of \lstinline{a} and \lstinline{b}, we tell the JIT
compiler not to generate code until it knows the exact RPython class of both.
Figure \ref{fig:tlc-abs-promotion-1} shows the
code\footnote{\lstinline{switch} is not a legal (R)Python statement, it is
  used here only as a pseudocode example.} generated while compiling the usual
\lstinline{abs} function: note that, compared to figure
\ref{fig:tlc-folded-virtualized}, the code stops just before the call
\lstinline{b.lt(a)}.

\begin{figure}[h]
\begin{center}
\begin{lstlisting}[language=Python]
def interp_eval_abs(args):
    v0 = args[0]
    v1 = IntObj(0)
    a, b = v0, v1
    # hint(a, promote_class=True) implemented as follows:
    cls_a = a.__class__
    switch cls_a:
        default: 
            continue_compilation(jitstate, cls_a)
\end{lstlisting}
\caption{Promotion step 1}
\label{fig:tlc-abs-promotion-1}
\end{center}
\end{figure}

\begin{figure}[h]
\begin{center}
\begin{lstlisting}[language=Python]
def interp_eval_abs(args):
    v0 = args[0]
    v1 = IntObj(0)
    a, b = v0, v1
    # hint(a, promote_class=True) implemented as follows:
    cls_a = a.__class__
    switch cls_a:
        IntObj:
            # hint(b, promote_class=True) needs no code
            v0 = IntObj(b.value < a.value)
            cond = v0
            if cond.value:
                return a
            else:
                v0 = IntObj(0)
                v1 = args[0]
                a, b = v0, v1
                v0 = IntObj(b.value - a.value)
                return v0
        default: 
            continue_compilation(jitstate, cls_a)
\end{lstlisting}
\caption{Promotion step 2}
\label{fig:tlc-abs-promotion-2}
\end{center}
\end{figure}

The first time the flexswitch is executed, the \lstinline{default} branch is
taken, and the special function \lstinline{continue_compilation} restarts the
JIT compiler, passing it the just-seen value of \lstinline{cls_a}.  The JIT
compiler generates new specialized code, and \emph{patches} the flexswitch to
add the new case, which is then executed.

If later an instance of \lstinline{IntObj} hits the flexswitch again, the
code is executed without needing more calls to the JIT compiler.  On the
other hand, if the flexswitch is hit by an instance of some other class, the
\lstinline{default} branch will be selected again and the whole process will
restart.

Now, let us examine the content of the \lstinline{IntObj} case: first, there
is a hint to promote the class of \lstinline{b}.  Although in general
promotion is implemented through a flexswitch, in this case it is not needed
as \lstinline{b} holds a \emph{virtual instance}, whose class is already
known (as described in the previous section).

Then, the compiler knows the exact class of \lstinline{b}, thus it can inline
the calls to \lstinline{lt}.  Moreover, inside \lstinline{lt} there is a
call to \lstinline{a.int_o()}, which is inlined as well for the very same
reason.  Moreover, as we saw in section \ref{sec:virtuals}, the \lstinline{IntObj}
instance can be virtualized, so that the subsequent \lstinline{BR_COND} opcode
can be compiled efficiently without needing any more flexswitch.

Figure~\ref{fig:tlc-abs-final} shows the final, fully optimized version of the
code, with all the instances virtualized and the unneeded temporary variables
removed.

\begin{figure}[h]
\begin{center}
\begin{lstlisting}[language=Python]
def interp_eval_abs(args):
    a = args[0]
    cls_a = a.__class__
    switch cls_a:
        IntObj:
            # 0 is the constant "value" field of the virtualized IntObj(0)
            if 0 < a.value:
                return a
            else:
                return IntObj(0 - a.value)
        default: 
            continue_compilation(jitstate, cls_a)
\end{lstlisting}
\caption{Final optimized version of the \lstinline{abs} example}
\label{fig:tlc-abs-final}
\end{center}
\end{figure}

\section{CLI backend for Rainbow interpreter}

\subsection{Promotion on CLI}

Implementing promotion on top of CLI is not straightforward, as it needs to
patch and modify the already generated code, and this is not possible on .NET.

To solve, we do xxx and yyy etc. etc.

\section{Benchmarks}
\label{sec:benchmarks}

In section \ref{sec:tlc-properties}, we saw that TLC provides most of the
features that usually make dynamically typed language so slow, such as
\emph{stack-based interpreter}, \emph{boxed arithmetic} and \emph{dynamic lookup} of
methods and attributes.

In the following sections, we present some benchmarks that show how our
generated JIT can handle all these features very well.

To measure the speedup we get with the JIT, we run each program three times:

\begin{enumerate}
\item By plain interpretation, without any jitting.
\item With the JIT enabled: this run includes the time spent by doing the
  compilation itself, plus the time spent by running the produced code.
\item Again with the JIT enabled, but this time the compilation has already
  been done, so we are actually measuring how good is the code we produced.
\end{enumerate}

Moreover, for each benchmark we also show the time taken by running the
equivalent program written in C\#.\footnote{The sources for both TLC and C\#
  programs are available at:

  http://codespeak.net/svn/pypy/extradoc/talk/ecoop2009/benchmarks/}

The benchmarks have been run on a machine with an Intel Pentium 4 CPU running at
3.20 GHz and 2 GB of RAM, running Microsoft Windows XP and Microsoft .NET
Framework 2.0.

\subsection{Arithmetic operations}

To benchmark arithmetic operations between integers, we wrote a simple program
that computes the factorial of a given number.  The algorithm is
straightforward, thus we are not showing the source code.  The loop contains only three operations: one
multiplication, one subtraction, and one comparison to check if we have
finished the job.

When doing plain interpretation, we need to create and destroy three temporary
objects at each iteration.  By contrast, the code generated by the JIT does
much better.  At the first iteration, the classes of the two operands of the
multiplication are promoted; then, the JIT compiler knows that both are
integers, so it can inline the code to compute the result.  Moreover, it can
\emph{virtualize} (see Section \ref{sec:virtuals}) all the temporary objects, because they never escape from
the inner loop.  The same remarks apply to the other two operations inside
the loop.

As a result, the code executed after the first iteration is close to optimal:
the intermediate values are stored as \lstinline{int} local variables, and the
multiplication, subtraction and \emph{less-than} comparison are mapped to a
single CLI opcode (\lstinline{mul}, \lstinline{sub} and \lstinline{clt},
respectively).

Similarly, we wrote a program to calculate the $n_{th}$ Fibonacci number, for
which we can do the same reasoning as above.

\begin{table}[ht]
  \begin{center}

  \begin{tabular}{l|rrrrrr}
    \multicolumn{5}{c}{\textbf{Factorial}} \\ [0.5ex]

    \textbf{$n$}          & $10$  & $10^7$           & $10^8$         & $10^9$         \\
    \hline
    \textbf{Interp}       & 0.031 & 30.984           & N/A            & N/A            \\
    \textbf{JIT}          & 0.422 &  0.453           & 0.859          & 4.844          \\
    \textbf{JIT 2}        & 0.000 &  0.047           & 0.453          & 4.641          \\
    \textbf{C\#}          & 0.000 &  0.031           & 0.359          & 3.438          \\
    \textbf{Interp/JIT 2} & N/A   & \textbf{661.000} & N/A            & N/A            \\
    \textbf{JIT 2/C\#}    & N/A   & \textbf{1.500}   & \textbf{1.261} & \textbf{1.350} \\ [3ex]


    \multicolumn{5}{c}{\textbf{Fibonacci}} \\ [0.5ex]

    \textbf{$n$}          & $10$  & $10^7$           & $10^8$         & $10^9$         \\
    \hline
    \textbf{Interp}       & 0.031 & 29.359           & 0.000          & 0.000          \\
    \textbf{JIT}          & 0.453 &  0.469           & 0.688          & 2.953          \\
    \textbf{JIT 2}        & 0.000 &  0.016           & 0.250          & 2.500          \\ 
    \textbf{C\#}          & 0.000 &  0.016           & 0.234          & 2.453          \\
    \textbf{Interp/JIT 2} & N/A   & \textbf{1879.962}& N/A            & N/A            \\
    \textbf{JIT 2/C\#}    & N/A   & \textbf{0.999}   & \textbf{1.067} & \textbf{1.019} \\
  \end{tabular}

  \end{center}
  \caption{Factorial and Fibonacci benchmarks}
  \label{tab:factorial-fibo}
\end{table}


Table \ref{tab:factorial-fibo} shows the seconds spent to calculate
the factorial and Fibonacci for various $n$.  As we can see, for small values
of $n$ the time spent running the JIT compiler is much higher than the time
spent to simply interpret the program.  This is an expected result
which, however, can be improved once we will have time
to optimize compilation and not only the generated code.

On the other, for reasonably high values of $n$ we obtain very good
results, which are valid despite the obvious overflow, since the 
same operations are performed for all experiments.
For $n$ greater than $10^7$, we did not run the interpreted program as it would have took too
much time, without adding anything to the discussion.

As we can see, the code generated by the JIT can be up to about 1800 times faster
than the non-jitted case.  Moreover, it often runs at the same speed as the
equivalent program written in C\#, being only 1.5 slower in the worst case.

The difference in speed it is probably due to both the fact that the current
CLI backend emits slightly non-optimal code and that the underyling .NET JIT
compiler is highly optimized to handle bytecode generated by C\# compilers.

As we saw in Section~\ref{sec:flexswitches-cli}, the implementation of
flexswitches on top of CLI is hard and inefficient.  However, our benchmarks
show that this inefficiency does not affect the overall performances of the
generated code.  This is because in most programs the vast majority of the
time is spent in the inner loop: the graphs are built in such a way that all
the blocks that are part of the inner loop reside in the same method, so that
all links inside are internal (and fast).


\subsection{Object-oriented features}

To measure how the JIT handles object-oriented features, we wrote a very
simple benchmark that involves attribute lookups and polymorphic method calls.
Since the TLC assembler source is long and hard to read,
figure~\ref{fig:accumulator} shows the equivalent program written in an
invented Python-like syntax.

\begin{figure}[h]
\begin{center}
\begin{lstlisting}
def main(n):
    if n < 0:
        n = -n
        obj = new(value, accumulate=count)
    else:
        obj = new(value, accumulate=add)
    obj.value = 0
    while n > 0:
        n = n - 1
        obj.accumulate(n)
    return obj.value

def count(x):
    this.value = this.value + 1

def add(x):
    this.value = this.value + x
\end{lstlisting}
\caption{The \emph{accumulator} example, written in a invented Python-like syntax}
\label{fig:accumulator}
\end{center}
\end{figure}

The two \lstinline{new} operations create an object with exactly one field
\lstinline{value} and one method \lstinline{accumulate}, whose implementation
is found in the functions \lstinline{count} and \lstinline{add}, respectively.
When calling a method, the receiver is implicity passed and can be accessed
through the special name \lstinline{this}.

The computation \emph{per se} is trivial, as it calculates either $-n$ or
$1+2...+n-1$, depending on the sign of $n$. The interesting part is the
polymorphic call to \lstinline{accumulate} inside the loop, because the interpreter has
no way to know in advance which method to call (unless it does flow analysis,
which could be feasible in this case but not in general).  The equivalent C\#
code we wrote uses two classes and a \lstinline{virtual} method call to
implement this behaviour.

As already discussed, our generated JIT does not compile the whole function at
once. Instead, it compiles and executes code chunk by chunk, waiting until it
knows enough informations to generate highly efficient code.  In particular,
at the time it emits the code for the inner loop it exactly knows the
type of \lstinline{obj}, thus it can remove the overhead of dynamic dispatch
and inline the method call.  Moreover, since \lstinline{obj} never escapes the
function, it is \emph{virtualized} and its field \lstinline{value} is stored
as a local variable.  As a result, the generated code turns out to be a simple loop
doing additions in-place.

\begin{table}[ht]
  \begin{center}

  \begin{tabular}{l|rrrrrr}
    \multicolumn{5}{c}{\textbf{Accumulator}} \\ [0.5ex]

    \textbf{$n$}          & $10$  & $10^7$           & $10^8$         & $10^9$         \\
    \hline
    \textbf{Interp}       & 0.031 & 43.063           & N/A            & N/A            \\
    \textbf{JIT}          & 0.453 &  0.516           & 0.875          & 4.188          \\
    \textbf{JIT 2}        & 0.000 &  0.047           & 0.453          & 3.672          \\
    \textbf{C\#}          & 0.000 &  0.063           & 0.563          & 5.953          \\
    \textbf{Interp/JIT 2} & N/A   & \textbf{918.765} & N/A            & N/A            \\
    \textbf{JIT 2/C\#}    & N/A   & \textbf{0.750}   & \textbf{0.806} & \textbf{0.617} \\

  \end{tabular}
  \end{center}
  \caption{Accumulator benchmark}
  \label{tab:accumulator}
\end{table}





Table \ref{tab:accumulator} show the results for the benchmark.  Again, we can
see that the speedup of the JIT over the interpreter is comparable to the
other two benchmarks.  However, the really interesting part is the comparison
with the equivalent C\# code, as the code generated by the JIT is up to 1.62 times
\textbf{faster}.

Probably, the C\# code is slower because:

\begin{itemize}
\item The object is still allocated on the heap, and thus there is an extra
  level of indirection to access the \lstinline{value} field.
\item The method call is optimized through a \emph{polymorphic inline cache}
  \cite{hoelzle_optimizing_1991}, that requires a guard check at each iteration.
\end{itemize}

Despite being only a microbenchmark, this result is very important as it proves
that our strategy of intermixing compile time and runtime can yield to better
performances than current techniques.  The result is even more impressive if
we take in account that dynamically typed languages as TLC are usually considered much
slower than the statically typed ones.

\section{Related Work}

Flexswitches are closely related to the concept of \emph{promotion}, as
described by \cite{PyPyJIT}, \cite{PyPyJIT09}.
Psyco is
a run-time specialiser for Python that uses promotion (called ``unlift'' in
\cite{DBLP:conf/pepm/Rigo04}). However, Psyco is a manually written JIT, is
not applicable to other languages and cannot be retargetted.  Psyco is a 
good example of how to implement flexswitches for targets that don't have the
limitations of the CLI.

The idea of promotion is a generalization of \emph{Polymorphic
  Inline Caches} \cite{hoelzle_optimizing_1991}, as well as the idea of using
runtime feedback to produce more efficient code
\cite{hoelzle_type_feedback_1994}.  The main difference between the two is 
that PICs only works on types, whereas promotion can work on every kind of value.

PyPy-style JIT compilers are hard to write manually, thus we chose to write a
JIT generator.  Tracing JIT compilers \cite{gal_hotpathvm_2006} also give
good results but are much easier to write, making the need for an automatic
generator less urgent.  However so far tracing JITs have less general
allocation removal techniques, which makes them get less speedup in a dynamic
language with boxing.  Another difference is that tracing JITs concentrate on
loops, which makes them produce a lot less code.  This issue is being addressed
by current research in PyPy \cite{PyPyTracing}.

The code generated by tracing JITs code typically contains guards; in recent research
\cite{gal_incremental_2006} on Java, these guards' behaviour is extended to be
similar to our promotion.  This has been used twice to implement a dynamic
language (JavaScript), by Tamarin\footnote{{\tt
http://www.mozilla.org/projects/tamarin/}} and in \cite{chang_efficient_2007}.

IronPython and Jython are two popular implementations of Python for,
respectively, the CLI and the JVM, whose approach differs fundamentally from
PyPy.  The source code of PyPy contains a Python interpreter, which the JIT
compiler is automatically generated from: the resulting executable contains
both the interpreter and the compiler, so that it is possible to compile only
the desired parts of the program.  On the other hand, both IronPython and
Jython implements only the compiler: both compile code lazily (when a Python
module is loaded), but since they do not exploit the extra information
potentially available at runtime, it is more a delayed static compilation than
a true JIT one.  As a result, they run Python programs much slower than their
equivalent written in
C\#\footnote{\texttt{http://shootout.alioth.debian.org/gp4/\\benchmark.php?test=all\&lang=iron\&lang2=csharp}}
or
Java\footnote{\texttt{http://blog.dhananjaynene.com/2008/07/performance-\\comparison-c-java-python-ruby-jython-jruby-groovy/}}.

The \emph{Dynamic Language Runtime}\footnote{\texttt{http://www.codeplex.com/dlr}}
(DLR) is a library designed to ease the implementation of dynamic languages
for .NET: the DLR is closely related to IronPython\footnote{In fact, the DLR
  started as a spin-off of IronPython, and nowadays the latter is based on the
  former.} and employs the techniques described above; thus, the remarks
about the differences between PyPy and IronPython apply to all DLR based
languages.

\section{Conclusion and Future Work}

In this paper we gave an overview of PyPy's JIT compiler generator,
which can automatically turn an interpreter into a JIT
compiler, requiring the language developers to only add few hints to
guide the generation process.

Then, we presented the CLI backend for PyPy's JIT compiler generator, whose
goal is to produce .NET bytecode at runtime.  We showed how it is possible to
circumvent intrinsic limitations of the virtual machine to implement
flexswitches.  As a result, we proved that the idea of \emph{JIT layering} is
worth of further exploration, as it makes possible for dynamically typed
languages to be even faster than their statically typed counterpart in some
cases.

As a future work, we want to explore different strategies to make the frontend
producing less code, maintaining comparable or better performances.  In
particular, we are working on a way to automatically detect loops in the user
code, as tracing JITs do \cite{gal_hotpathvm_2006}.  By compiling whole
loops at once, the backends should be able to produce better code.

At the moment, some bugs and minor missing features prevent the CLI JIT
backend to handle more complex languages such as Python and Smalltalk.  We are
confident that once these problems will be fixed, we will get performance
results comparable to TLC, as the other backends already demonstrate
\cite{PyPyJIT}.  However, if the current implementation of flexswitches will
turn out to be too slow for some purposes, alternative
implementation strategies could be explored by considering the novel features 
offered the new generation of virtual machines.

In particular, the \emph{Da Vinci Machine
  Project} \footnote{\texttt{http://openjdk.java.net/projects/mlvm/}} is exploring and
implementing new features to ease the implementation of dynamic languages on
top of the JVM: some of these features, such as the new
\emph{invokedynamic}\footnote{\texttt{http://jcp.org/en/jsr/detail?id=292}} instruction and the \emph{tail call
  optimization} can probably be exploited by a potential JVM backend to
generate even more efficient code.



\section*{Acknowledgements}

The authors would like to thank Carl Friedrich Bolz, Maciej Fijalkowski and
the referees of ICOOOLPS'09 for helpful comments on earlier versions of this
paper.



\bibliographystyle{plain}
\bibliography{main}

\end{document}
