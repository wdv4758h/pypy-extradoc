\documentclass{llncs}


 %%%% Compression Light+: LNCS margin reduced by +/-7mm along all edges (RG).
%\textwidth=130mm   % LNCS: 122mm
%\textheight=203mm  % LNCS: 193mm

\renewcommand{\baselinestretch}{0.97}

\usepackage{amssymb}
\usepackage{amsmath}
\usepackage[sans]{dsfont}
\usepackage{color}
\usepackage{ifthen}
\usepackage{xspace}
\usepackage{listings}
\usepackage{fancyvrb}
\usepackage{multirow}
\usepackage[pdftex]{graphicx}

%\input{macros}

\pagestyle{plain}

%\lstset{mathescape=true,language=Java,basicstyle=\tt,keywordstyle=\bf}
\lstset{language=Python,
        basicstyle=\scriptsize\ttfamily,
        keywordstyle=\color{blue}, % I couldn't find a way to make chars both bold and tt
        frame=none,
        stringstyle=\color{blue},
        fancyvrb=true,
        xleftmargin=20pt,xrightmargin=20pt,
        showstringspaces=false}

\setlength{\tabcolsep}{1ex}


%\renewcommand{\baselinestretch}{.98}
\newboolean{showcomments}
\setboolean{showcomments}{false}
\ifthenelse{\boolean{showcomments}}
  {\newcommand{\nb}[2]{
    \fbox{\bfseries\sffamily\scriptsize#1}
    {\sf\small$\blacktriangleright$\textit{#2}$\blacktriangleleft$}
   }
   \newcommand{\version}{\emph{\scriptsize$-$Id: main.tex 19055 2008-06-05 11:20:31Z cfbolz $-$}}
  }
  {\newcommand{\nb}[2]{}
   \newcommand{\version}{}
  }

\newcommand\dacom[1]{\nb{DA}{#1}}
\newcommand\cfbolz[1]{\nb{CFB}{#1}}
\newcommand\anto[1]{\nb{ANTO}{#1}}
\newcommand\arigo[1]{\nb{AR}{#1}}
\newcommand{\commentout}[1]{}

\let\oldcite=\cite

\renewcommand\cite[1]{\ifthenelse{\equal{#1}{XXX}}{[citation~needed]}{\oldcite{#1}}}


\begin{document}
\title{Tutorial Proposal: Writing Interpreters for Dynamic Languages Using PyPy
(and Getting Them Fast)}


\author{Carl Friedrich Bolz\inst{2} \email{cfbolz@gmx.de} \and Antonio Cuni\inst{1} \email{anto.cuni@gmail.com} \and Armin Rigo\inst{2} \email{arigo@tunes.org}}

\institute{DISI, University of Genova, Italy 
\and 
Softwaretechnik und Programmiersprachen
 Heinrich-Heine-Universit\"at D\"usseldorf}

\maketitle

The easiest way to implement a dynamic language such as JavaScript or Python is
to write an interpreter for it; however, interpreters are slow. An alternative
is to write a compiler; writing a compiler that targets a high level virtual
machine like CLI or JVM is easier than targeting a real CPU, but it still
require a lot of work, as IronPython, Jython, JRuby demonstrate. Moreover,
the various platforms have to be targeted independently.

The PyPy project \cite{RiBo07_223} aims to make the implementation of dynamic
languages easier by providing a toolchain that allows to translate an
interpreter to various target platforms, including the JVM, .NET and C/Posix. 
In addition, we are working on being able to automatically transform the
interpreter into a specializing JIT-compiler to vastly increase performance.
The goal is to minimize the effort required to get a fast implementation of a
dynamic language.



Outline of the tutorial:

 - introduction to PyPy
 - present the small language and its interpreter
 - show how to compile the interpreter to various platforms
 - show how to apply JIT-generator


\bibliographystyle{plain}
\bibliography{main}

\end{document}
