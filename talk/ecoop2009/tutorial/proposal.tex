\documentclass{llncs}


 %%%% Compression Light+: LNCS margin reduced by +/-7mm along all edges (RG).
%\textwidth=130mm   % LNCS: 122mm
%\textheight=203mm  % LNCS: 193mm

\renewcommand{\baselinestretch}{0.97}

\usepackage{amssymb}
\usepackage{amsmath}
\usepackage[sans]{dsfont}
\usepackage{color}
\usepackage{ifthen}
\usepackage{xspace}
\usepackage{listings}
\usepackage{fancyvrb}
\usepackage{multirow}
\usepackage[pdftex]{graphicx}

%\input{macros}

\pagestyle{plain}

%\lstset{mathescape=true,language=Java,basicstyle=\tt,keywordstyle=\bf}
\lstset{language=Python,
        basicstyle=\scriptsize\ttfamily,
        keywordstyle=\color{blue}, % I couldn't find a way to make chars both bold and tt
        frame=none,
        stringstyle=\color{blue},
        fancyvrb=true,
        xleftmargin=20pt,xrightmargin=20pt,
        showstringspaces=false}

\setlength{\tabcolsep}{1ex}


%\renewcommand{\baselinestretch}{.98}
\newboolean{showcomments}
\setboolean{showcomments}{false}
\ifthenelse{\boolean{showcomments}}
  {\newcommand{\nb}[2]{
    \fbox{\bfseries\sffamily\scriptsize#1}
    {\sf\small$\blacktriangleright$\textit{#2}$\blacktriangleleft$}
   }
   \newcommand{\version}{\emph{\scriptsize$-$Id: main.tex 19055 2008-06-05 11:20:31Z cfbolz $-$}}
  }
  {\newcommand{\nb}[2]{}
   \newcommand{\version}{}
  }

\newcommand\dacom[1]{\nb{DA}{#1}}
\newcommand\cfbolz[1]{\nb{CFB}{#1}}
\newcommand\anto[1]{\nb{ANTO}{#1}}
\newcommand\arigo[1]{\nb{AR}{#1}}
\newcommand{\commentout}[1]{}

\let\oldcite=\cite

\renewcommand\cite[1]{\ifthenelse{\equal{#1}{XXX}}{[citation~needed]}{\oldcite{#1}}}


\begin{document}
\title{Tutorial Proposal: Writing Interpreters for Dynamic Languages Using PyPy
(and Getting Them Fast)}


\author{Carl Friedrich Bolz\inst{2} \email{cfbolz@gmx.de} \and Antonio Cuni\inst{1} \email{anto.cuni@gmail.com} \and Armin Rigo\inst{2} \email{arigo@tunes.org}}

\institute{DISI, University of Genova, Italy 
\and 
Softwaretechnik und Programmiersprachen
 Heinrich-Heine-Universit\"at D\"usseldorf}
\maketitle


\begin{abstract}
We propose to give a tutorial on how to use the PyPy project to write
interpreters for dynamic languages. PyPy supports compiling such interpreters
to various target environments (like .NET or C/Posix) and automatically
generating a Just-In-Time compiler for them. PyPy is thus an emerging tool to
help implement fast, portable multi-platform interpreters with less effort.
\end{abstract}

The easiest way to implement a dynamic language such as JavaScript or Python is
to write an interpreter for it; however, interpreters are slow. An alternative
is to write a compiler; writing a compiler that targets a high level virtual
machine like CLI or JVM is easier than targeting a real CPU, but it still
requires a lot of work, as demonstrated by the IronPython, Jython, JRuby
projects. Moreover, the various platforms (either VM or real hardware) have to
be targeted independently.

The PyPy project\footnote{\url{http://codespeak.net/pypy}} \cite{RigoPedroni06}
\cite{RiBo07_223}
aims to make the implementation of dynamic
languages easier by providing a toolchain that allows to compile an
interpreter to various target platforms, including the JVM, .NET and C/Posix. 
In addition, we are working on being able to automatically transform the
interpreter into a specializing JIT-compiler to vastly increase performance.
The goal is to minimize the effort required to get a fast implementation for a
dynamic language.

This tutorial will provide insights into the usage of the PyPy translation
toolchain by giving a step-by-step introduction. The topics covered are how to
write an interpreter for a dynamic language with PyPy and make it possible to
automatically generate a JIT for it.  We will showcase a typical interpreter for a
small object-oriented dynamic language as a running example.


Outline of the tutorial:
\begin{itemize}
 \item \textbf{Introduction to PyPy:} Introduce the PyPy project and give an
 overview of its goals.
 \item \textbf{Overview of the example language:} Present the small language and
 its interpreter that will be used as a running example throughout the tutorial.
 \item \textbf{Translation:} Show how to compile the interpreter to various
 platforms, including .NET and C/Posix.
 \item \textbf{JIT-Generation:} Show how to apply the JIT-generator to the
 interpreter by placing a few hints to automatically get a JIT for the language.
\end{itemize}


There is a current resurgence in research on JIT-compilers for dynamic
languages, mostly focusing on JavaScript implementations. PyPy generates such
JITs so that they don't have to be hand-written. This requires much less effort
for the language implementor. PyPy is the first project that attempts to
generate JITs at this scale \cite{PyPyJIT}.

The presenters are part of the PyPy core development team and are all involved
in the ongoing research on the JIT generator. 

\bibliographystyle{plain}
\bibliography{proposal}

\end{document}
