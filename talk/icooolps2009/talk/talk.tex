\documentclass[utf8x]{beamer}

\mode<presentation>
{
  \usetheme{Warsaw}

  %\setbeamercovered{transparent}
}


\usepackage[english]{babel}

\usepackage[utf8x]{inputenc}

\usepackage{times}
\usepackage[T1]{fontenc}

\title[PyPy's Tracing JIT Compiler]{
    Tracing the Meta-Level: PyPy's Tracing JIT Compiler
}
\author[Bolz, Cuni, Fijalkowski, Rigo]
{
    \textcolor{green!50!black}{Carl~Friedrich~Bolz}\inst{1} \and
    Antonio Cuni\inst{2} \and
    Maciej Fija\l{}kowski\inst{3} \and
    Armin Rigo
}

\institute[Düsseldorf]
{
    \inst{1}
    Softwaretechnik und Programmiersprachen\\ Heinrich-Heine-Universit\"at D\"usseldorf
    \and%
    \vskip-2mm
    \inst{2}
    University of Genova, Italy
    \and%
    \vskip-2mm
    \inst{3}
    merlinux GmbH
}

\date{6th of July 2009, ICOOOLPS '09, Genova}


% Delete this, if you do not want the table of contents to pop up at
% the beginning of each subsection:
%\AtBeginSubsection[]
%{
%  \begin{frame}<beamer>
%    \frametitle{Outline}
%    \tableofcontents[currentsection,currentsubsection]
%  \end{frame}
%}


% If you wish to uncover everything in a step-wise fashion, uncomment
% the following command:

%\beamerdefaultoverlayspecification{<+->}


\begin{document}

\begin{frame}
  \titlepage
\end{frame}

%\begin{frame}
%  \frametitle{Outline}
%  \tableofcontents
  % You might wish to add the option [pausesections]
%\end{frame}

\begin{frame}
    \frametitle{Motivation}
    \begin{itemize}
    \item writing good JIT compilers for dynamic programming languages is hard and error-prone
    \item tracing JIT compilers are a new approach to JITs that are supposed to be easier
    \item what happens when a tracing JIT is applied ``one level down'', i.e. to an interpreter
    \item how to solve the occurring problems
    \end{itemize}
\end{frame}

\begin{frame}
    \frametitle{Context: The PyPy Project}
    \begin{itemize}
    \item a general environment for implementing dynamic languages
    \item contains a compiler for a subset of Python (``RPython'')
    \item interpreters for dynamic languages written in that subset
    \item various interpreters written with PyPy: Python, Prolog, Smalltalk, Scheme, JavaScript, GameBoy emulator
    \item can be translated to a variety of target environment: C, JVM, .NET
    \end{itemize}
\end{frame}

\begin{frame}
    \frametitle{Tracing JIT Compilers}
    \begin{itemize}
    \item idea from Dynamo project: dynamic rewriting of machine code
    \item later used for a lightweight Java JIT
    \item seems to also work for dynamic languages (see TraceMonkey)
    \end{itemize}
    \pause
    \begin{block}{Basic Assumption of a Tracing JIT}
        \begin{itemize}
        \item programs spend most of their time executing loops
        \item several iterations of a loop are likely to take similar code paths
        \end{itemize}
    \end{block}
\end{frame}

\begin{frame}
    \frametitle{Tracing JIT Compilers}
    \begin{itemize}
    \item mixed-mode execuction environment
    \item at first, everything is interpreted
    \item lightweight profiling to discover hot loops
    \item code generation only for common paths of hot loops
    \item when a hot loop is discovered, start to produce a trace
    \item when a full loop is traced, the trace is converted to machine code
    \end{itemize}
\end{frame}

\frame[containsverbatim, plain, shrink=10]{
  \frametitle{Example}
  \begin{verbatim}
def strange_sum(n):
    result = 0
    while n >= 0:
        result = f(result, n)
        n -= 1
    return result

def f(a, b):
    if b % 46 == 41:
        return a - b
    else:
        return a + b










\end{verbatim}
}

\frame[containsverbatim, plain, shrink=10]{
  \frametitle{Example}
  \begin{verbatim}
def strange_sum(n):
    result = 0
    while n >= 0:
        result = f(result, n)
        n -= 1
    return result

def f(a, b):
    if b % 46 == 41:
        return a - b
    else:
        return a + b

# loop_header(result0, n0)
# i0 = int_mod(n0, Const(46))
# i1 = int_eq(i0, Const(41))
# guard_false(i1)
# result1 = int_add(result0, n0)
# n1 = int_sub(n0, Const(1))
# i2 = int_ge(n1, Const(0))
# guard_true(i2)
# jump(result1, n1)
\end{verbatim}
}

\begin{frame}
    \frametitle{(Dis-)Advantages of Tracing JITs}
    \begin{block}{Good Points of the Approach}
        \begin{itemize}
        \item easy and fast machine code generation: needs so support only one path
        \item (things are more complex, but let's ignore that for now)
        \item interpreter does a lot of the work
        \item can be added to an existing interpreter unobtrusively
        \item automatic inlining
        \item produces comparatively little machine code
        \end{itemize}
    \end{block}
    \pause
    \begin{block}{Bad Points of the Approach}
        \begin{itemize}
        \item unclear whether assumptions are true often enough
        \item switching between interpretation and machine code execution takes time
        \end{itemize}
    \end{block}
\end{frame}

\begin{frame}
    \frametitle{Applying a Tracing JIT to an Interpreter}
    \begin{itemize}
    \item Question: What happens if the program is itself a bytecode interpreter?
    \item the (most important) hot loop of a bytecode interpreter is the bytecode dispatch loop
    \item Assumption violated: consecutive iterations of the dispatch loop will usually take very different code paths
    \item what can we do?
    \end{itemize}
    \pause
    \begin{block}{Terminology}
        \begin{itemize}
        \item \emph{tracing interpreter:} the interpreter that originally runs the program and produces traces
        \item \emph{language interpreter:} the bytecode interpreter run on top
        \item \emph{user program:} the program run by the language interpreter
        \end{itemize}
    \end{block}
\end{frame}

\frame[containsverbatim, plain, shrink=10]{
  \begin{verbatim}
def interpret(bytecode, a):
    regs = [0] * 256
    pc = 0
    while True:
        opcode = ord(bytecode[pc])
        pc += 1
        if opcode == JUMP_IF_A:
            target = ord(bytecode[pc])
            pc += 1
            if a:
                pc = target
        elif opcode == MOV_R_A:
            n = ord(bytecode[pc])
            pc += 1
            a = regs[n]
        elif opcode == MOV_A_R:
            ...
        elif opcode == ADD_R_TO_A:
            ...
        elif opcode == DECR_A:
            a -= 1
        elif opcode == RETURN_A:
            return a



  \end{verbatim}
}

\frame[containsverbatim, plain, shrink=10]{
  \begin{verbatim}
def interpret(bytecode, a):
    regs = [0] * 256                    |
    pc = 0                              |  # Example bytecode
    while True:                         |  # Square the accumulator:
        opcode = ord(bytecode[pc])      |
        pc += 1                         |  MOV_A_R     0  # i = a
        if opcode == JUMP_IF_A:         |  MOV_A_R     1  # copy of 'a'
            target = ord(bytecode[pc])  |
            pc += 1                     |  # 4:
            if a:                       |  MOV_R_A     0  # i--
                pc = target             |  DECR_A
        elif opcode == MOV_R_A:         |  MOV_A_R     0
            n = ord(bytecode[pc])       |
            pc += 1                     |  MOV_R_A     2  # res += a
            a = regs[n]                 |  ADD_R_TO_A  1
        elif opcode == MOV_A_R:         |  MOV_A_R     2
            ...                         |
        elif opcode == ADD_R_TO_A:      |  MOV_R_A     0  # if i!=0:
            ...                         |  JUMP_IF_A   4  #    goto 4
        elif opcode == DECR_A:          |
            a -= 1                      |  MOV_R_A     2  # return res
        elif opcode == RETURN_A:        |  RETURN_A                     
            return a                    |



  \end{verbatim}
}

\frame[containsverbatim, plain, shrink=10]{
    \frametitle{Trace}
    ~\\
    Resulting trace when tracing bytecode \texttt{DECR\_A}:
    \begin{verbatim}

loop_start(a0, regs0, bytecode0, pc0)
opcode0 = strgetitem(bytecode0, pc0)
pc1 = int_add(pc0, Const(1))
guard_value(opcode0, Const(7))
a1 = int_sub(a0, Const(1))
jump(a1, regs0, bytecode0, pc1)
\end{verbatim}
}

\begin{frame}
    \frametitle{Idea for a Solution}
    \begin{itemize}
    \item goal: try to trace the loops \emph{in the user program,}
          and not just one iteration of the bytecode dispatch loop
    \item tracing interpreter needs information about the language interpreter
    \item provided by adding \emph{three hints} to the language interpreter
    \end{itemize}
    \pause
    \begin{block}{Hints Give Information About:}
    \begin{itemize}
    \item which variables make up the program counter of the language interpreter (together those are called \emph{position key})
    \item where the bytecode dispatch loop is
    \item which bytecodes can correspond to backward jumps
    \end{itemize}
    \end{block}
\end{frame}

\frame[containsverbatim, plain, shrink=10]{
    \frametitle{Interpreter with Hints}
\begin{verbatim}
tlrjitdriver = JitDriver(['pc', 'bytecode'])

def interpret(bytecode, a):
    regs = [0] * 256
    pc = 0
    while True:
        tlrjitdriver.start_dispatch_loop()
        opcode = ord(bytecode[pc])
        pc += 1
        if opcode == JUMP_IF_A:
            target = ord(bytecode[pc])
            pc += 1
            if a:
                pc = target
                if target < pc:
                    tlrjitdriver.backward_jump()
        elif opcode == MOV_A_R:
            ... # rest unmodified
\end{verbatim}
}

\begin{frame}
    \frametitle{Modifying Tracing}
    \begin{itemize}
    \item goal: try to trace the loops \emph{in the user program,}
          and not just one iteration of the bytecode dispatch loop
    \item tracing interpreter stops tracing only when:
        \begin{itemize}
        \item it sees a backward jump in the language interpreter
        \item the position key of the language interpreter matches an ealier value
        \end{itemize}
    \item in this way, full user loops are traced
    \end{itemize}
\end{frame}
\frame[containsverbatim, plain, shrink=20]{
    \frametitle{Result When Tracing \texttt{SQUARE}}
\begin{verbatim}
loop_start(a0, regs0, bytecode0, pc0)
# MOV_R_A 0
opcode0 = strgetitem(bytecode0, pc0)
pc1 = int_add(pc0, Const(1))
guard_value(opcode0, Const(2))
n1 = strgetitem(bytecode0, pc1)
pc2 = int_add(pc1, Const(1))
a1 = list_getitem(regs0, n1)
# DECR_A
# MOV_A_R 0
# MOV_R_A 2
# ADD_R_TO_A 1
# MOV_A_R 2
# MOV_R_A 0
...
# JUMP_IF_A 4
opcode6 = strgetitem(bytecode0, pc13)
pc14 = int_add(pc13, Const(1))
guard_value(opcode6, Const(3))
target0 = strgetitem(bytecode0, pc14)
pc15 = int_add(pc14, Const(1))
i1 = int_is_true(a5)
guard_true(i1)
jump(a5, regs0, bytecode0, target0)
\end{verbatim}
}

\begin{frame}
    \frametitle{What Have We Won?}
    \begin{itemize}
    \item trace corresponds to one loop of the user program
    \item however, most operations are concerned with manipulating bytecode and program counter
    \item bytecode and program counter are part of the position key
    \item thus they are constant at the beginning of the loop
    \item therefore they can and should be constant-folded
    \end{itemize}
\end{frame}

\frame[containsverbatim, plain, shrink=20]{
    \frametitle{Result When Tracing \texttt{SQUARE} With Constant-Folding}
\begin{verbatim}
loop_start(a0, regs0)
# MOV_R_A 0
a1 = list_getitem(regs0, Const(0))
# DECR_A
a2 = int_sub(a1, Const(1))
# MOV_A_R 0    
list_setitem(regs0, Const(0), a2)
# MOV_R_A 2
list_getitem(regs0, Const(2))
# ADD_R_TO_A  1
i0 = list_getitem(regs0, Const(1))
a4 = int_add(a3, i0)
# MOV_A_R 2
list_setitem(regs0, Const(2), a4)
# MOV_R_A 0
a5 = list_getitem(regs0, Const(0))
# JUMP_IF_A 4
i1 = int_is_true(a5)
guard_true(i1)
jump(a5, regs0)
\end{verbatim}
}

\begin{frame}
    \frametitle{Results}
    \begin{itemize}
    \item almost only computations related to the user program remain
    \item list of registers is only vestige of language interpreter
    \end{itemize}
    \pause
    \begin{block}{Timing Results Computing Square of 10'000'000}
    \begin{tabular}{|l|l|r|r|}
\hline
& &Time (ms) &speedup\\
\hline
1 &No JIT &442.7 $\pm$ 3.4 &1.00\\
2 &JIT, Normal Trace Compilation &1518.7 $\pm$ 7.2 &0.29\\
3 &JIT, Unrolling of Interp. Loop &737.6 $\pm$ 7.9 &0.60\\
4 &JIT, Full Optimizations &156.2 $\pm$ 3.8 &2.83\\
\hline
\end{tabular}
\end{block}
\end{frame}

\begin{frame}
    \frametitle{Scaling to Large Interpreters?}
    \begin{itemize}
    \item we can apply this approach to PyPy's Python interpreter (70 KLOC)
    \item speed-ups promising (see next slide)
    \item no Python-specific bugs!
    \end{itemize}
\end{frame}

\frame[containsverbatim, plain, shrink=10]{
    \frametitle{Timings for Python Interpreter}
\begin{verbatim}
def f(a):
    t = (1, 2, 3)
    i = 0
    while i < a:
        t = (t[1], t[2], t[0])
        i += t[0]
    return i
\end{verbatim}
\begin{block}{Timings of \texttt{f(10000000)}}
\begin{tabular}{|l|l|r|r|}
\hline
& &Time (ms) &speedup\\
\hline
1 &PyPy compiled to C, no JIT &1793 $\pm$ 11 &1.00\\
2 &PyPy comp'd to C, with JIT &483 $\pm$ 6 &3.71\\
3 &CPython 2.6 &1869 $\pm$ 11 & 0.96\\
4 &CPython 2.6 + Psyco 1.6 & 511 $\pm$ 7 &3.51\\\hline
\end{tabular}
\end{block}
}


\begin{frame}
    \frametitle{Conclusions}
    \begin{itemize}
    \item some small changes to a tracing JIT makes it possible to effectively apply it to bytecode interpreters
    \item result is similar to a tracing JIT for that language
    \item bears resemblance to partial evaluation, arrived at by different means
    \item maybe enough to write exactly one tracing JIT?
    \end{itemize}
    \pause
    \begin{block}{Outlook}
        \begin{itemize}
        \item better optimizations of the traces
        \item escape analysis
        \item optimize frame objects
        \item speed up tracing itself
        \item apply to other interpreters and larger programs
        \end{itemize}
    \end{block}
\end{frame}

\begin{frame}
    \frametitle{Thank you! Questions?}
    \begin{itemize}
    \item some small changes to a tracing JIT makes it possible to effectively apply it to bytecode interpreters
    \item result is similar to a tracing JIT for that language
    \item bears resemblance to partial evaluation, arrived at by different means
    \item maybe enough to write exactly one tracing JIT?
    \end{itemize}
    \begin{block}{Outlook}
        \begin{itemize}
        \item better optimizations of the traces
        \item escape analysis
        \item optimize frame objects
        \item speed up tracing itself
        \item apply to other interpreters and larger programs
        \end{itemize}
    \end{block}
\end{frame}

\end{document}
