
\documentclass[utf8x, 14pt]{beamer}

\mode<presentation>
{
  \usetheme{Warsaw}
}

\usepackage[english]{babel}

\usepackage[utf8x]{inputenc}
\usepackage{tikz}

\usepackage{times}
\usepackage[T1]{fontenc}
\usepackage{color}

\title{The speed of PyPy}

\author{Maciej Fijałkowski}

\institute[merlinux GmbH]
{ merlinux GmbH }

\date{Pycon 2010, February 20th 2010, Atlanta}

\begin{document}

\begin{frame}
  \titlepage
  \begin{figure}
    \begin{tabular}{c c c}
    \includegraphics[width=.30\textwidth]{../common/pypy-logo.png}
    &
    \hspace{2cm}
    &
    \includegraphics[width=.25\textwidth]{../common/merlinux-logo.png}
    \end{tabular}
  \end{figure}
\end{frame}

\begin{frame}
  \frametitle{How fast is PyPy?}
  \pause
  \begin{itemize}
    \item pretty fast, in places
    \item slower than cpython in other places
    \item overall, it depends
    \item graphs
  \end{itemize}
\end{frame}

\begin{frame}
  \frametitle{JIT - what's that about?}
  \begin{itemize}
      \pause
    \item {\bf JIT is not a magical device!}
      \pause
    \item removes bytecode overhead
    \item removes frame overhead
    \item can make runtime decisions
    \item more classic optimization that can follow
  \end{itemize}
\end{frame}

\begin{frame}
  \frametitle{The main idea}
  \begin{itemize}
    \item python has advanced features (frame introspection,
      arbitrary code execution, overloading globals)
    \item with JIT, you don't pay for them if you don't use
      them
    \item however, you pay if you use them, but they work
  \end{itemize}
\end{frame}

\begin{frame}
  \frametitle{A piece of advice}
  \begin{itemize}
    \item don't use advanced features if you don't have to
  \end{itemize}
\end{frame}

\begin{frame}
  \frametitle{Tracing JIT}
  \begin{itemize}
    \item compiler traces the actual execution of Python program
    \item then compiles linear path to assembler
    \item example
    \item mostly for speeding up loops and to certain extent
      recursion
  \end{itemize}
\end{frame}

\begin{frame}[fragile]
  \frametitle{Removing frame overhead}
  \begin{verbatim}
x = y + z
  \end{verbatim}
  \begin{itemize}
    \item above has 5 frame accesses
    \item they can all be removed (faster!)
      \pause
    \item {\bf this enables further optimizations}
  \end{itemize}
\end{frame}

\begin{frame}[fragile]
  \frametitle{Removing object boxing}
  \begin{verbatim}
i = 0
while i < 100:
  i += 1
  \end{verbatim}
  \begin{itemize}
    \item for each iteration we do a comparison and addition
    \item xxx integers on valuestack and xxx integers in locals
    \item all boxing can be removed
  \end{itemize}
\end{frame}

\begin{frame}
  \frametitle{Variable access costs}
  \begin{itemize}
    \item local access costs nothing
    \item global access is cheap, if you don't change global {\ttfamily \_\_dict\_\_} too much XXX rephrase
  \end{itemize}
\end{frame}

\begin{frame}
  \frametitle{Frame escapes}
  \begin{itemize}
    \item {\bf JIT} normally removes frame overhead, but
    \item calling {\ttfamily sys.\_getframe()}, {\ttfamily sys.exc\_info()}
    \item exception escaping
    \item prevents a lot of optimizations
  \end{itemize}
\end{frame}

\begin{frame}
  \frametitle{Shared dicts (aka hidden classes)}
  \begin{itemize}
    \item instance {\ttfamily \_\_dict\_\_ } lookup becomes an array lookup
    \item if you're evil, it'll bail back to dict lookup
      \pause
    \item {\bf only for newstyle classes}
  \end{itemize}
\end{frame}

\begin{frame}
  \frametitle{Version tags}
  \begin{itemize}
    \item dicts on types are version-controlled
    \item this means method lookup can be removed
      \pause
    \item ... if you don't modify them too often
    \item counters on classes are bad
  \end{itemize}
\end{frame}

\begin{frame}
  \frametitle{Call costs}
  \begin{itemize}
    \item calls can be inlined
    \item simple arguments are by far the best
    \item avoid {\ttfamily *args} and {\ttfamily **kwds}
    \item however, {\ttfamily f(a=3, b=c)} is fine
  \end{itemize}
\end{frame}

\begin{frame}
  \frametitle{Allocation patterns}
  \begin{itemize}
    \item PyPy uses a moving GC (like JVM, .NET, etc.)
    \item pretty efficient for usecases with a lot of
      short-living objects
    \item objects are smaller than on CPython
    \item certain behaviors are different than on CPython
  \end{itemize}
\end{frame}

\begin{frame}
  \frametitle{Differencies}
  \begin{itemize}
    \item no refcounting semantics
    \item {\ttfamily id(obj)} can be expensive as it's a complex
      operation on a moving GC
    \item a large list of new objects is a bad case behavior
  \end{itemize}
\end{frame}

\begin{frame}
  \frametitle{General rules}
  \begin{itemize}
    \item don't try to outsmart your compiler
    \item simple is better than complex
    \item metaprogramming is your friend
    \item measurment is the only meaningful way to check
  \end{itemize}
\end{frame}

\begin{frame}
  \frametitle{Problems}
  \begin{itemize}
    \item long traces - tracing is slow
    \item megamorphic calls
    \item metaclasses
    \item class global state
      \pause
    \item years of optimizations against CPython
  \end{itemize}
\end{frame}

\begin{frame}
  \frametitle{Future}
  \begin{block}{release end March}
    \begin{itemize}
      \item will contain a working JIT
      \item will not speed up all cases
      \item might eat all your memory
    \end{itemize}
  \end{block}
\end{frame}

\begin{frame}
  \frametitle{That's all!}
  \begin{itemize}
    \item Q \& A
    \item http://morepypy.blogspot.com
    \item http://pypy.org
    \item http://merlinux.eu
  \end{itemize}
\end{frame}

\end{document}
