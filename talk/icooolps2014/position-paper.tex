
\documentclass{sigplanconf}

% The following \documentclass options may be useful:

% preprint      Remove this option only once the paper is in final form.
% 10pt          To set in 10-point type instead of 9-point.
% 11pt          To set in 11-point type instead of 9-point.
% authoryear    To obtain author/year citation style instead of numeric.

\usepackage[utf8]{inputenc}

\usepackage{amsmath}


\begin{document}

\special{papersize=8.5in,11in}
\setlength{\pdfpageheight}{\paperheight}
\setlength{\pdfpagewidth}{\paperwidth}

\conferenceinfo{ICOOOLPS workshop 2014}{July 28th, 2014, Uppsala, Sweden}
\copyrightyear{2014}
\copyrightdata{978-1-nnnn-nnnn-n/yy/mm}
\doi{nnnnnnn.nnnnnnn}

% Uncomment one of the following two, if you are not going for the
% traditional copyright transfer agreement.

%\exclusivelicense                % ACM gets exclusive license to publish,
                                  % you retain copyright

%\permissiontopublish             % ACM gets nonexclusive license to publish
                                  % (paid open-access papers,
                                  % short abstracts)

%% \titlebanner{banner above paper title}        % These are ignored unless
%% \preprintfooter{short description of paper}   % 'preprint' option specified.

\title{Title Text}
\subtitle{Position Paper, ICOOOLPS'14}

\authorinfo{Remi Meier}
           {Department of Computer Science\\ ETH Zürich}
           {remi.meier@inf.ethz.ch}
\authorinfo{Armin Rigo}
           {www.pypy.org}
           {arigo@tunes.org}

\maketitle

\begin{abstract}
This is the text of the abstract.
\end{abstract}

\category{CR-number}{subcategory}{third-level}

% general terms are not compulsory anymore,
% you may leave them out
%% \terms
%% term1, term2

\keywords
transactional memory, dynamic languages, parallelism, global interpreter lock

\section{Introduction}

\subsection*{Issue}
efficiently supporting multi-CPU usage on dynamic languages that were designed with GIL semantics in
mind

(supporting (large) atomic blocks for synchronization)

\subsection*{Our Position}
Current solutions for replacing the GIL include STM, HTM, and
fine-grained locking. STM is usually too slow, HTM very limited, and
locking suffers from complexity that makes it hard to implement
correctly. We argue that the best way forward is still STM and that
its performance problem can be solved.

%% Current solutions like STM, HTM, and fine-grained locking are slow, hard
%% to implement correctly, and don't fit the specific problems of dynamic
%% language.  STM is the best way forward but has bad performance, so we
%% fix that.

\section{Discussion}
\paragraph{dynamic language VM problems}

- high allocation rate (short lived objects)\\
- (don't know anything about the program that runs until it actually runs: arbitrary atomic block size)

\paragraph{GIL}

- nice semantics\\
- easy support of atomic blocks\\
- no parallelism

\paragraph{fine-grained locking}

- support of atomic blocks?\\
- hard to get right (deadlocks, performance, lock-granularity)\\
- very hard to get right for a large language\\
- hard to retro-fit, as all existing code assumes GIL semantics\\
- (there are some semantic differences, right? not given perfect lock-placement, but well)
( http://www.jython.org/jythonbook/en/1.0/Concurrency.html )

\paragraph{multiprocessing / no-sharing models}

- often needs major restructuring of programs (explicit data exchange)\\
- sometimes communication overhead is too large\\
- shared memory is a problem, copies of memory are too expensive

\paragraph{HTM}

- false-sharing on cache-line level\\
- limited capacity (caches, undocumented)\\
- random aborts (haswell)\\
- generally: transaction-length limited (no atomic blocks)

\paragraph{STM}

- overhead (100-1000\%) (barrier reference resolution, kills performance on low \#cpu)
(FastLane: low overhead, not much gain)\\
- unlimited transaction length (easy atomic blocks)

\section{Potential Approach}
possible solution:\\
- use virtual memory paging to somehow lower the STM overhead\\
- tight integration with GC and jit?


\appendix
\section{Appendix Title}

This is the text of the appendix, if you need one.

\acks

Acknowledgments, if needed.

% We recommend abbrvnat bibliography style.

\bibliographystyle{abbrvnat}

% The bibliography should be embedded for final submission.

\begin{thebibliography}{}
\softraggedright

\bibitem[Smith et~al.(2009)Smith, Jones]{smith02}
P. Q. Smith, and X. Y. Jones. ...reference text...

\end{thebibliography}


\end{document}

%                       Revision History
%                       -------- -------
%  Date         Person  Ver.    Change
%  ----         ------  ----    ------

%  2013.06.29   TU      0.1--4  comments on permission/copyright notices
