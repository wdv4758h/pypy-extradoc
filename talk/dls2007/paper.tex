\documentclass{acm_proc_article-sp}

\begin{document}

\title{Get Your Own Just-In-Time Specializing Compiler For Free}

\numberofauthors{2}
\author{
\alignauthor Armin Rigo\\
       \affaddr{Heinrich-Heine-Universit�t D�sseldorf}\\
       \affaddr{Institut f�r Informatik}\\ 
       \affaddr{Universit�tsstra{\ss}e 1}\\
       \affaddr{D-40225 D�sseldorf}\\
       \affaddr{Deutschland}\\
       \email{arigo@tunes.org}
\alignauthor Samuele Pedroni\\
       \affaddr{Open End AB}\\
       \affaddr{Norra �gatan 10A}\\
       \affaddr{416 64  G�teborg}\\
       \affaddr{Sweden}\\
       \email{pedronis@strakt.com}
}
\date{31 May 2007}
\maketitle

%\category{D.3.4}{Programming Languages}{Processors}[code generation,
%interpreters, run-time environments]
%\category{F.3.2}{Logics and Meanings of Programs}{Semantics of Programming
%Languages}[program analysis]

\begin{abstract}
PyPy's translation tool-chain -- from the interpreter written in RPython
to generated VMs for low-level platforms -- is now able to extend those
VMs with an automatically generated dynamic compiler, derived from the
interpreter. This is achieved by a pragmatic application of partial
evaluation techniques guided by a few hints added to the source of the
interpreter. Crucial for the effectiveness of dynamic compilation is
the use of run-time information to improve compilation results: in
our approach, a novel powerful primitive called "promotion" that "promotes"
run-time values to compile-time is used to that effect.  In this paper,
we describe it along with other novel techniques that allow the approach
to scale to something as large as PyPy's Python interpreter.
\footnote{This research was partially supported by the EU funded
 project: IST 004779 PyPy (PyPy: Implementing Python in Python).}
\end{abstract}

\section{Introduction}

Dynamic compilers are resource costly to write and hard to maintain,
but highly desirable for competitive performance. Straight-forward
bytecode interpreters are easier to write. Hybrid approaches have been
experimented with \cite{REJIT}, but this is clearly an area in need of
research and innovative approaches.

One of the central goals of the PyPy project \cite{PyPy} is to automatically
produce dynamic compilers from an interpreter, with as little
modifications of the interpreter code base itself as possible.

PyPy contains a complete interpreter for the Python language, written in
a high-level language, RPython, which is a subset of Python amenable to
static analysis.  It also contains a translation toolchain for compiling
this interpreter to either C (or C-like) environments, or to the higher
level environments provided by general-purpose virtual machines like
Java's and .NET.  The translation toolchain can input any RPython
program, although our focus was on the translation of RPython programs
that are interpreters for dynamic languages.\footnote{We also have an
interpreter for Prolog and the beginning of one for JavaScript.}

The translation framework uses control flow graphs in SSI format as its
intermediate representation (SSI is a stricter subset of SSA).  The
details of this process are beyond the scope of the present paper, and
have been presented in \cite{pypyvmconstruction}.
The present paper describes a
special optional transformation that we integrated with this translation
framework: deriving a dynamic compiler from the interpreter.  In other
words, our translation framework is able to input an interpreter for any
language (it works best for dynamic languages); as long as it is
written in RPython and contains a small number of extra hints, then it
can produce from it a complete virtual machine
\emph{that contains a just-in-time compiler for the dynamic language.}

Partial evaluation techniques should, at least theoretically,
allow such a derivation of a compiler from an interpreter
\cite{partial-evaluation}, but it
is not reasonable to expect the code produced for an input program by
a compiler derived using partial evaluation to be very good,
especially in the case of a dynamic language.  Essentially, the input
program doesn't contain enough information to generate good code; for
example the input program contains mostly no kind of type
information in that case.

What is really desired is not to produce a compiler doing static
ahead-of-time compilation, as classical partial evaluation would do,
but one capable of dynamic compilation, exploiting run-time
information in its result. Compilation should be able to suspend, let
the produced code run to collect run-time information (for example
language-level types), and then resume with this extra information.
This will allows the compiler to generate code optimized for the
effective run-time behaviour of the program.

Inspired by Psyco \cite{psyco-paper}, which is a hand-written dynamic compiler
based on partial evaluation for Python, we developed a technique -
\emph{promotion} - for our dynamic compiler generator. Simply put, promotion
on a value stops compilation and waits until the run-time reaches this
point.  When it does, the actual run-time value is promoted into a
compile-time constant, and compilation resumes with this extra
information.

Promotion is an essential technique to be able to generate really
dynamic compilers that can exploit run-time information.
Besides promotion (section \ref{promotion}),
the novel techniques introduced by PyPy that allow
the approach to scale are virtualizable structures
(section \ref{virtualizable}) and need-oriented
binding time analysis (section \ref{bta}).


\subsection{Overview of partial evaluation}

\def\code#1{\texttt{#1}}

Partial evaluation is the process of evaluating a function, say \code{f(x,
y)}, with only partial information about the values of its arguments,
say the value of the \code{x} argument only.  This produces a \emph{residual}
function \code{g(y)}, which takes less arguments than the original -- only
the information not specified during the partial evaluation process needs
to be provided to the residual function, in this example the \code{y}
argument.

Partial evaluation (PE) comes in two flavors:
%
\begin{enumerate}

\item\emph{On-line PE:} a compiler-like algorithm takes the source code of the
  function \code{f(x, y)} (or its intermediate representation, i.e.\ its
  control flow graph in PyPy's terminology), and some partial
  information, e.g.\ \code{x=5}.  From this, it produces the residual
  function \code{g(y)} directly, by following in which operations the
  knowledge \code{x=5} can be used, which loops can be unrolled, etc.

\item\emph{Off-line PE:} in many cases, the goal of partial evaluation is to
  improve performance in a specific application.  Assume that we have a
  single known function \code{f(x, y)} in which we think that the value of
  \code{x} will change slowly during the execution of our program -- much
  more slowly than the value of \code{y}.  An obvious example is a loop
  that calls \code{f(x, y)} many times with always the same value \code{x}.
  We could then use an on-line partial evaluator to produce a \code{g(y)}
  for each new value of \code{x}.  In practice, the overhead of the partial
  evaluator might be too large for it to be executed at run-time.
  However, if we know the function \code{f} in advance, and if we know
  \emph{which} arguments are the ones that we will want to partially evaluate
  \code{f} with, then we do not need a full compiler-like analysis of \code{f}
  every time the value of \code{x} changes.  We can precompute once and for
  all a specialized function \code{f1(x)}, which when called produces the
  residual function \code{g(y)} corresponding to \code{x}.  This is
  \emph{off-line partial evaluation;} the specialized function \code{f1(x)}
  is called a \emph{generating extension.}

\end{enumerate}

The PyPy JIT generation framework is based on off-line partial
evaluation.  The function called \code{f(x, y)} above is typically the main
loop of some interpreter written in RPython.  The size of the interpreter can range
from a three-liner used for testing purposes to the whole of PyPy's
Python interpreter.  In all cases, \code{x} stands for the input program
(the bytecode to interpret) and \code{y} stands for the input data (like a
frame object with the binding of the input arguments and local
variables).  Our framework is capable of automatically producing the
corresponding generating extension \code{f1(x)}, which takes an input
program only and produces a residual function \code{g(y)}.  This \code{f1(x)}
is a compiler\footnote{
    What we get in PyPy is more precisely a \emph{just-in-time compiler:}
    if promotion is used, compiling ahead of time is not possible.
}
for the very same language for which \code{f(x, y)} is
an interpreter.

Off-line partial evaluation is based on \emph{binding-time analysis,} which
is the process of determining among the variables used in a function (or
a set of functions) which ones are going to be known in advance and
which ones are not.  In the example of \code{f(x, y)}, such an analysis
would be able to infer that the constantness of the argument \code{x}
implies the constantness of many intermediate values used in the
function.  The \emph{binding time} of a variable determines how early the
value of the variable will be known.

Once binding times have been determined, one possible approach to
producing the generating extension itself is by self-applying on-line
partial evaluators.  This is known as the second Futamura projection
\cite{Futamura}.  So far it is unclear if this approach can lead to optimal
results, or even if it scales well.  In PyPy we selected a more direct
approach: the generating extension is produced by transformation of the
control flow graphs of the interpreter, guided by the binding times.  We
call this process \emph{timeshifting.}


\subsection{Related work}

XXX PE; Psyco; REJIT; ?


\section{Architecture and Principles}

PyPy contains a framework for generating just-in-time compilers using
off-line partial evaluation.  As such, there are three distinct phases:
%
\begin{enumerate}

\item\emph{Translation time:} during the normal translation of an RPython
  program, say PyPy's Python interpreter, we perform binding-time
  analysis and off-line specialization ("timeshifting") of the
  interpreter.  This produces a generating extension, which is linked
  with the rest of the program.

\item\emph{Compile time:} during the execution of the program, when a new
  bytecode is about to be interpreted, the generating extension is
  invoked instead.  As the generating extension is a compiler, all the
  computations it performs are called compile-time computations.  Its
  sole effect is to produce residual code.

\item\emph{Run time:} the normal execution of the program (which includes the
  time spent running the residual code created by the generating
  extension).

\end{enumerate}

Translation time is a purely off-line phase; compile time and run time
are actually highly interleaved during the execution of the program.


\subsection{Binding Time Analysis}
\label{bta}

At translation time, PyPy performs binding-time analysis of the source
RPython program after it has been turned to low-level graphs, i.e.\ at
the level at which operations manipulate pointer-and-structure-like
objects.

The binding-time terminology that we are using in PyPy is based on the
colors that we use when displaying the control flow graphs:
%
\begin{itemize}
\item\emph{Green} variables contain values that are known at compile-time;
\item\emph{Red} variables contain values that are not known until run-time.
\end{itemize}

The binding-time analyzer of our translation tool-chain is based on the
same type inference engine that is used on the source RPython program,
the annotator.  In this mode, it is called the \emph{hint-annotator;} it
operates over input graphs that are already low-level instead of
RPython-level, and propagates annotations that do not track types but
value dependencies and manually-provided binding time hints.

The normal process of the hint-annotator is to propagate the binding
time (i.e.\ color) of the variables using the following kind of rules:
%
\begin{itemize}

\item For a foldable operation (i.e.\ one without side effect and which
  depends only on its argument values), if all arguments are green,
  then the result can be green too.

\item Non-foldable operations always produce a red result.

\item At join points, where multiple possible values (depending on control
  flow) are meeting into a fresh variable, if any incoming value comes
  from a red variable, the result is red.  Otherwise, the color of the
  result might be green.  We do not make it eagerly green, because of
  the control flow dependency: the residual function is basically a
  constant-folded copy of the source function, so it might retain some
  of the same control flow.  The value that needs to be stored in the
  fresh join variable thus depends on which branches are taken in the
  residual graph.

\end{itemize}

\subsubsection*{Hints}

Our goal in designing our approach to binding-time analysis was to
minimize the number of explicit hints that the user must provide in
the source of the RPython program.  This minimalism was not pushed to
extremes, though, to keep the hint-annotator reasonably simple.  

The driving idea was that hints should be need-oriented.  Indeed, in a
program like an interpreter, there are a small number of places where it
would be clearly beneficial for a given value to be known at
compile-time, i.e.\ green: this is where we require the hints to be
added.

The hint-annotator assumes that all variables are red by default, and
then propagates annotations that record dependency information.
When encountering the user-provided hints, the dependency information
is used to make some variables green.  All
hints are in the form of an operation \code{hint(v1, someflag=True)}
which semantically just returns its first argument unmodified.

The crucial need-oriented hint is
$$\code{v2 = hint(v1, concrete=True)}$$
which should be used in places where the programmer considers the
knowledge of the value to be essential.  This hint is interpreted by
the hint-annotator as a request for both \code{v1} and \code{v2} to be green.  It
has a \emph{global} effect on the binding times: it means that not only
\code{v1} but all the values that \code{v1} depends on -- recursively --
are forced to be green.  The hint-annotator complains if the
dependencies of \code{v1} include a value that cannot be green, like
a value read out of a field of a non-immutable structure.

Such a need-oriented backward propagation has advantages over the
commonly used forward propagation, in which a variable is compile-time
if and only if all the variables it depends on are also compile-time.  A
known issue with forward propagation is that it may mark as compile-time
either more variables than expected (which leads to over-specialization
of the residual code), or less variables than expected (preventing
specialization to occur where it would be the most useful).  Our
need-oriented approach reduces the problem of over-specialization, and
it prevents under-specialization: an unsatisfiable \code{hint(v1,
concrete=True)} is reported as an error.

In our context, though, such an error can be corrected.  This is done by
promoting a well-chosen variable among the ones that \code{v1} depends on.

Promotion is invoked with the use of a hint as well:
\code{v2 = hint(v1, promote=True)}.
This hint is a \emph{local} request for \code{v2} to be green, without
requiring \code{v1} to be green.  Note that this amounts to copying
a red value into a green one, which is not possible in classical
approaches to partial evaluation.  See section \ref{promotion} for a
complete discussion of promotion.

For examples and further discussion on how the hints are applied in practice
see \emph{Make your own JIT compiler} at
\code{http://codespeak.net/pypy/dist/pypy/doc/jit.html}. % XXX check url

\subsection{Timeshifting}

Once binding times (colors) have been assigned to all variables in a
family of control flow graphs, the next step is to mutate the graphs\footnote{
    One should keep in mind that the program described as the "source RPython
    program" in this document is typically an interpreter -- the canonical
    example is that it is the whole PyPy Standard Interpreter.  This
    program is meant to execute at run-time, and directly compute the
    intended result and side-effects. The translation process transforms
    it into a forest of flow graphs.  These are the flow graphs that
    timeshifting processes (and not the application-level program, which typically
    cannot be expressed as low-level flow graphs).
}
accordingly in order to produce a generating extension.  We call
this process \emph{timeshifting} because it changes the time at
which the graphs are meant to be run, from run-time to compile-time.

Despite the execution time and side-effects shift to produce only
residual code, the timeshifted graphs have a shape (flow of control)
that is closely related to that of the original graphs.  This is because
at compile-time the timeshifted graphs go over all the operations that
the original graphs would have performed at run-time, following the same
control flow; some of these operations and control flow constructs are
constant-folded at compile-time, and the rest is turned into equivalent
residual code.  Another point of view is that as the timeshifted graphs
form a generating extension, they perform the equivalent of an abstract
interpretation of the original graphs over a domain containing
compile-time values and run-time value locations.

The rest of this section describes this timeshifting process in more
detail.

\subsubsection*{Red and Green Operations}

The basic idea of timeshifting is to transform operations in a way that
depends on the color of their operands and result.  Variables themselves
need to be represented based on their color:
%
\begin{itemize}

\item The red (run-time) variables have abstract values at compile-time;
  no actual value is available for them during compile-time. For them
  we use a boxed representation that can carry either a run-time storage
  location (a stack frame position or a register name) or an immediate
  constant (for when the value is, after all, known at compile-time).

\item On the other hand, the green variables are the ones that can carry
  their value already at compile-time, so they are left untouched during
  timeshifting.

\end{itemize}

The operations of the original graphs are then transformed as follows:
%
\begin{itemize}

\item If an operation has no side effect nor any other run-time dependency, and
  if it only involves green operands, then it can stay unmodified in the
  graph.  In this case, the operation that was run-time in the original
  graph becomes a compile-time operation, and it will never be generated
  in the residual code.  (This is the case that makes the whole approach
  worthwhile: some operations become purely compile-time.)

\item In all other cases, the operation might have to be generated in the
  residual code.  In the timeshifted graph it is replaced by a call
  to a helper which will generate a residual operation manipulating
  the input run-time values and return a new boxed representation
  for the run-time result location.

\end{itemize}

These helpers will constant-fold the operation if the inputs
are immediate constants and if the operation has no side-effects. Immediate constants can occur even though the
corresponding variable in the graph was red: a variable can be
dynamically found to contain a compile-time constant at a particular
point in (compile)-time, independently of the hint-annotator
proving that it is always the case.
In Partial Evaluation terminology, the timeshifted graphs are
performing some \emph{on-line} partial evaluation in addition to the
off-line job enabled by the hint-annotator.

\subsubsection*{Merges and Splits}

The timeshifted code carries around an object that stores the
compilation-time state -- mostly the current bindings of the variables.
This state is used to shape the control flow of the generated residual
code, as follows.

After a \emph{split,} i.e.\ after a conditional branch that could not be
folded at compile-time, the compilation state is duplicated and both
branches are compiled independently.  Conversely, after a \emph{merge point,}
i.e.\ when two control flow paths meet each other, we try to join the two
paths in the residual code.  This part is more difficult because the two
paths may need to be compiled with different variable bindings --
e.g.\ different variables may be known to take different compile-time constant
values in the two branches.  The two paths can either be kept separate
or merged; in the latter case, the merged compilation-time state needs
to be a generalization \emph{(widening)} of the two already-seen states.
Deciding when to do each is a classical problem of partial evaluation,
as merging too eagerly may loose important precision and not merging
eagerly enough may create too many redundant residual code paths (to the
point of preventing termination of the compiler).

So far, we did not investigate this problem in detail.  We settled for a
simple widening heuristic: two different compile-time constants merge as
a run-time value, but we try to preserve the richer models of run-time
information that are enabled by the techniques described in the sequel
(promotion (\ref{promotion}), virtual structures (\ref{virtual})...).
This heuristic seems to work
for PyPy to some extent.

\subsubsection*{Calls and inlining}

For calls timeshifting can either produce code to generate a residual
call operation or recursively invoke the timeshifted version of the
callee.  The residual operations generated by the timeshifted callee
will grow the compile-time produced residual function; this
effectively amounts to the compile-time inlining of the original callee into
its caller. This is the default behaviour for calls within the
user-controlled subset of original graphs of the interpreter that are
timeshifted. Inlining only stops at re-entrant calls to the
interpreter main loop; the net result is that at the level of the
interpreted language, each function (or method) gets compiled into
a single piece of residual code.

\subsection{Promotion}
\label{promotion}

In the sequel, we describe in more details one of the main new
techniques introduced in our approach, which we call \emph{promotion.}  In
short, it allows an arbitrary run-time value to be turned into a
compile-time value at any point in time.  Each promotion point is
explicitly defined with a hint that must be put in the source code of
the interpreter.

From a partial evaluation point of view, promotion is the converse of
the operation generally known as "lift".  Lifting a value means
copying a variable whose binding time is compile-time into a variable
whose binding time is run-time -- it corresponds to the compiler
"forgetting" a particular value that it knew about.  By contrast,
promotion is a way for the compiler to gain \emph{more} information about
the run-time execution of a program. Clearly, this requires
fine-grained feedback from run-time to compile-time, thus a
dynamic setting.

Promotion requires interleaving compile-time and run-time phases,
otherwise the compiler can only use information that is known ahead of
time. It is impossible in the "classical" approaches to partial
evaluation, in which the compiler always runs fully ahead of execution
This is a problem in many large use cases.  For example, in an
interpreter for a dynamic language, there is mostly no information
that can be clearly and statically used by the compiler before any
code has run.

A more theoretical way to see the issue is to consider that the
possible binding time for each variable in the interpreter is
constrained by the binding time of the other variables it depends on.
For some kind of interpreters this set of constraints may have no
interesting global solution -- if most variables can ultimately depend
on a value, even in just one corner case, which cannot be
compile-time, then in any solution most variables will be run-time.
In the presence of promotion, though, these constraints can be
occasionally violated: corner cases do not necessarily have to
influence the common case, and local solutions can be patched
together.

A very different point of view on promotion is as a generalization of
techniques that already exist in dynamic compilers as found in modern
object-oriented language virtual machines.  In this context feedback
techniques are crucial for good results.  The main goal is to
optimize and reduce the overhead of dynamic dispatching and indirect
invocation.  This is achieved with variations on the technique of
polymorphic inline caches \cite{polymorphic-inline-caches}:
the dynamic lookups are cached and
the corresponding generated machine code contains chains of
compare-and-jump instructions which are modified at run-time.  These
techniques also allow the gathering of information to direct inlining for even
better optimization results.

In the presence of promotion, dispatch optimization can usually be
reframed as a partial evaluation task.  Indeed, if the type of the
object being dispatched to is known at compile-time, the lookup can be
folded, and only a (possibly inlined) direct call remains in the
generated code.  In the case where the type of the object is not known
at compile-time, it can first be read at run-time out of the object and
promoted to compile-time.  As we will see in the sequel, this produces
very similar machine code.\footnote{
    This can also be seen as a generalization of a partial
    evaluation transformation called "The Trick" (see e.g.\ \cite{partial-evaluation}),
    which again produces similar code but which is only
    applicable for finite sets of values.
}

The essential advantage is that it is no longer tied to the details of
the dispatch semantics of the language being interpreted, but applies in
more general situations.  Promotion is thus the central enabling
primitive to make timeshifting a practical approach to language
independent dynamic compiler generation.

\subsubsection*{Promotion in practice}

The implementation of promotion requires a tight coupling between
compile-time and run-time: a \emph{callback,} put in the generated code,
which can invoke the compiler again.  When the callback is actually
reached at run-time, and only then, the compiler resumes and uses the
knowledge of the actual run-time value to generate more code.

The new generated code is potentially different for each run-time value
seen.  This implies that the generated code needs to contain some sort
of updatable switch, which can pick the right code path based on the
run-time value.

While this describes the general idea, the details are open to slight
variations.  Let us show more precisely the way the JIT compilers
produced by PyPy 1.0 work.  Our first example is purely artificial:
%
\begin{verbatim}
    ...
    b = a / 10
    c = hint(b, promote=True)
    d = c + 5
    print d
    ...
\end{verbatim}

In this example, \code{a} and \code{b} are run-time variables and \code{c} and
\code{d} are compile-time variables; \code{b} is copied into \code{c} via a
promotion.  The division is a run-time operation while the addition is a
compile-time operation.

The compiler derived from an interpreter containing the above code
generates the following machine code (in pseudo-assembler notation),
assuming that \code{a} comes from register \code{r1}:
%
\begin{verbatim}
 ...
    r2 = div r1, 10
 Label1:
    jump Label2
    <some reserved space here>

 Label2:
    call continue_compilation(r2, <state data ptr>)
    jump Label1
\end{verbatim}

The first time this machine code runs, the function called
\code{continue\_compilation()}
resumes the compiler.  The two arguments to the function are
the actual run-time value from the register \code{r2}, which the compiler
will now consider as a compile-time constant, and an immediate pointer
to data that was generated along with the above code snippet and which
contains enough information for the compiler to know where and with
which state it should resume.

Assuming that the first run-time value taken by \code{r1} is, say, 42, then
the compiler will see \code{r2 == 4} and update the above machine code as
follows:
%
\begin{verbatim}
 ...
    r2 = div r1, 10
 Label1:
    compare r2, 4            # patched
    jump-if-equal Label3     # patched
    jump Label2              # patched
    <less reserved space left>

 Label2:
    call continue_compilation(r2, <state data ptr>)
    jump Label1

 Label3:                     # new code
    call print(9)            # new code
    ...
\end{verbatim}

Notice how the addition is constant-folded by the compiler.  (Of course,
in real examples, different promoted values typically make the compiler
constant-fold complex code path choices in different ways, and not just
simple operations.)  Note also how the code following \code{Label1} is an
updatable switch which plays the role of a polymorphic inline cache.
The "polymorphic" terminology does not apply in our context, though, as
the switch does not necessarily have to be on the type of an object.

After the update, the original call to \code{continue\_compilation()}
returns and execution loops back to the now-patched switch at
\code{Label1}.  This run and all following runs in which \code{r1} is between
40 and 49 will thus directly go to \code{Label3}.  Obviously, if other
values show up, \code{continue\_compilation()} will be invoked again, so new
code will be generated and the code at \code{Label1} further patched to
check for more cases.

If, over the course of the execution of a program, too many cases are
seen, the reserved space after \code{Label1} will eventually run out.
Currently, we simply reserve more space elsewhere and patch the final
jump accordingly.  There could be better strategies which which we did
not implement so far, such as discarding old code and reusing their slots
in the switch, or sometimes giving up entirely and compiling a general
version of the code in which the value remains run-time.


\subsubsection*{Implementation notes}

The state data pointer in the example above contains a snapshot of the
state of the compiler when it reached the promotion point.  Its memory
impact is potentially large -- a complete continuation for each generated
switch, which can never be reclaimed because new run-time values may
always show up later during the execution of the program.

To reduce the problem we compress the state into a so-called \emph{path.}
The full state is only stored at a few specific points.\footnote{
    More precisely, at merge points that the user needs to mark
    as "global".  The control flow join point corresponding to the
    looping of the interpreter main loop is a typical place to put
    such a global merge point.
}
The compiler
records a trace of the multiple paths it followed from the last full
snapshot in a lightweight tree structure.  The state data pointer is
then only a pointer to a node in the tree; the branch from that node to
the root describes a path that let the compiler quickly \emph{replay} its
actions (without generating code again) from the latest full snapshot to
rebuild its internal state and get back to the original promotion point.

For example, if the interpreter source code contains promotions inside a
run-time condition:
%
\begin{verbatim}
        if condition:
            ...
            hint(x, promote=True)
            ...
        else:
            ...
            hint(y, promote=True)
            ...
\end{verbatim}

then the tree will contain three nodes: a root node storing the
snapshot, a child with a "True case" marker, and another child with a
"False case" marker.  Each promotion point generates a switch and a call
to \code{continue\_compilation()} pointing to the appropriate child node.
The compiler can re-reach the correct promotion point by following the
markers on the branch from the root to the child.


\subsection{Virtual structures}
\label{virtual}

Interpreters for dynamic languages typically allocate a lot of small
objects, for example due to boxing.  For this reason, we
implemented a way for the compiler to generate residual memory
allocations as lazily as possible.  The idea is to try to keep new
run-time structures "exploded": instead of a single run-time pointer to
a heap-allocated data structure, the structure is "virtualized" as a set
of fresh variables, one per field.  In the compiler, the variable that
would normally contain the pointer to the structure gets instead a
content that is neither a run-time value nor a compile-time constant,
but a special \emph{virtual structure} -- a compile-time data structure that
recursively contains new variables, each of which can again store a
run-time, a compile-time, or a virtual structure value.

This approach is based on the fact that the "run-time values" carried
around by the compiler really represent run-time locations -- the name of
a CPU register or a position in the machine stack frame.  This is the
case for both regular variables and the fields of virtual structures.
It means that the compilation of a \code{getfield} or \code{setfield}
operation performed on a virtual structure simply loads or stores such a
location reference into the virtual structure; the actual value is not
copied around at run-time.

It is not always possible to keep structures virtual.  The main
situation in which it needs to be "forced" (i.e.\ actually allocated at
run-time) is when the pointer escapes to some non-virtual location like
a field of a real heap structure.

Virtual structures still avoid the run-time allocation of most
short-lived objects, even in non-trivial situations.  The following
example shows a typical case.  Consider the Python expression \code{a+b+c}.
Assume that \code{a} contains an integer.  The PyPy Python interpreter
implements application-level integers as boxes -- instances of a
\code{W\_IntObject} class with a single \code{intval} field.  Here is the
addition of two integers:
%
\begin{verbatim}
  def add(w1, w2):          # w1, w2 are instances
      value1 = w1.intval    # of W_IntObject
      value2 = w2.intval
      result = value1 + value2
      return W_IntObject(result)
\end{verbatim}

When interpreting the bytecode for \code{a+b+c}, two calls to \code{add()} are
issued; the intermediate \code{W\_IntObject} instance is built by the first
call and thrown away after the second call.  By contrast, when the
interpreter is turned into a compiler, the construction of the
\code{W\_IntObject} object leads to a virtual structure whose \code{intval}
field directly references the register in which the run-time addition
put its result.  This location is read out of the virtual structure at
the beginning of the second \code{add()}, and the second run-time addition
directly operates on the same register.

An interesting effect of virtual structures is that they play nicely with
promotion.  Indeed, before the interpreter can call the proper \code{add()}
function for integers, it must first determine that the two arguments
are indeed integer objects.  In the corresponding dispatch logic, we
have added two hints to promote the type of each of the two arguments.
This produces a compiler that has the following behavior: in the general
case, the expression \code{a+b} will generate two consecutive run-time
switches followed by the residual code of the proper version of
\code{add()}.  However, in \code{a+b+c}, the virtual structure representing
the intermediate value will contain a compile-time constant as type.
Promoting a compile-time constant is trivial -- no run-time code is
generated.  The whole expression \code{a+b+c} thus only requires three
switches instead of four.  It is easy to see that even more switches can
be skipped in larger examples; typically, in a tight loop manipulating
only integers, all objects are virtual structures for the compiler and
the residual code is theoretically optimal -- all type propagation and
boxing/unboxing occurs at compile-time.


\subsection{Virtualizable structures}
\label{virtualizable}

In the PyPy interpreter there are cases where structures cannot be
virtual -- because they escape, or are allocated outside the
JIT-generated code -- but where we would still like to keep the
"exploding" effect and carry the fields of the structure as local
variables in the generated code.

It is likely that the same problem occurs more generally in many
interpreters: the typical example is that of frame objects, which stores
among other things the value of the local variables of each function
invocation.  Ideally, the effect we would like to achieve is to keep the
frame object as a purely virtual structure, and the same for the array
or dictionary implementing the bindings of the locals.  Then each local
variable of the interpreted language can be represented as a separate
run-time value in the generated code, or be itself further virtualized
(e.g.\ as a virtual \code{W\_IntObject} structure as seen above).

The issue is that the frame object is sometimes built in advance by
non-JIT-generated code; even when it is not, it immediately escapes into
the global list of frames that is used to support the frame stack
introspection primitives that Python exposes.  In other words, the frame
object cannot be purely virtual because a pointer to it must be stored
into a global data structure (even though in practice most of frame
objects are deallocated without ever having been introspected).

To solve this problem, we introduced \emph{virtualizable structures,} a mix
between regular run-time structures and virtual structures.  A virtualizable structure is a
structure that exists at run-time in the heap, but that is
simultaneously treated as virtual by the compiler.  Accesses to the
structure from the code generated by the JIT are virtualized away,
i.e.\ don't involve run-time copying.  The trade-off is that in order
to keep both views synchronized, accesses to the run-time structure
from regular code not produced by the JIT needs to perform an extra
check.

Because of this trade-off, a hint needs to be inserted manually to mark
the classes whose instances should be implemented in this way -- the
class of frame objects, in the case of PyPy.  The hint is used by the
translation toolchain to add a hidden field to all frame objects, and to
translate all accesses to the object fields into low-level code that
first checks the hidden field.  This is the only case so far in which
the presence of the JIT compiler imposes a global change to the rest of
the program during translation.\footnote{
    This is not a problem per se, as it is anyway just a small
    extension to the translation framework, but it imposes a performance
    overhead to all code manipulating frame objects.  To mitigate this, we
    added a way to declare during RPython type inference that the
    indirection check is not needed in some parts of the code where we know
    that the frame object cannot have a virtual counterpart.
}

The hidden field is set when the frame structure enters JIT-generated
code, and cleared when it leaves.  When a recursive call to
non-JIT-generated code finds a structure with the field set, it invokes
a JIT-generated callback to perform the reading or updating of the field
from the point of view of its virtual structure representation.  The
actual fields in the heap structure are not used during this time.

The effect that can be obtained in this way is that although frame
objects are still allocated in the heap, most of them will always remain
essentially empty.  A pointer to these empty frames is pushed into and
popped off the global frame list, allowing the introspection mechanisms
to still work perfectly.


\subsection{Other implementation details}

We quickly mention below a few other features and implementation details
of the implementation of the JIT generation framework.  More information
can be found in the on-line documentation \cite{PyPy}.  % => ref to web site
%
\begin{itemize}

\item There are more user-specified hints available, like \emph{deep-freezing,}
  which marks an object as immutable in order to allow accesses to
  its content to be constant-folded at compile-time.

\item The compiler representation of a run-time value for a non-virtual
  structure may additionally remember that some fields are actually
  compile-time constants.  This occurs for example when a field is
  read from the structure at run-time and then promoted to compile-time.

\item In addition to virtual structures, lists and dictionaries can also be
  virtual.

\item Exception handling is achieved by inserting explicit operations into
  the graphs before they are timeshifted.  Most of these run-time
  exception manipulations are then virtualized away, by treating the
  exception state as virtual.

\item Timeshifting is performed in two phases: a first step transforms the
  graphs by updating their control flow and inserting pseudo-operations
  to drive the compiler; a second step (based on the RTyper \cite{D05.1})
  replaces all necessary operations by calls to support code.

\item The support code implements the generic behaviour of the compiler,
  e.g.\ the merge logic.  It is about 3500 lines of RPython code.  The
  rest of the hint-annotator and timeshifter is about 3800 lines of
  Python code.

\item The machine code backends (two so far, Intel IA32 and PowerPC) are
  about 3500 further lines of RPython code each.  There is a
  well-defined interface between the JIT compiler support code and the
  backends, making writing new backends relatively easy.  The unusual
  part of the interface is the support for the run-time updatable
  switches.

\end{itemize}


\section{Results}

The following test function is an example of purely arithmetic code
written in Python, which the PyPy JIT can run extremely fast:
%
\begin{verbatim}
   def f1(n):
       "Arbitrary test function."
       i = 0
       x = 1
       while i<n:
           j = 0
           while j<=i:
               j = j + 1
               x = x + (i&j)
           i = i + 1
       return x
\end{verbatim}

We measured the time required to compute \code{f1(2117)} on the following
interpreters:
%
\begin{itemize}

\item Python 2.4.4, the standard CPython implementation.

\item A version of pypy-c (our Python interpreter translated to a stand-alone
  executable via C) including a generated JIT compiled.

\item gcc 4.1.1 compiling the above function rewritten in C (which, unlike
  the other two, does not do any overflow checking on the arithmetic
  operations).

\end{itemize}

The relative results have been found to vary by 25\% depending on the
machine.  On our reference benchmark machine, a 4-cores Intel(R)
Xeon(TM) CPU 3.20GHz with 5GB of RAM, we obtained the following results
(the numbers in parenthesis are the slow-down ratio relative to the
unoptimized gcc compilation):

\begin{tabular}{|l|ll|}
\hline
Interpreter & \multicolumn{2}{|c|}{Seconds per call} \\
\hline
Python 2.4.4                            & 0.82   & (132x)   \\
Python 2.4.4 with Psyco 1.5.2           & 0.0062 & (1.00x)  \\
pypy-c with the JIT turned off          & 1.77   & (285x)   \\
pypy-c with the JIT turned on           & 0.0091 & (1.47x)  \\
gcc                                     & 0.0062 & (1x)     \\
gcc -O2                                 & 0.0022 & (0.35x)  \\
\hline
\end{tabular}

This table shows that the PyPy JIT is able to generate residual code
that runs within the same order of magnitude as an unoptimizing gcc.  It
shows that all the abstraction overhead has been correctly removed from
the residual code; the remaining slow-downs are only due to a suboptimal
low-level machine code generation backend.  We have thus reached our
goal of automatically generating a JIT whose performance is similar to
the hand-written Psyco without having its limitations.\footnote{
    As mentioned above, Psyco gives up compiling Python functions
    if they use constructs it does not support, and is not 100\%
    compatible with introspection of frames.  By construction the
    PyPy JIT does not have these limitations.  The PyPy JIT is
    also easier to retarget, and already supports more architectures
    than Psyco does, namely the Intel Mac OS/X and the PowerPC Mac OS/X.
}

In particular, the ratio of 1.47x between the unoptimizing gcc and the
PyPy JIT matches the target of 1.5x that we set ourselves as our goal
within the duration of the EU project.  We should also mention that on
an Intel-based Mac OS/X machine we have measured this ratio to be as low
as 1.15x.


\section{Future work}

Here are what we think are the most important points that will need
attention in order to make the approach more robust:
%
\begin{itemize}

\item The timeshifted graphs currently compile many branches eagerly.  This
  can easily result in residual code explosion.  Depending on the source
  interpreter this can also result in non-termination issues, where
  compilation never completes.  The opposite extreme would be to always
  compile branches lazily, when they are about to be executed, as Psyco
  does.  While this neatly sidesteps termination issues, the best
  solution is probably something in between these extremes.

\item As described in the Promotion section (\ref{promotion}),
  we need fall-back solutions for when the
  number of promoted run-time values seen at a particular point becomes
  too large.

\item We need more flexible control about what to inline or not to inline in
  the residual code.

\item The widening heuristics for merging needs to be refined.

\item The JIT generation framework needs to be made aware of some other
  translation-time aspects in order to produce the correct residual code
  (e.g.\ code calling the correct Garbage Collection routines or
  supporting Stackless-style stack unwinding \cite{D07.1}).

\item We did not work yet on profile-directed identification of program hot
  spots.  Currently, the interpreter must decide when to invoke the JIT
  or not (which can itself be based on explicit requests from the interpreted
  program).

\item The machine code backends can be improved.

\end{itemize}

The latter point opens an interesting future research direction: can we
layer our kind of JIT compiler on top of a virtual machine that already
contains a lower-level JIT compiler?  In other words, can we delegate
the difficult questions of machine code generation to a lower
independent layer, e.g.\ inlining, re-optimization of frequently executed
code, etc.?  What changes would be required to an existing virtual
machine, e.g.\ a Java Virtual Machine, to support this?


\section{Conclusion}

Producing the results described in the previous section requires the
generated compiler to completely cut the overhead and fold at
compile-time some rather involved lookup algorithms like Python's binary
operation dispatch.  Promotion proved itself to be sufficiently
powerful to achieve this.  Other features we introduced allowed to
preserve information about intermediate value types, to avoid their
boxing and to propagate them in the CPU stack and registers.

Some slight reorganisation of the interpreter main loop without semantics
influence, marking the frames as virtualizable
(section \ref{virtualizable}), and adding hints at
a few crucial points was all that was necessary for our Python
interpreter.

We think that our results make viable an approach to implement dynamic
languages that needs only a straight-forward bytecode interpreter to
be written. Dynamic compilers would be generated automatically guided
by the placement of hints.

These implementations should stay flexible and evolvable.  Dynamic
compiler would be robust against language changes up to the need to
maintain and possibly change the hints.

We consider this as a major breakthrough in term of the possibilities it
opens for language design and implementation; it was one of the main
goals of the research program within the PyPy project.  Only groups with
very large amounts of resources can affort the high costs of writing
just-in-time compilers from scratch.  Communities with limited available
resources for the implementation and maintenance of a language, such as,
generally, academic and open source projects, cannot afford such costs
-- and even when experimental just-in-time compilers exist, the mere
fact of having to maintain them in parallel with other implementations
is taxing for such communities, particularly when the languages in
question evolve quickly.  In the PyPy approach, from a single simple
implementation for the language, we can generate stand-alone virtual
machines whose performance far excess that of traditional hand-written
virtual machines (like CPython, the reference C implementation of
Python); with the generation of a dynamic compiler, we achieve
state-of-the-art performance.

% XXX balance columns


%.. References (title not necessary, latex generates it)
%
%.. [D05.1] `Compiling Dynamic Language Implementations`, PyPy EU-Report, 2005
%
%.. [D05.4] `Encapsulating Low-Level Aspects`, PyPy EU-Report, 2005
%
%.. [D07.1] `Support for Massive Parallelism, Optimisation results, Practical
%           Usages and Approaches for Translation Aspects`, PyPy EU-Report,
%           2006
%
%.. [D08.1] `Release a JIT Compiler for PyPy Including Processor Backends
%           for Intel and PowerPC`, PyPy EU-Report, 2007
%
%.. [FU]    `Partial evaluation of computation process -- an approach to a
%           compiler-compiler`, Yoshihito Futamura, Higher-Order and
%           Symbolic Computation, 12(4):363-397, 1999.  Reprinted from
%           Systems Computers Controls 2(5), 1971
%
%.. [PE]   `Partial evaluation and automatic program generation`,
%           Neil D. Jones, Carsten K. Gomard, Peter Sestoft,
%           Prentice-Hall, Inc., Upper Saddle River, NJ, 1993
%
%.. [REJIT] `Retargeting JIT Compilers by using C-Compiler Generated
%           Executable Code`, M. Anton Ertl, David Gregg, Proc. of 
%           the 13th Intl. Conf. on Parallel Architectures and 
%           Compilation Techniques, 2004.
%
%.. [PIC] `Optimizing Dynamically-Typed Object-Oriented Languages With
%         Polymorphic Inline Caches`, U. Hölzle, C. Chambers, D. Ungar,
%         ECOOP'91 Conference Proceedings, Geneva, 1991.
%
%.. [PSYCO] `Representation-based just-in-time specialization and the
%           psyco prototype for python`, Armin Rigo, in PEPM '04: Proceedings
%           of the 2004 ACM SIGPLAN symposium on Partial evaluation and
%           semantics-based program manipulation, pp. 15-26, ACM Press, 2004
%
%.. [VMCDLS]  `PyPy's approach to virtual machine construction`, Armin Rigo,
%           Samuele Pedroni, in OOPSLA '06: Companion to the 21st ACM SIGPLAN
%           conference on Object-oriented programming languages, systems, and
%           applications, pp. 944-953, ACM Press, 2006

% ---- Bibliography ----
\bibliographystyle{abbrv}
\bibliography{paper}

\end{document}
