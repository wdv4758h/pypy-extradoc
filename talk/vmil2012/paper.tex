\documentclass[10pt]{sigplanconf}

\usepackage{ifthen}
\usepackage{fancyvrb}
\usepackage{color}
\usepackage{wrapfig}
\usepackage{ulem}
\usepackage{xspace}
\usepackage{relsize}
\usepackage{epsfig}
\usepackage{amssymb}
\usepackage{amsmath}
\usepackage{amsfonts}
\usepackage[utf8]{inputenc}
\usepackage{setspace}

\usepackage{listings}

\usepackage[T1]{fontenc}
\usepackage[scaled=0.81]{beramono}


\definecolor{commentgray}{rgb}{0.3,0.3,0.3}

\lstset{
  basicstyle=\ttfamily\footnotesize,
  language=Python,
  keywordstyle=\bfseries,
  stringstyle=\color{blue},
  commentstyle=\color{commentgray}\textit,
  fancyvrb=true,
  showstringspaces=false,
  %keywords={def,while,if,elif,return,class,get,set,new,guard_class}
  numberstyle = \tiny,
  numbersep = -20pt,
}

\newboolean{showcomments}
\setboolean{showcomments}{true}
\ifthenelse{\boolean{showcomments}}
  {\newcommand{\nb}[2]{
    \fbox{\bfseries\sffamily\scriptsize#1}
    {\sf\small$\blacktriangleright$\textit{#2}$\blacktriangleleft$}
   }
   \newcommand{\version}{\emph{\scriptsize$-$Id: main.tex 19055 2008-06-05 11:20:31Z cfbolz $-$}}
  }
  {\newcommand{\nb}[2]{}
   \newcommand{\version}{}
  }

\newcommand\cfbolz[1]{\nb{CFB}{#1}}
\newcommand\toon[1]{\nb{TOON}{#1}}
\newcommand\anto[1]{\nb{ANTO}{#1}}
\newcommand\arigo[1]{\nb{AR}{#1}}
\newcommand\fijal[1]{\nb{FIJAL}{#1}}
\newcommand\pedronis[1]{\nb{PEDRONIS}{#1}}
\newcommand\bivab[1]{\nb{DAVID}{#1}}
\newcommand{\commentout}[1]{}

\newcommand{\noop}{}


\newcommand\ie{i.e.,\xspace}
\newcommand\eg{e.g.,\xspace}

\normalem

\let\oldcite=\cite

\renewcommand\cite[1]{\ifthenelse{\equal{#1}{XXX}}{[citation~needed]}{\oldcite{#1}}}

\definecolor{gray}{rgb}{0.5,0.5,0.5}

\begin{document}

\title{Efficiently Handling Guards in the Low Level Design of RPython's tracing JIT}

\authorinfo{David Schneider$^{a}$ \and Carl Friedrich Bolz$^a$}
           {$^a$Heinrich-Heine-Universität Düsseldorf, STUPS Group, Germany
           }
           {XXX emails}

\conferenceinfo{VMIL'12}{}
\CopyrightYear{2012}
\crdata{}

\maketitle

\category{D.3.4}{Programming Languages}{Processors}[code generation,
incremental compilers, interpreters, run-time environments]

\terms
Languages, Performance, Experimentation

\keywords{XXX}

\begin{abstract}

\end{abstract}


%___________________________________________________________________________
\section{Introduction}

In this paper we describe and analyze how deoptimization works in the context
of tracing just-in-time compilers. What instructions are used in the
intermediate and low-level representation of the JIT instructions and how these
are implemented.

Although there are several publications about tracing just-in-time compilers, to
our knowledge, there are none that describe the use and implementation of
guards in this context. With the following contributions we aim to shed some
light (to much?) on this topic.
The contributions of this paper are:
\begin{itemize}
 \item
\end{itemize}

The paper is structured as follows:

\section{Background}
\label{sec:Background}

\subsection{RPython and the PyPy Project}
\label{sub:pypy}


The RPython language and the PyPy Project were started in 2002 with the goal of
creating a Python interpreter written in a high level language, allowing easy
language experimentation and extension. PyPy is now a fully compatible
alternative implementation of the Python language\bivab{mention speed}. The
Implementation takes advantage of the language features provided by RPython
such as the provided tracing just-in-time compiler described below.

RPython, the language and the toolset originally developed to implement the
Python interpreter have developed into a general environment for experimenting
and developing fast and maintainable dynamic language implementations.
\bivab{Mention the different language impls}

RPython is built of two components, the language and the translation toolchain
used to transform RPython programs to executable units.  The RPython language
is a statically typed object oriented high level language. The language provides
several features such as automatic memory management (aka. Garbage Collection)
and just-in-time compilation. When writing an interpreter using RPython the
programmer only has to write the interpreter for the language she is
implementing.  The second RPython component, the translation toolchain, is used
to transform the program to a low level representations suited to be compiled
and run on one of the different supported target platforms/architectures such
as C, .NET and Java. During the transformation process
different low level aspects suited for the target environment are automatically
added to program such as (if needed) a garbage collector and with some hints
provided by the author a just-in-time compiler.



\subsection{PyPy's Meta-Tracing JIT Compilers}
\label{sub:tracing}

 * Tracing JITs
 * JIT Compiler
   * describe the tracing jit stuff in pypy
   * reference tracing the meta level paper for a high level description of what the JIT does
   * JIT Architecture
   * Explain the aspects of tracing and optimization

%___________________________________________________________________________

\begin{figure}
    \begin{lstlisting}[language=Python, numbers=right]
class Base(object):
    def __init__(self, n):
        self.value = n
    @staticmethod
    def build(n):
        if n & 1 == 0:
            return Even(n)
        else:
            return Odd(n)

class Odd(Base):
    def step(self):
        return Even(self.value * 3 + 1)

class Even(Base):
    def step(self):
        n = self.value >> 2
        if n == 1:
            return None
        return self.build(n)

def check_reduces(a):
    j = 1
    while j < 100:
        j += 1
        if a is None:
            return True
        a = a.step()
    return False
\end{lstlisting}

    \caption{Example Program}
    \label{fig:trace-log}
\end{figure}

\section{Resume Data}
\label{sec:Resume Data}

Since tracing linearizes control flow by following one concrete execution,
not the full control flow of a program is observed.
The possible points of deviation from the trace are guard operations
that check whether the same assumptions as during tracing still hold.
In later executions of the trace the guards can fail.
If that happens, execution needs to continue in the interpreter.
This means it is necessary to attach enough information to a guard
to construct the interpreter state when that guard fails.
This information is called the \emph{resume data}.

To do this reconstruction, it is necessary to take the values
of the SSA variables of the trace
and build interpreter stack frames.
Tracing aggressively inlines functions.
Therefore the reconstructed state of the interpreter
can consist of several interpreter frames.

If a guard fails often enough, a trace is started from it
to create a trace tree.
When that happens another use case of resume data
is to construct the tracer state.

There are several forces guiding the design of resume data handling.
Guards are a very common operations in the traces.
However, a large percentage of all operations
are optimized away before code generation.
Since there are a lot of guards
the resume data needs to be stored in a very compact way.
On the other hand, tracing should be as fast as possible,
so the construction of resume data must not take too much time.

\subsection{Capturing of Resume Data During Tracing}
\label{sub:capturing}

Every time a guard is recorded during tracing
the tracer attaches preliminary resume data to it.
The data is preliminary in that it is not particularly compact yet.
The preliminary resume data takes the form of a stack of symbolic frames.
The stack contains only those interpreter frames seen by the tracer.
The frames are symbolic in that the local variables in the frames
do not contain values.
Instead, every local variables contains the SSA variable of the trace
where the value would later come from, or a constant.

\subsection{Compression of Resume Data}
\label{sub:compression}

The core idea of storing resume data as compactly as possible
is to share parts of the data structure between subsequent guards.
This is often useful because the density of guards in traces is so high,
that quite often not much changes between them.
Since resume data is a linked list of symbolic frames
often only the information in the top frame changes from one guard to the next.
The other frames can often be just reused.
The reason for this is that during tracing only the variables
of the currently executing frames can change.
Therefore if two guards are generated from code in the same function
the resume data of the rest of the stack can be reused.

In addition to sharing as much as possible between subsequent guards
a compact representation of the local variables of symbolic frames is used.
Every variable in the symbolic frame is encoded using two bytes.
Two bits are used as a tag to denote where the value of the variable
comes from.
The remaining 14 bits are a payload that depends on the tag bits.

The possible source of information are:

\begin{itemize}
    \item For small integer constants
        the payload contains the value of the constant.
    \item For other constants
        the payload contains an index into a per-loop list of constants.
    \item For SSA variables,
        the payload is the number of the variable.
    \item For virtuals,
        the payload is an index into a list of virtuals, see next section.
\end{itemize}


\subsection{Interaction With Optimization}
\label{sub:optimization}

Guards interact with optimizations in various ways.
Most importantly optimizations try to remove as many operations
and therefore guards as possible.
This is done with many classical compiler optimizations.
In particular guards can be removed by subexpression elimination.
If the same guard is encountered a second time in the trace,
the second one can be removed.
This also works if a later guard is weaker implied by a earlier guard.

One of the techniques in the optimizer specific to tracing for removing guards
is guard strengthening~\cite{bebenita_spur:_2010}.
The idea of guard strengthening is that if a later guard is stronger
than an earlier guard it makes sense to move the stronger guard
to the point of the earlier, weaker guard and to remove the weaker guard.
Moving a guard to an earlier point is always valid,
it just means that the guard fails earlier during the trace execution
(the other direction is clearly not valid).

The other important point of interaction between resume data and the optimizer
is RPython's allocation removal optimization~\cite{bolz_allocation_2011}.
This optimization discovers allocations in the trace that create objects
that do not survive long.
An example is the instance of \lstinline{Even} in the example\cfbolz{reference figure}.
Allocation removal makes resume data more complex.
Since allocations are removed from the trace it becomes necessary
to reconstruct the objects that were not allocated so far when a guard fails.
Therefore the resume data needs to store enough information
to make this reconstruction possible.

Adding this additional information is done as follows.
So far, every variable in the symbolic frames
contains a constant or an SSA variable.
After allocation removal the variables in the symbolic frames can also contain
``virtual'' objects.
These are objects that were not allocated so far,
because the optimizer removed their allocation.
The virtual objects in the symbolic frames describe exactly
how the heap objects that have to be allocated on guard failure look like.
To this end, the content of every field of the virtual object is described
in the same way that the local variables of symbolic frames are described.
The fields of the virtual objects can therefore be SSA variables, constants
or other virtual objects.
They are encoded using the same compact two-byte representation
as local variables.

During the storing of resume data virtual objects are also shared
between subsequent guards as much as possible.
The same observation as about frames applies:
Quite often a virtual object does not change from one guard to the next.
Then the data structure is shared.

Similarly, stores into the heap are delayed as long as possible.
This can make it necessary to perform these delayed stores
when leaving the trace via a guard.
Therefore the resume data needs to contain a description
of the delayed stores to be able to perform them when the guard fails.
So far no special compression is done with this information.

% subsection Interaction With Optimization (end)

   * tracing and attaching bridges and throwing away resume data
   * compiling bridges
\bivab{mention that the failargs also go into the bridge}
% section Resume Data (end)

\begin{figure}
    \begin{lstlisting}[mathescape, numbers=right]
[$j_1$, $a_1$]
label($j_1$, $a_1$, descr=label0))
$j_2$ = int_add($j_1$, 1)
guard_nonnull_class($a_1$, Even)
$i_1$ = getfield_gc($a_1$, descr='value')
$i_2$ = int_rshift($i_1$, 2)
$b_1$ = int_eq($i_2$, 1)
guard_false($b_1$)
$i_3$ = int_and($i_2$, 1)
$i_4$= int_is_zero($i_3$)
guard_true($i_4$)
$b_2$ = int_lt($j_2$, 100)
guard_true($b_2$)

label($j_2$, $i_2$, descr=label1)
$j_3$ = int_add($j_2$, 1)
$i_5$ = int_rshift($i_2$, 2)
$b_3$ = int_eq($i_5$, 1)
guard_false($b_3$)
$i_6$ = int_and($i_5$, 1)
$b_4$ = int_is_zero($i_6$)
guard_true($b_4$)
$b_5$ = int_lt($j_3$, 100)
guard_true($b_5$)
jump($j_3$, $i_5$, descr=label1)
\end{lstlisting}

    \caption{Optimized trace}
    \label{fig:trace-log}
\end{figure}
% section Resume Data (end)
\section{Guards in the Backend}
\label{sec:Guards in the Backend}

After optimization the resulting trace is handed to the backend to be compiled
to machine code. The compilation phase consists of two passes over the lists of
instructions, a backwards pass to calculate live ranges of IR-level variables
and a forward one to emit the instructions. During the forward pass IR-level
variables are assigned to registers and stack locations by the register
allocator according to the requirements of the to be emitted instructions.
Eviction/spilling is performed based on the live range information collected in
the first pass. Each IR instruction is transformed into one or more machine
level instructions that implement the required semantics, operations withouth
side effects whose result is not used are not emitted. Guards instructions are
transformed into fast checks at the machine code level that verify the
corresponding condition.  In cases the value being checked by the guard is not
used anywhere else the guard and the operation producing the value can merged,
reducing even more the overhead of the guard. Figure \ref{fig:trace-compiled}
shows how an \texttt{int\_eq} operation followed by a guard that checks the
result of the operation are compiled to pseudo-assembler if the operation and
the guard are compiled separated or if they are merged.

\bivab{Figure needs better formatting}
\begin{figure}[ht]
  \noindent
  \centering
  \begin{minipage}{1\columnwidth}
    \begin{lstlisting}[mathescape]
$b_1$ = int_eq($i_2$, 1)
guard_false($b_1$)
    \end{lstlisting}
  \end{minipage}
  \begin{minipage}{.40\columnwidth}
    \begin{lstlisting}
CMP r6, #1
MOVEQ r8, #1
MOVNE r8, #0
CMP r8, #0
BEQ <bailout>
    \end{lstlisting}
  \end{minipage}
  \hfill
  \begin{minipage}{.40\columnwidth}
    \begin{lstlisting}
CMP r6, #1
BNE <bailout>
...
...
...
    \end{lstlisting}
  \end{minipage}
  \caption{Separated and merged compilation of operations and guards}
  \label{fig:trace-compiled}
\end{figure}

Each guard in the IR has attached to it a list of the IR-variables required to
rebuild the execution state in case the trace is left through the side-exit
corresponding to the guard. When a guard is compiled, additionally to the
condition check two things are generated/compiled. First a special data
structure called \emph{low-level resume data} is created that encodes the
information provided by the register allocator about where the values
corresponding to each IR-variable required by the guard will be stored when
execution reaches the code emitted for the corresponding guard. \bivab{go into
more detail here?!} This encoding needs to be as compact as possible to
maintain an acceptable memory profile.

\bivab{example for low-level resume data goes here}

Second a piece of code is generated for each guard that acts as a trampoline.
Guards are implemented as a conditional jump to this trampoline. In case the
condition checked in the guard fails execution and a side-exit should be taken
execution jumps to the trampoline. In the trampoline the pointer to the
\emph{low-level resume data} is loaded and jumps to generic bail-out handler
that is used to leave the compiled trace in case of a guard failure.

Using the encoded location information the bail-out handler reads from the
saved execution state the values that the IR-variables had  at the time of the
guard failure and stores them in a location that can be read by the fronted.
After saving the information the control is passed to the frontend signaling
which guard failed so the frontend can read the information passed and restore
the state corresponding to the point in the program.

As in previous sections the underlying idea for the design of guards is to have
a fast on-trace profile and a potentially slow one in the bail-out case where
the execution takes one of the side exits due to a guard failure. At the same
time the data stored in the backend needed to rebuild the state should be be
as compact as possible to reduce the memory overhead produced by the large
number of guards\bivab{back this}.

As explained in previous sections, when a specific guard has failed often enough
a new trace, referred to as a \emph{bridge}, starting from this guard is recorded and
compiled. When compiling bridges the goal is that future failures of the guards
that led to the compilation of the bridge should execute the bridge without
additional overhead, in particular the failure of the guard should not lead
to leaving the compiled code prior to execution the code of the bridge.

The process of compiling a bridge is very similar to compiling a loop.
Instructions and guards are processed in the same way as described above. The
main difference is the setup phase. When compiling a trace we start with a clean
slate. The compilation of a bridge is started from a state (register and stack
bindings) that corresponds to the state during the compilation of the original
guard. To restore the state needed to compile the bridge we use the encoded
representation created for the guard to rebuild the bindings from IR-variables
to stack locations and registers used in the register allocator.  With this
reconstruction all bindings are restored to the state as they were in the
original loop up to the guard.

Once the bridge has been compiled the trampoline method stub is redirected to
the code of the bridge. In future if the guard fails again it jumps to the code
compiled for the bridge instead of bailing out. Once the guard has been
compiled and attached to the loop the guard becomes just a point where
control-flow can split. The loop after the guard and the bridge are just
conditional paths. \cfbolz{maybe add the unpatched and patched assembler of the trampoline as well?}

%* Low level handling of guards
%   * Fast guard checks v/s memory usage
%   * memory efficient encoding of low level resume data
%   * fast checks for guard conditions
%   * slow bail out
%
% section Guards in the Backend (end)

%___________________________________________________________________________


\section{Evaluation}
\label{sec:evaluation}

The following analysis is based on a selection of benchmarks taken from the set
of benchmarks used to measure the performance of PyPy as can be seen
on\footnote{http://speed.pypy.org/}. The selection is based on the following
criteria \bivab{??}. The benchmarks were taken from the PyPy benchmarks
repository using revision
\texttt{ff7b35837d0f}\footnote{https://bitbucket.org/pypy/benchmarks/src/ff7b35837d0f}.
The benchmarks were run on a version of PyPy based on the
tag~\texttt{release-1.9} and patched to collect additional data about the
guards in the machine code
backends\footnote{https://bitbucket.org/pypy/pypy/src/release-1.9}. All
benchmark data was collected on a MacBook Pro 64 bit running Max OS
10.7.4 \bivab{do we need more data for this kind of benchmarks} with the loop
unrolling optimization disabled\bivab{rationale?}.

Figure~\ref{fig:ops_count} shows the total number of operations that are
recorded during tracing for each of the benchmarks on what percentage of these
are guards. Figure~\ref{fig:ops_count} also shows the number of operations left
after performing the different trace optimizations done by the trace optimizer,
such as xxx. The last columns show the overall optimization rate and the
optimization rate specific for guard operations, showing what percentage of the
operations was removed during the optimizations phase.

\begin{figure*}
    \include{figures/benchmarks_table}
    \caption{Benchmark Results}
    \label{fig:ops_count}
\end{figure*}

\bivab{should we rather count the trampolines as part of the guard data instead
of counting it as part of the instructions}

Figure~\ref{fig:backend_data} shows
the total memory consumption of the code and of the data generated by the machine code
backend for the different benchmarks mentioned above. Meaning the operations
left after optimization take the space shown in Figure~\ref{fig:backend_data}
after being compiled. Also the additional data stored for the guards to be used
in case of a bailout and attaching a bridge.
\begin{figure*}
    \include{figures/backend_table}
    \caption{Total size of generated machine code and guard data}
    \label{fig:backend_data}
\end{figure*}

Both figures do not take into account garbage collection. Pieces of machine
code can be globally invalidated or just become cold again. In both cases the
generated machine code and the related data is garbage collected. The figures
show the total amount of operations that are evaluated by the JIT and the
total amount of code and data that is generated from the optimized traces.

* Evaluation
   * Measure guard memory consumption and machine code size
   * Extrapolate memory consumption for guard other guard encodings
      * compare to naive variant
   * Measure how many guards survive optimization
   * Measure the of guards and how many of these ever fail

\section{Related Work}

\subsection{Guards in Other Tracing JITs}
\label{sub:Guards in Other Tracing JITs}

SPUR~\cite{bebenita_spur:_2010} is a tracing JIT compiler
for a C\# virtual machine.
It handles guards by always generating code for every one of them
that transfers control back to the unoptimized code.
Since the transfer code needs to reconstruct the stack frames
of the unoptimized code,
the transfer code is quite large.

\bivab{mention Gal et al.~\cite{Gal:2009ux} trace stitching}
and also mention \bivab{Dynamo's fragment linking~\cite{Bala:2000wv}} in
relation to the low-level guard handling.

LuaJIT, ...

% subsection Guards in Other Tracing JITs (end)

\subsection{Deoptimization in Method-Based JITs}
\label{sub:Deoptimization in Method-Based JITs}

Deoptimization in method-based JITs is used if one of the assumptions
of the code generated by a JIT-compiler changes.
This is often the case when new code is added to the system,
or when the programmer tries to debug the program.

Deutsch et. al.~\cite{deutsch_efficient_1984} describe the use of stack descriptions
to make it possible to do source-level debugging of JIT-compiled code.
Self uses deoptimization to reach the same goal~\cite{XXX}.
When a function is to be debugged, the optimized code version is left
and one compiled without inlining and other optimizations is entered.
Self uses scope descriptors to describe the frames
that need to be re-created when leaving the optimized code.
The scope descriptors are between 0.45 and 0.76 times
the size of the generated machine code.

Java Hotspot~\cite{paleczny_java_2001} contains a deoptimization framework that is used
for debugging and when an uncommon trap is triggered.
To be able to do this, Hotspot stores a mapping from optimized states
back to the interpreter state at various deoptimization points.
There is no discussion of the memory use of this information.

The deoptimization information of Hotspot is extended
to support correct behaviour
when scalar replacement of fields is done for non-escaping objects~\cite{kotzmann_escape_2005}.
The approach is extremely similar to how RPython's JIT handles virtual objects.
For every object that is not allocated in the code,
the deoptimization information contains a description
of the content of the fields.
When deoptimizing code, these objects are reallocated
and their fields filled with the values
described by the deoptimization information.
The paper does not describe any attempts to store this information compactly.


% subsection Deoptimization in Method-Based JITs (end)



\section{Conclusion}


\section*{Acknowledgements}

\bibliographystyle{abbrv}
\bibliography{zotero,paper}

\end{document}
