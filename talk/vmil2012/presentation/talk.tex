\documentclass[utf8x]{beamer}

% This file is a solution template for:

% - Talk at a conference/colloquium.
% - Talk length is about 20min.
% - Style is ornate.

\mode<presentation>
{
  \usetheme{Warsaw}
  % or ...

  %\setbeamercovered{transparent}
  % or whatever (possibly just delete it)
}


\usepackage[english]{babel}
\usepackage{listings}
\usepackage{ulem}
\usepackage{color}
\usepackage{alltt}

\usepackage[utf8x]{inputenc}


\newcommand\redsout[1]{{\color{red}\sout{\hbox{\color{black}{#1}}}}}

% or whatever

% Or whatever. Note that the encoding and the font should match. If T1
% does not look nice, try deleting the line with the fontenc.


\title{The Efficient Handling of Guards in the Design of RPython's Tracing JIT}

\author[David Schneider, Carl Friedrich Bolz]{David Schneider \and \emph{Carl Friedrich Bolz}}
% - Give the names in the same order as the appear in the paper.
% - Use the \inst{?} command only if the authors have different
%   affiliation.

\institute[Heinrich-Heine-Universität Düsseldorf]{
Heinrich-Heine-Universität Düsseldorf, STUPS Group, Germany
}

\date{2012 VMIL, 21st of October, 2012}
% - Either use conference name or its abbreviation.
% - Not really informative to the audience, more for people (including
%   yourself) who are reading the slides online


% If you have a file called "university-logo-filename.xxx", where xxx
% is a graphic format that can be processed by latex or pdflatex,
% resp., then you can add a logo as follows:




% Delete this, if you do not want the table of contents to pop up at
% the beginning of each subsection:
%\AtBeginSubsection[]
%{
%  \begin{frame}<beamer>
%    \frametitle{Outline}
%    \tableofcontents[currentsection,currentsubsection]
%  \end{frame}
%}


% If you wish to uncover everything in a step-wise fashion, uncomment
% the following command: 

%\beamerdefaultoverlayspecification{<+->}


\begin{document}

\begin{frame}
  \titlepage
\end{frame}

\section{Introduction}

\begin{frame}
  \frametitle{Tracing JITs Compile by Observing an Interpreter}
  \begin{itemize}
      \item VM contains both an interpreter and the tracing JIT compiler
      \item JIT works by observing and logging what the interpreter does
      \begin{itemize}
          \item for interesting, commonly executed code paths
          \item produces a linear list of operations (trace)
      \end{itemize}
      \item trace is optimized and then turned into machine code
  \end{itemize}
\end{frame}

\begin{frame}
  \frametitle{Guards}

\end{frame}

% this talk wants to go over a lot of details that are usually glossed over as
% "easy" when tracing JITs are introduced.

\begin{frame}
  \frametitle{Bridges}
\end{frame}

\begin{frame}
  \frametitle{RPython and PyPy}
\end{frame}

\begin{frame}
  \frametitle{Running Example}
\end{frame}

\section{High-Level}

\begin{frame}
  \frametitle{Symbolic Frame Capturing}
\end{frame}

\begin{frame}
  \frametitle{Symbolic Frame Compression}
\end{frame}

\begin{frame}
  \frametitle{Interaction with Optimization}
\end{frame}

\begin{frame}
  \frametitle{Emitting Guards}
\end{frame}

\begin{frame}
  \frametitle{Patching Guards for Bridges}
\end{frame}

\section{Evaluation}

%as in paper
%fancy graphs
%something about execution speed
% <demo, if there is time>
\end{document}
