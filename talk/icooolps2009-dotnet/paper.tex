%\documentclass{acm_proc_article-sp}
\documentclass{sig-alternate}


\usepackage{ifthen}
\usepackage{fancyvrb}
\usepackage{color}
\usepackage{ulem}
\usepackage{listings}

\lstset{language=Python,
        basicstyle=\scriptsize\ttfamily,
        keywordstyle=\bf, %\color{blue}, % I couldn't find a way to make chars both bold and tt
        frame=none,
        stringstyle=\color{blue},
        fancyvrb=true,
        xleftmargin=10pt,xrightmargin=10pt,
        showstringspaces=false}


\newboolean{showcomments}
\setboolean{showcomments}{false}
\ifthenelse{\boolean{showcomments}}
  {\newcommand{\nb}[2]{
    \fbox{\bfseries\sffamily\scriptsize#1}
    {\sf\small$\blacktriangleright$\textit{#2}$\blacktriangleleft$}
   }
   \newcommand{\version}{\emph{\scriptsize$-$Id: main.tex 19055 2008-06-05 11:20:31Z cfbolz $-$}}
  }
  {\newcommand{\nb}[2]{}
   \newcommand{\version}{}
  }

\newcommand\davide[1]{\nb{DAV}{#1}}
\newcommand\cfbolz[1]{\nb{CFB}{#1}}
\newcommand\anto[1]{\nb{ANTO}{#1}}
\newcommand\arigo[1]{\nb{AR}{#1}}
\newcommand{\commentout}[1]{}

\normalem

\let\oldcite=\cite

\renewcommand\cite[1]{\ifthenelse{\equal{#1}{XXX}}{[citation~needed]}{\oldcite{#1}}}

\begin{document}


%% \title{Automatic generation of JIT compilers for dynamic languages in
%%   .NET\thanks{This work has been partially supported by MIUR EOS DUE -
%%     Extensible Object Systems for Dynamic and Unpredictable Environments and
%%     by the EU-funded project: IST 004779 PyPy (PyPy: Implementing Python in
%%     Python).}}

\title{Faster than C\#: efficient implementation of dynamic languages on
  .NET\thanks{This work has been partially supported by MIUR EOS DUE -
    Extensible Object Systems for Dynamic and Unpredictable Environments and
    by the EU-funded project: IST 004779 PyPy (PyPy: Implementing Python in
    Python).}}

%% Alternative title: Faster than C\#: the future of dynamic languages on .NET


\numberofauthors{3}
\author{
\alignauthor Antonio Cuni\\
       \affaddr{DISI, University of Genova}\\
       \affaddr{Italy}\\
       \email{cuni@disi.unige.it}
\alignauthor Davide Ancona\\
       \affaddr{DISI, University of Genova}\\
       \affaddr{Italy}\\
       \email{davide@disi.unige.it}
\alignauthor Armin Rigo\\
       \email{arigo@tunes.org}
}
\conferenceinfo{ICOOOLPS}{'09 Genova, Italy}
\CopyrightYear{2009}
\crdata{978-1-60558-541-3/09/07}

\maketitle

\category{D.3.4}{Programming Languages}{Processors}[code generation,
incremental compilers, optimization, interpreters, run-time environments]

\begin{abstract}
The Common Language Infrastructure (CLI) is a virtual machine expressly
designed for implementing statically typed languages such as C\#, therefore
programs written in dynamically typed languages are typically much slower than C\# when executed on .NET.

Recent developments show that \emph{Just In Time} (JIT) compilers can exploit runtime type
information to generate quite efficient code.  Unfortunately, writing a JIT
compiler is far from being simple.  

In this paper we report our positive
experience with automatic generation of JIT compilers as supported by the PyPy
infrastructure, by focusing on JIT compilation for .NET.
Following this approach, we have in fact added a second layer of JIT compilation, by allowing dynamic generation of more efficient .NET bytecode, which
in turn can be compiled to machine code by the .NET JIT compiler.   

The main and novel contribution of this paper is to show that this
\emph{two-layers JIT} technique is effective, since programs written in dynamic languages 
can run on .NET as fast as (and in some cases even faster than) the equivalent C\# programs.

The practicality of the approach is demonstrated by showing some promising
experiments done with benchmarks written in a simple dynamic language.
\end{abstract}

\section{Introduction}
Implementing a dynamic language such as Python with a compiler rather than with an interpreter improves performances at the cost of
an increasing complexity. Furthermore, generating code for high level virtual machines like CLI or JVM enhances portability and inter-operability.

Writing a compiler that targets the CLI or JVM is easier than targeting a real CPU, but
it still requires a lot of work, as IronPython\footnote{\texttt{http://www.codeplex.com/IronPython}},
Jython\footnote{\texttt{http://www.jython.org/}} and JRuby\footnote{\texttt{http://jruby.codehaus.org/}} demonstrate.
Finally, if one really seeks for an efficent language implementation, \emph{Just In Time} (JIT) compilation needs
to be considered; only in this way  the compiler can exploit runtime type
information to generate quite efficient code. Note that JIT compilation
has not to be confused with lazy compilation of IronPython and Jython which is exploited to save memory, 
since in these cases no runtime type information is ever used to generate more efficient code.

Unfortunately, writing a JIT compiler is a very complex task.  
To make this task simpler, the solution proposed by PyPy \cite{RiBo07_223}  
is automatic generation of JIT compilers with the help of partial evaluation techniques: 
the user has only to provide an interpreter manually annotated with \emph{hints}
specifying how interpretation and JIT compilation has to be interleaved \cite{PyPyJIT09}.

More precisely, in this paper we focus on the ability of generating a JIT compiler able to emit code
for the .NET virtual machine. To our knowledge, this is the first experiment with an interpreter with
two \emph{layers} of JIT compilation, since, before being executed, the
emitted code is eventually compiled again by .NET's own JIT compiler.

The main contribution of this paper is to demonstrate that the idea of
\emph{JIT layering} can give good results, as dynamic languages can be even
faster than their static counterparts.

\subsection{Overview of PyPy}

The \emph{PyPy} project\footnote{\texttt{http://codespeak.net/pypy/}}
\cite{RigoPedroni06} was initially conceived to develop an implementation of Python which
could be easily portable and extensible without renouncing efficiency.
To achieve these aims, the PyPy implementation is based on a highly
modular design which allows high-level aspects
to be separated from lower-level implementation details.
The abstract semantics of Python is defined by an interpreter written
in a high-level language, called RPython \cite{AACM-DLS07}, which is in fact a subset of
Python where some dynamic features have been sacrificed to allow an
efficient translation of the interpreter to low-level code\footnote{But note that it's a full Python interpreter; RPython is only the
language in which this interpreter is written.}.

Compilation of the interpreter is implemented as a stepwise
refinement by means of a translation toolchain which performs type
analysis, code optimizations and several transformations aiming at 
incrementally providing implementation details such as memory management or the threading model.
The different kinds of intermediate codes  which are refined 
during the translation process are all represented by a collection of control flow graphs,
at several levels of abstractions.

Finally, the low-level control flow graphs produced by the toolchain
can be translated to executable code for a specific platform by a
corresponding backend.
Currently, three fully developed backends are available to produce
executable C/POSIX code, Java and CLI/.NET bytecode. 

Despite having been specifically developed for Python, the PyPy infrastructure
can in fact be used for implementing other languages. Indeed, there were
successful experiments of using PyPy to implement several other languages such
as Smalltalk \cite{BolzEtAl08}, JavaScript, Scheme and Prolog.

\commentout{
As suggested by Figure~\ref{pypy-fig}, a portable interpreter for a
generic language $L$  can be
easily developed once an interpreter for $L$ has been implemented in
RPython.
}

\subsection{PyPy and JIT-Generation}
\label{sec:jitgen}

One of the most important aspects that PyPy's translation toolchain can weave
in is the \emph{automatic generation of a JIT compiler}.  This section will
give a high-level overview of how the JIT-generation process works. More
details can be found in \cite{PyPyJIT} and \cite{PyPyJIT09}.

The main difference between the JIT compilers generated by PyPy and the
ones found in other projects like IronPython is that the latter compile
code at the method granularity: they can do little to optimize most of
the operations inside, because few assumptions can be made about the
types of the arguments and the
global state of the program.  The PyPy JITs, on the other hand, work at
a sub-method granularity, as described next.

When using PyPy, the first step is to write an interpreter for the chosen language.  Since it
must be fed to the translation toolchain, the interpreter has to be written in
RPython.  Then, to guide the process, we need to add few manual
annotations (called hints) to the interpreter, in order to teach the JIT generator which
information is important to know at compile-time.  
From these annotations, PyPy will statically generate an interpreter and a JIT
compiler in a single executable (here a .NET executable).

The interesting property of the generated JIT compiler is to delay the
compilation until it knows all the information needed to generate
efficient code.  In other words, at runtime, when the interpreter notice
that it is useful to compile a given piece of code, it sends it to the
JIT compiler; however, if at some point the JIT compiler does not know
about something it needs, it generates a \emph{callback} into itself and stops
execution.

Later, when the generated code is executed, the callback might be hit and the JIT
compiler is restarted again.  At this point, the JIT knows exactly the state
of the program and can exploit all this extra knowledge to generate highly
efficient code.  Finally, the old code is patched and linked to the newly
generated code, so that the next time the JIT compiler will not be invoked
again.  As a result, \textbf{runtime and compile-time are continuously
interleaved}. 

Potentially, the JIT compiler generates new code for each different run-time
value seen in variables it is interested in.
This implies that the generated code needs to contain some sort of updatable
switch, called \emph{flexswitch}, which can pick the right code path based on the
run-time value.  Typically, the value we switch on is the runtime dynamic type
of a value, so that the JIT compiler has all information needed to produce
very good code for that specific case.

Modifying the old code to link to the newly generated one is very challenging on
.NET, as the framework does not offer any primitive to do this.  Section
\ref{sec:clibackend} explains how it is possible to obtain this behaviour.

\section{CLI backend for Rainbow interpreter}

\subsection{Promotion on CLI}

Implementing promotion on top of CLI is not straightforward, as it needs to
patch and modify the already generated code, and this is not possible on .NET.

To solve, we do xxx and yyy etc. etc.

\section{Benchmarks}
\label{sec:benchmarks}

In section \ref{sec:tlc-properties}, we saw that TLC provides most of the
features that usually make dynamically typed language so slow, such as
\emph{stack-based interpreter}, \emph{boxed arithmetic} and \emph{dynamic lookup} of
methods and attributes.

In the following sections, we present some benchmarks that show how our
generated JIT can handle all these features very well.

To measure the speedup we get with the JIT, we run each program three times:

\begin{enumerate}
\item By plain interpretation, without any jitting.
\item With the JIT enabled: this run includes the time spent by doing the
  compilation itself, plus the time spent by running the produced code.
\item Again with the JIT enabled, but this time the compilation has already
  been done, so we are actually measuring how good is the code we produced.
\end{enumerate}

Moreover, for each benchmark we also show the time taken by running the
equivalent program written in C\#.\footnote{The sources for both TLC and C\#
  programs are available at:

  http://codespeak.net/svn/pypy/extradoc/talk/ecoop2009/benchmarks/}

The benchmarks have been run on a machine with an Intel Pentium 4 CPU running at
3.20 GHz and 2 GB of RAM, running Microsoft Windows XP and Microsoft .NET
Framework 2.0.

\subsection{Arithmetic operations}

To benchmark arithmetic operations between integers, we wrote a simple program
that computes the factorial of a given number.  The algorithm is
straightforward, thus we are not showing the source code.  The loop contains only three operations: one
multiplication, one subtraction, and one comparison to check if we have
finished the job.

When doing plain interpretation, we need to create and destroy three temporary
objects at each iteration.  By contrast, the code generated by the JIT does
much better.  At the first iteration, the classes of the two operands of the
multiplication are promoted; then, the JIT compiler knows that both are
integers, so it can inline the code to compute the result.  Moreover, it can
\emph{virtualize} (see Section \ref{sec:virtuals}) all the temporary objects, because they never escape from
the inner loop.  The same remarks apply to the other two operations inside
the loop.

As a result, the code executed after the first iteration is close to optimal:
the intermediate values are stored as \lstinline{int} local variables, and the
multiplication, subtraction and \emph{less-than} comparison are mapped to a
single CLI opcode (\lstinline{mul}, \lstinline{sub} and \lstinline{clt},
respectively).

Similarly, we wrote a program to calculate the $n_{th}$ Fibonacci number, for
which we can do the same reasoning as above.

\begin{table}[ht]
  \begin{center}

  \begin{tabular}{l|rrrrrr}
    \multicolumn{5}{c}{\textbf{Factorial}} \\ [0.5ex]

    \textbf{$n$}          & $10$  & $10^7$           & $10^8$         & $10^9$         \\
    \hline
    \textbf{Interp}       & 0.031 & 30.984           & N/A            & N/A            \\
    \textbf{JIT}          & 0.422 &  0.453           & 0.859          & 4.844          \\
    \textbf{JIT 2}        & 0.000 &  0.047           & 0.453          & 4.641          \\
    \textbf{C\#}          & 0.000 &  0.031           & 0.359          & 3.438          \\
    \textbf{Interp/JIT 2} & N/A   & \textbf{661.000} & N/A            & N/A            \\
    \textbf{JIT 2/C\#}    & N/A   & \textbf{1.500}   & \textbf{1.261} & \textbf{1.350} \\ [3ex]


    \multicolumn{5}{c}{\textbf{Fibonacci}} \\ [0.5ex]

    \textbf{$n$}          & $10$  & $10^7$           & $10^8$         & $10^9$         \\
    \hline
    \textbf{Interp}       & 0.031 & 29.359           & 0.000          & 0.000          \\
    \textbf{JIT}          & 0.453 &  0.469           & 0.688          & 2.953          \\
    \textbf{JIT 2}        & 0.000 &  0.016           & 0.250          & 2.500          \\ 
    \textbf{C\#}          & 0.000 &  0.016           & 0.234          & 2.453          \\
    \textbf{Interp/JIT 2} & N/A   & \textbf{1879.962}& N/A            & N/A            \\
    \textbf{JIT 2/C\#}    & N/A   & \textbf{0.999}   & \textbf{1.067} & \textbf{1.019} \\
  \end{tabular}

  \end{center}
  \caption{Factorial and Fibonacci benchmarks}
  \label{tab:factorial-fibo}
\end{table}


Table \ref{tab:factorial-fibo} shows the seconds spent to calculate
the factorial and Fibonacci for various $n$.  As we can see, for small values
of $n$ the time spent running the JIT compiler is much higher than the time
spent to simply interpret the program.  This is an expected result
which, however, can be improved once we will have time
to optimize compilation and not only the generated code.

On the other, for reasonably high values of $n$ we obtain very good
results, which are valid despite the obvious overflow, since the 
same operations are performed for all experiments.
For $n$ greater than $10^7$, we did not run the interpreted program as it would have took too
much time, without adding anything to the discussion.

As we can see, the code generated by the JIT can be up to about 1800 times faster
than the non-jitted case.  Moreover, it often runs at the same speed as the
equivalent program written in C\#, being only 1.5 slower in the worst case.

The difference in speed it is probably due to both the fact that the current
CLI backend emits slightly non-optimal code and that the underyling .NET JIT
compiler is highly optimized to handle bytecode generated by C\# compilers.

As we saw in Section~\ref{sec:flexswitches-cli}, the implementation of
flexswitches on top of CLI is hard and inefficient.  However, our benchmarks
show that this inefficiency does not affect the overall performances of the
generated code.  This is because in most programs the vast majority of the
time is spent in the inner loop: the graphs are built in such a way that all
the blocks that are part of the inner loop reside in the same method, so that
all links inside are internal (and fast).


\subsection{Object-oriented features}

To measure how the JIT handles object-oriented features, we wrote a very
simple benchmark that involves attribute lookups and polymorphic method calls.
Since the TLC assembler source is long and hard to read,
figure~\ref{fig:accumulator} shows the equivalent program written in an
invented Python-like syntax.

\begin{figure}[h]
\begin{center}
\begin{lstlisting}
def main(n):
    if n < 0:
        n = -n
        obj = new(value, accumulate=count)
    else:
        obj = new(value, accumulate=add)
    obj.value = 0
    while n > 0:
        n = n - 1
        obj.accumulate(n)
    return obj.value

def count(x):
    this.value = this.value + 1

def add(x):
    this.value = this.value + x
\end{lstlisting}
\caption{The \emph{accumulator} example, written in a invented Python-like syntax}
\label{fig:accumulator}
\end{center}
\end{figure}

The two \lstinline{new} operations create an object with exactly one field
\lstinline{value} and one method \lstinline{accumulate}, whose implementation
is found in the functions \lstinline{count} and \lstinline{add}, respectively.
When calling a method, the receiver is implicity passed and can be accessed
through the special name \lstinline{this}.

The computation \emph{per se} is trivial, as it calculates either $-n$ or
$1+2...+n-1$, depending on the sign of $n$. The interesting part is the
polymorphic call to \lstinline{accumulate} inside the loop, because the interpreter has
no way to know in advance which method to call (unless it does flow analysis,
which could be feasible in this case but not in general).  The equivalent C\#
code we wrote uses two classes and a \lstinline{virtual} method call to
implement this behaviour.

As already discussed, our generated JIT does not compile the whole function at
once. Instead, it compiles and executes code chunk by chunk, waiting until it
knows enough informations to generate highly efficient code.  In particular,
at the time it emits the code for the inner loop it exactly knows the
type of \lstinline{obj}, thus it can remove the overhead of dynamic dispatch
and inline the method call.  Moreover, since \lstinline{obj} never escapes the
function, it is \emph{virtualized} and its field \lstinline{value} is stored
as a local variable.  As a result, the generated code turns out to be a simple loop
doing additions in-place.

\begin{table}[ht]
  \begin{center}

  \begin{tabular}{l|rrrrrr}
    \multicolumn{5}{c}{\textbf{Accumulator}} \\ [0.5ex]

    \textbf{$n$}          & $10$  & $10^7$           & $10^8$         & $10^9$         \\
    \hline
    \textbf{Interp}       & 0.031 & 43.063           & N/A            & N/A            \\
    \textbf{JIT}          & 0.453 &  0.516           & 0.875          & 4.188          \\
    \textbf{JIT 2}        & 0.000 &  0.047           & 0.453          & 3.672          \\
    \textbf{C\#}          & 0.000 &  0.063           & 0.563          & 5.953          \\
    \textbf{Interp/JIT 2} & N/A   & \textbf{918.765} & N/A            & N/A            \\
    \textbf{JIT 2/C\#}    & N/A   & \textbf{0.750}   & \textbf{0.806} & \textbf{0.617} \\

  \end{tabular}
  \end{center}
  \caption{Accumulator benchmark}
  \label{tab:accumulator}
\end{table}





Table \ref{tab:accumulator} show the results for the benchmark.  Again, we can
see that the speedup of the JIT over the interpreter is comparable to the
other two benchmarks.  However, the really interesting part is the comparison
with the equivalent C\# code, as the code generated by the JIT is up to 1.62 times
\textbf{faster}.

Probably, the C\# code is slower because:

\begin{itemize}
\item The object is still allocated on the heap, and thus there is an extra
  level of indirection to access the \lstinline{value} field.
\item The method call is optimized through a \emph{polymorphic inline cache}
  \cite{hoelzle_optimizing_1991}, that requires a guard check at each iteration.
\end{itemize}

Despite being only a microbenchmark, this result is very important as it proves
that our strategy of intermixing compile time and runtime can yield to better
performances than current techniques.  The result is even more impressive if
we take in account that dynamically typed languages as TLC are usually considered much
slower than the statically typed ones.

\section{Related Work}

Flexswitches are closely related to the concept of \emph{promotion}, as
described by \cite{PyPyJIT}, \cite{PyPyJIT09}.
Psyco is
a run-time specialiser for Python that uses promotion (called ``unlift'' in
\cite{DBLP:conf/pepm/Rigo04}). However, Psyco is a manually written JIT, is
not applicable to other languages and cannot be retargetted.  Psyco is a 
good example of how to implement flexswitches for targets that don't have the
limitations of the CLI.

The idea of promotion is a generalization of \emph{Polymorphic
  Inline Caches} \cite{hoelzle_optimizing_1991}, as well as the idea of using
runtime feedback to produce more efficient code
\cite{hoelzle_type_feedback_1994}.  The main difference between the two is 
that PICs only works on types, whereas promotion can work on every kind of value.

PyPy-style JIT compilers are hard to write manually, thus we chose to write a
JIT generator.  Tracing JIT compilers \cite{gal_hotpathvm_2006} also give
good results but are much easier to write, making the need for an automatic
generator less urgent.  However so far tracing JITs have less general
allocation removal techniques, which makes them get less speedup in a dynamic
language with boxing.  Another difference is that tracing JITs concentrate on
loops, which makes them produce a lot less code.  This issue is being addressed
by current research in PyPy \cite{PyPyTracing}.

The code generated by tracing JITs code typically contains guards; in recent research
\cite{gal_incremental_2006} on Java, these guards' behaviour is extended to be
similar to our promotion.  This has been used twice to implement a dynamic
language (JavaScript), by Tamarin\footnote{{\tt
http://www.mozilla.org/projects/tamarin/}} and in \cite{chang_efficient_2007}.

IronPython and Jython are two popular implementations of Python for,
respectively, the CLI and the JVM, whose approach differs fundamentally from
PyPy.  The source code of PyPy contains a Python interpreter, which the JIT
compiler is automatically generated from: the resulting executable contains
both the interpreter and the compiler, so that it is possible to compile only
the desired parts of the program.  On the other hand, both IronPython and
Jython implements only the compiler: both compile code lazily (when a Python
module is loaded), but since they do not exploit the extra information
potentially available at runtime, it is more a delayed static compilation than
a true JIT one.  As a result, they run Python programs much slower than their
equivalent written in
C\#\footnote{\texttt{http://shootout.alioth.debian.org/gp4/\\benchmark.php?test=all\&lang=iron\&lang2=csharp}}
or
Java\footnote{\texttt{http://blog.dhananjaynene.com/2008/07/performance-\\comparison-c-java-python-ruby-jython-jruby-groovy/}}.

The \emph{Dynamic Language Runtime}\footnote{\texttt{http://www.codeplex.com/dlr}}
(DLR) is a library designed to ease the implementation of dynamic languages
for .NET: the DLR is closely related to IronPython\footnote{In fact, the DLR
  started as a spin-off of IronPython, and nowadays the latter is based on the
  former.} and employs the techniques described above; thus, the remarks
about the differences between PyPy and IronPython apply to all DLR based
languages.

\section{Conclusion and Future Work}

In this paper we gave an overview of PyPy's JIT compiler generator,
which can automatically turn an interpreter into a JIT
compiler, requiring the language developers to only add few hints to
guide the generation process.

Then, we presented the CLI backend for PyPy's JIT compiler generator, whose
goal is to produce .NET bytecode at runtime.  We showed how it is possible to
circumvent intrinsic limitations of the virtual machine to implement
flexswitches.  As a result, we proved that the idea of \emph{JIT layering} is
worth of further exploration, as it makes possible for dynamically typed
languages to be even faster than their statically typed counterpart in some
cases.

As a future work, we want to explore different strategies to make the frontend
producing less code, maintaining comparable or better performances.  In
particular, we are working on a way to automatically detect loops in the user
code, as tracing JITs do \cite{gal_hotpathvm_2006}.  By compiling whole
loops at once, the backends should be able to produce better code.

At the moment, some bugs and minor missing features prevent the CLI JIT
backend to handle more complex languages such as Python and Smalltalk.  We are
confident that once these problems will be fixed, we will get performance
results comparable to TLC, as the other backends already demonstrate
\cite{PyPyJIT}.  However, if the current implementation of flexswitches will
turn out to be too slow for some purposes, alternative
implementation strategies could be explored by considering the novel features 
offered the new generation of virtual machines.

In particular, the \emph{Da Vinci Machine
  Project} \footnote{\texttt{http://openjdk.java.net/projects/mlvm/}} is exploring and
implementing new features to ease the implementation of dynamic languages on
top of the JVM: some of these features, such as the new
\emph{invokedynamic}\footnote{\texttt{http://jcp.org/en/jsr/detail?id=292}} instruction and the \emph{tail call
  optimization} can probably be exploited by a potential JVM backend to
generate even more efficient code.



\section*{Acknowledgements}

The authors would like to thank Carl Friedrich Bolz, Maciej Fijalkowski and
the referees of ICOOOLPS'09 for helpful comments on earlier versions of this
paper.






\bigskip

\bibliographystyle{abbrv}
\bibliography{paper}

\end{document}

% LocalWords:  JIT PyPy
